\section{Introduction}
\PARstart{D}{ifferential} radar interferometry\cite{Gabriel1989, Massonnet1993,Rosen2000,Bamler1999} is widely used  to monitor and study changes in the natural and built environment. The ability to accurately measure movements along the line of sight over large areas makes them suitable for many  applications. Some examples are the estimation of subsidence rate associated with groundwater or oil extraction and the study of aquifer dynamics\cite{Hudnut1998,Strozzi2001,Galloway2007a}, the monitoring of inflation/deflation connected to volcanic activity\cite{Massonnet1995}, the mapping of ice sheet and glacier motion\cite{Goldstein1993,Mohr1998} , the observation of landslides and instable slopes\cite{Carnec1996,Catani2005} and the measurement of seismic displacements\cite{Massonnet1993b,Zebker1994}.\\
Fully polarimetric radar data provides additional information on the scattering mechanism within each resolution cell, which are employed for classification of the surface cover\cite{Cloude1997, Lee1999}, to extract geophysical parameters such as moisture content\cite{Hajnsek2003} or information on the vegetation \cite{Ulaby1987} or the height of fresh snow\cite{Leinss2014}.\\
The combination of polarimetric and differential interferometric measurement could bring additional benefits for the observation of natural processes; for example by selecting the scattering mechanism that provides the best coherence and thus the least noisy phase measurement\cite{Pipia2009a, Iglesias2014b}.\\
Currently, the majority of radar data employed for differential interferometry is acquired using sensors carried by satellite or aircrafts. These platform are convenient in that they offer a large coverage in a single pass. Nevertheless, because of costs, technical, physical and logistical restrictions, the revisit time of these system is limited to few hours at best. 
In many cases, to better understand the dynamics of natural processes and for real time surveillance and alarming, a denser temporal sampling over longer time spans is desired than the ones afforded by current radar earth observation systems.\\
\subsection{State of the Art}
Several ground based radar systems were developed for these situations, operating in C \cite{Leva2003, Rudolf1999a,Kang2009}, X \cite{Aguasca2004,Pipia2007a} or Ku band \cite{Leva2003, Rudolf1999a,Werner2008, Rodelsperger2012}. The majority of these systems are based on aperture synthesis, using a moving antenna assembly on a motorized rail. 
An alternative to aperture synthesis is to rotationally scan an electrically large fan beam antenna, resolving  scatterers by using its narrow azimuth beam\cite{Werner2008,werner_gpri_2012}. This imaging method has been called type II in \cite{Caduff2015}. 
This configuration has certain advantages over ground-based synthetic aperture systems\cite{Monserrat2014} of comparable rail length: \begin{enumerate*}
  \item Unlike a SAR system, each azimuth sample is 
acquired independently, eliminating the problem of decorrelation caused by moving targets and atmospheric phase screens during the acquisition time. 
These changes destroy the coherence of the scene over the aperture length and will spread these scatterer in azimuth, with a severe impact on image quality, especially regarding the stability of coherent targets.\\
 \item A very large 
angular section can be imaged in a single pass, potentially up to $360^{\circ}$. This is harder to achieve with a rail-based synthetic aperture radar, where usually only a smaller angular sector
can be covered at once.\\
\item An electrically large antenna provides an higher gain and hence a better SNR; this is not the case for  aperture synthesis  where the antenna has to be electrically small to provide a sufficiently wide beam.\\
\end{enumerate*}\\
The majority of polarimetric ground based radar system is based on the aperture synthesis principle. An indoor system is presented in \cite{Bennett1996} followed by a portable outdoors version \cite{Bennett2000}. A broadband polarimetric SAR system with two dimensional aperture synthesis is introduced in \cite{Zhou2004}, with measurement results presented in \cite{Hamasaki2005}.
Another example  of a synthetic aperture system is is UPCs RiskSAR \cite{Iglesias2014, Aguasca2004,Pipia2007a,Pipia2009, Pipia2013,Iglesias2014}. Example of ground based SAR polarimetric data at X and C band are shown in \cite{Kang2009, Kang2010}.\\ A dual polarization, multiband GB-SAR system is used in \cite{Yitayew2014} to produce tomograms of snow covered sea ice. A similar concept is used in\cite{Frey2015,Frey2016} to produce full polarimetric tomograms of a snow pack by synthesizing an aperture in the elevation direction.\\
Much less research is available on real aperture polarimetric systems, excluding non-imaging systems such as ground based scatterometers. One of the few examples is a C-Band version of the GPRI\cite{Cherukumilli2012}.
\subsection{KAPRI: Real Aperture Polarimetric FMCW Radar}
This paper will focus on KAPRI, the Ku-Band Polarimetric Advanced Radar Interferometer (KAPRI). It is an extension of GPRI \cite{werner_gpri_2012,Strozzi2011, Werner2008}, a portable real-aperture, $K_u$ band radar interferometer operating at 17.2 GHz with a wavelength of 1.74 cm, designed 
for the monitoring of unstable slopes using zero baseline differential interferometry \cite{Massonnet1993}. It is equipped with two antennas arranged along a spatial baseline and a dual channel receiver to derive local digital elevation models using single pass interferometric methods.\\
It employs  2 meter long, vertically polarized slotted waveguide antennas, the system an azimuth resolution of $0.4^\circ$ and an elevation beamwidth of $40^\circ$.\\
In terms of hardware, the feature distinguishing KAPRI from its predecessor is the addition of horizontally polarized antennas and switches that permit to connect transmitter and receiver to either type of antenna. Together with modifications in the control software, they enable it to acquire a full polarimetric-interferometric dataset by cycling through all the combinations of transmitted and received polarization during the acquisition.
\subsection{Goal of This Paper}
In this paper,the pre-processing  and polarimetric calibration of KAPRI/GPRI data is addressed. In particular, two main aspects of the processing chain are discussed in \autoref{sec:methods}:
\begin{enumerate}
	\item Processing of samples as acquired by the digitizer into single look complex (SLC) data. (\autoref{sec:proc_SLC})
	\item Polarimetric calibration of SLC data. (\autoref{sec:proc_polcal})
\end{enumerate}
The discussion of the processing methods is split into parts that correspond to subsequent stages in the evolution of GPRI-II into KAPRI. Part \ref{sec:proc_SLC} presents  methods employed to process raw data into SLC form and that are used by both systems. This includes a derivation of the FMCW signal model and the correction of frequency-dependent beam squint. This discussion also forms the basis for \ref{sec:proc_polcal}; properly focused SLCs being a prerequisite for polarimetric calibration.  \ref{sec:proc_polcal} addresses issues specific to KAPRI, especially the problems arising when the antennas for the different polarizations are not collocated and the correction of phase and amplitude imbalances.\\
In \autoref{sec:results}, the methods are applied on data acquired during a calibration campaign, where a number of trihedral corner reflectors (TCR) were used as reference targets. The results are discussed in two separate parts, reflecting the structure of \autoref{sec:methods}. In \autoref{sec:res_SLC} the effect of antenna beam-squint correction is illustrated by analyzing the oversampled range compressed response of a TCR alongside the corresponding envelope in the slow-fast time plane. Because an unexplained azimuthal phase variation is observed in the trihedral reflector responses, a model explaining it in terms of a phase center displacement is introduced in \autoref{sec:azimuth_processing} and an additional correction step is presented. In section \ref{sec:misalingment}, the correction of an azimuthal shift between the polarimetric channels is discussed and its effect on crosspolar measurements are shown by analyzing the response of a dihedral reflector.\\
The calibration methods of \ref{sec:proc_polcal} are then applied  and the quality of the calibration is assessed by plotting polarimetric signatures and computing the residual calibration parameters on the calibrated data.\\ This way of structuring the results section is nonstandard, but the authors feel that it better reflects the iterations that led to the final processing strategy.\\
In \autoref{sec:conclusions} the methods are summarized again and some conclusions are drawn.