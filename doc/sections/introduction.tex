\section{Introduction}
\PARstart{D}{ifferential} radar interferometry\cite{Gabriel1989, Massonnet1993} techniques widely used  to monitor and study changes in the natural and built environment. The ability to accurately measure movements along the line of sight over large areas makes them suitable for many applications. Some examples are the estimation of subsidence rate associated with groundwater or oil extraction and the study of aquifer dynamics\cite{Fielding1998,Strozzi2001,Galloway2007a}, the monitoring of inflation/deflation connected to volcanic activity\cite{Massonnet1995}, the mapping of ice sheet and glacier motion\cite{Goldstein1993,Mohr1998} , the observation of landslides and instable slopes\cite{Carnec1996,Catani2005} and the measurement of seismic displacements\cite{Massonnet1993b,Zebker1994}.\\
Full polarimetric radar data provides additional information on the type of scatterers contained in each resolution cells, permitting a classification of the surface cover\cite{Cloude1997, Lee1999} and to extract geophysical parameters such as moisture content\cite{Hajnsek2003} or informations on the vegetation cover\cite{Ulaby1987} or the height of fresh snow\cite{Leinss2014}.\\
The combination of polarimetric and differential interferometric measurement could bring additional benefits for the observation of natural processes; for example by selecting the polarization that provides the best coherence and thus the least noisy phase measurement\cite{Pipia2009a, Iglesias2014b}.\\
Currently, the majority of radar data employed for differential interferometry is acquired using sensors carried by satellite or aircrafts. These platform are convenient in that they offer a large coverage in a single pass. Nevertheless, because of costs, technical, physical and logistical restrictions, the revisit time of these system is limited to a few hours at best. 
In many cases, to better understand the dynamics of natural processes and for real time surveillance and alarming, a denser temporal sampling over longer time spans is desired than the ones afforded by current airborne and spaceborne systems.\\
Several Ground based radar systems were developed for these situations, mostly operating in the centimeter and millimeter wave region. A few examples exist of ground based radars designed  for polarimetric imaging\cite{Iglesias2014,LEE, Kang2009}.\\
The majority of these systems\cite{Rudolf1999a, Rodelsperger2011, Aguasca2004,Rodelsperger2012} are based on aperture synthesis using a moving antenna assembly on a short motorized rail.
\subsection{KAPRI: Real Aperture Polarimetric FMCW Radar}
An alternative to aperture synthesis is to rotationally scan a fan beam antenna, resolving the scatterers by using the narrow beam produced by an electrically large antenna \cite{Werner2008}. This imaging method has been called type II in \cite{Caduff2015}. 
This configuration has two advantages over ground-based synthetic aperture systems\cite{Monserrat2014} of comparable aperture length: \begin{enumerate*}
  \item unlike a SAR system, each azimuth line is 
acquired independently, eliminating the problem of decorrelation caused by moving targets and atmospheric phase screens during the acquisition time. 
These changes destroy the coherence of the scene over the aperture length and will spread these scatterer in azimuth, with a severe impact on image quality, especially regarding the stability of coherent targets.\\ \item a very large 
angular section can be imaged in one single pass, potentially up to $360^{\circ}$. This is not possible with a conventional ground based synthetic aperture radar, where only a small angular sector
can be covered at once; if it is desired to image another sector, the rail has to be rotated in order to look in the desired direction.\\
\item A physically large antenna provides an higher gain and hence a better SNR; this is not the case for  aperture synthesis  where the antenna has to be physically small to provide a sufficiently wide beam.\\
\end{enumerate*}\\
One major limitation common to all ground based systems is the difficulty of achieving range independent azimuth resolutions; this would require impossibly large antennas or rails.\\
Only very few polarimetric real aperture radars (type II systems) are in existence. A C-Band version of the GPRI has been produced\cite{Cherukumilli2012}, very little literature is avilable on this device.\\ This paper will focus on KAPRI, the Ku-Band Polarimetric Advanced Radar Interferometer (KAPRI). KAPRI is an extension of GPRI \cite{werner_gpri_2012,Strozzi2011, Werner2008}, a portable real-aperture, $K_u$ band radar interferometer operating at 17.2 GHz with a wavelength of 1.74 cm.\\ The original system is designed 
for the monitoring of unstable slopes using zero baseline differential interferometry \cite{Massonnet1993,JGRB:JGRB7093}. It is equipped with two receivers and two antennas arranged along a spatial baseline, permitting to derive local digital elevation models with single pass interferometric methods.\\
To discriminate the scatters in the azimuth a 2 meter long vertically polarized slotted waveguide antenna is employed, giving an azimuth beamwidth of $0.4^\circ$ and an elevation beamwdith of approx. $30^\circ$.\\
In terms of hardware, the main distinguishing feature of KAPRI is the addition of horizontally polarized antennas and switches that permit to connect the transmitter and the receiver to either type of antenna. These changes, together with modifications in the control software, permit to acquire a full polarimetric-interferometric dataset in the space of four regular GPRI II pulses by cycling through all the combinations of transmitted and received polarization.
\subsection{Goal of This Paper}
In this paper, the complete processing of KAPRI/GPRI data is addressed. In particular, two independent aspect of the processing chain are discussed in \autoref{sec:methods} :
\begin{enumerate}
	\item Processing of samples as acquired by the digitizer into single look complex (SLC) data. (\autoref{sec:proc_SLC})
	\item Polarimetric calibration of the SLC data. (\autoref{sec:proc_polcal})
\end{enumerate}
The discussion of the processing methods is split into parts that correspond to subsequent stages in the evolution of GPRI-II into KAPRI. Part \ref{sec:proc_SLC} presents  methods employed to process raw data into SLC form and that are used by both systems. This includes a derivation of the FMCW signal model and the correction of frequency-dependent beam squint. This discussion also forms the basis for part \ref{sec:proc_polcal}because properly focused SLCs  are a prerequisite for the polarimetric calibration. Part \ref{sec:proc_polcal} addresses issues specific to KAPRI, especially the problems arising when the antennas for the different polarizations are not collocated and the correction of phase and amplitude imbalances.\\
In \autoref{sec:results}, the methods are applied on data acquired during a calibration campaign, where a number of trihedral corner reflectors (TCR) was used a reference target. The results are discussed in two separate parts, reflecting the structure of \autoref{sec:methods}. In \autoref{sec:res_SLC} the effect of antenna beam-squint correction is illustrated by analyzing the oversampled range compressed response of a TCR alongside the corresponding envelope in the slow-fast time plane. Because an unexplained azimuthal phase variation is observed in the trihedral reflector responses, a model explaining it in terms of a phase center displacement is introduced in \autoref{sec:azimuth_processing} and an additional correction step is presented. In section \ref{sec:misalingment}, the correction of an azimuthal shift between the polarimetric channels is discussed and its effect on crosspolar measurements are shown by analyzing the response of a dihedral reflector.\\
The calibration methods of \ref{sec:proc_polcal} are then applied  and the quality of the calibration is assessed by plotting polarimetric signatures and computing the residual calibration parameters on the calibrated data.\\ This way of structuring the results section is nonstandard, but the authors feel that it better reflects the iterations that led to the final processing strategy.\\
In \autoref{sec:conclusions} the methods are summarized again and some conclusions are drawn.