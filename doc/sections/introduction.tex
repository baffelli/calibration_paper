\section{Introduction}
\PARstart{D}{ifferential} radar interferometry~\cite{Gabriel1989, Massonnet1993,Rosen2000,Bamler1999} is widely used  to monitor and study changes in the natural and built environment. The ability to  measure the line-of-sight component  of movements over large areas makes it suitable for many  applications. Some examples are the estimation of subsidence rate associated with temporal changes in the water table of aquifers, oil and gas extraction, deep mining and tunnel excavation~\cite{Galloway1998, Strozzi2001,Galloway2007}, the monitoring of inflation/deflation connected to volcanic activity~\cite{Massonnet1995}, the mapping of ice sheet and glacier motion~\cite{Goldstein1993,Mohr1998} , the observation of  instable slopes in rugged natural terrain and open pit mines~\cite{Carnec1996,Catani2005} and the measurement of seismic displacements~\cite{Massonnet1993a,Zebker1994}.\\
Fully polarimetric radar data provides additional information on the scattering mechanism within each resolution cell, which is employed for classification of the surface cover~\cite{Cloude1997, Lee1999}, to extract geophysical parameters such as moisture content~\cite{Hajnsek2003}, to estimate the orientation of the vegetation canopy ~\cite{Ulaby1987} or the height of fresh snow~\cite{Leinss2014}.\\
The availability of polarimetric information in addition to the interferometric time series allows to combine the scattering matrix and the interferometric coherence to better characterize the natural processes observed with the radar: a salient example being coherence optimization, where the scattering mechanism providing the best coherence and thus the least noisy phase measurement is selected\cite{Pipia2009a, Iglesias2014b}.\\
Spaceborne SAR systems, such as ERS-1 and 2, Envisat, TerraSAR-X, Sentinel 1-A/1-B, Radarsat-1 and 2 and ALOS, ALOS 2, Cosmo SkyMed, have been an important data source to build up interferometric time series for displacement measurements --- sometimes spanning over many years ---. These platforms are convenient in that they offer a large coverage in a single pass. However, the revisit time of these system is limited to few hours at best. 
In many cases to better understand the dynamics of natural processes and for real-time surveillance and alarming, a denser temporal sampling over longer time spans is desired than the ones afforded by current radar earth observation systems.
\subsection{State of the Art}
Today, several ground-based radar systems are available, operating in C~\cite{Leva2003, Rudolf1999a,Kang2009}, X~\cite{Aguasca2004,Pipia2007a} or Ku band~\cite{Leva2003, Rudolf1999a,Werner2008, Rodelsperger2012}. The majority of these systems are based on aperture synthesis, using a moving antenna on a rail. 
An alternative imaging approach is to scan a fan beam by rotating a large antenna\cite{Werner2008,werner_gpri_2012}. This imaging method has been called type II in~\cite{Caduff2015}. 
This configuration has certain advantages over ground-based synthetic aperture systems\cite{Monserrat2014} of comparable rail length: \begin{enumerate}
  \item Azimuth samples are acquired independently and do not require focusing, eliminating defocusing and loss of coherence caused by moving targets and atmospheric phase screens during aperture time. 
These changes may adversely affect the coherence of the scene over the aperture length and will worsen the azimuth resolution. This is especially problematic for the analysis of coherent targets.\\
 \item A large 
angular section of up to 270$^{\circ}$ can be imaged in a single pass. This is more difficult to obtain using a rail-based synthetic aperture radar.\\
\end{enumerate}
An important advantage of rail based SAR systems is the better azimuth resolution: a SAR using a rail of length $L$ will have an angular resolution of: 
\begin{equation}
 	\theta_{3dB}^{SAR} = \frac{\lambda}{2L},
\end{equation}
while an antenna with physical aperture size $L$ has an azimuth 3dB resolution of:
\begin{equation}
 	\theta_{3dB}^{RAR} = \frac{\lambda}{L}.
\end{equation}
The majority of polarimetric ground-based radar system is based on the aperture synthesis principle. An indoor system is presented in~\cite{Bennett1996} followed by a portable outdoors version~\cite{Bennett2000}. A broadband polarimetric SAR system with two dimensional aperture synthesis is introduced in~\cite{Zhou2004}, with measurement results presented in~\cite{Hamasaki2005}.
Another example  of a synthetic aperture system is is UPCs RiskSAR~\cite{Iglesias2014, Aguasca2004,Pipia2007a,Pipia2009, Pipia2013,Iglesias2014}. Example of ground-based polarimetric SAR data at X and C band are shown in~\cite{Kang2009, Kang2010}.\\ A dual polarization, multiband GB-SAR system is used in~\cite{Yitayew2014} to produce tomograms of snow covered sea ice. A similar concept is used in\cite{Frey2015,Frey2016} to produce full polarimetric tomographic profiles of a snow pack by synthesizing an aperture in the elevation direction.\\
Excluding non-imaging devices such as ground-based scatterometers, only few real aperture polarimetric ground-based radars exist, one example being the C-Band version of the GPRI\cite{Cherukumilli2012}.
\subsection{KAPRI: Real Aperture Polarimetric FMCW Radar}
This paper introduces KAPRI, the Ku-band  Advanced Polarimetric Radar Interferometer (KAPRI)\cite{Baffelli2016a}. It is an extension of GPRI (Gamma Portable Radar Interferometer)~\cite{werner_gpri_2012,Strozzi2011, Werner2008}, a real-aperture radar interferometer operating in Ku band at 17.2 GHz. It is designed 
to monitor unstable slopes using zero baseline differential interferometry~\cite{Massonnet1993}; two antennas arranged along a spatial baseline and a dual channel receiver permit to acquire local digital elevation models.\\
GPRI employs  2 meter long, vertically polarized slotted waveguide antennas, giving the system a 3 dB azimuth beamwidth of $0.385^\circ$ and a 3 dB elevation beamwidth of $35^\circ$.\\
The feature distinguishing KAPRI and GPRI is the addition of horizontally polarized antennas and switches that permit to connect transmitter and receiver to either type of antenna. Together with modifications in the control software, they enable it to acquire a full polarimetric-interferometric dataset by cycling through all the combinations of transmitted and received polarization during the acquisition.\\  In \autoref{tab:kapri_parameters} the main hardware characteristics of KAPRI are summarized.
\begin{table*}
	\centering
	\pgfplotstabletypeset[
		col sep=&,	% specify the column separation character
		row sep=\\,	% specify the row separation character
		columns/Parameter/.style={string type, column type=l}, % specify the type of data
		columns/Value/.style={string type, column type=l}, % specify the type of data for all columns
		every head row/.style={before row=\toprule,after row=\midrule},
		every last row/.style={after row=\bottomrule},
	]{
		Parameter & Value\\
		Modulation & FM-CW (250 $\mathrm{\mu} s$ to 16 ms chirp duration)\\
		Center frequency & 17.2 GHz\\
		Bandwidth & 200 MHz\\
		Range resolution& 0.95 m 3dB resolution @ -26 dB peak sidelobe ratio (PSLR)\\
		Azimuth 3dB beamwidth& $\mathrm{0.385^\circ}$ \\
		Elevation 3dB beamwidth& $\mathrm{35^\circ}$ \\
		Polarization & fully polarimetric, selectable TX and RX polarization\\
	}
	\caption{Summary of main KAPRIs parameters.}
	\label{tab:kapri_parameters}
\end{table*}
\subsection{Contributions of This Paper}
The following contributions are made in this paper:
\begin{enumerate}
	\item Preprocessing methods adapted to KAPRIs hardware are presented, that can be used to generate correct SLC images from the acquired raw data.
	\item A polarimetric calibration model adapted to the system design of KAPRI is presented. It includes the correction of effects caused by different designs of vertical and horizontal polarized antennas and the presence of spatial baselines between their phase centers.
	\item The proposed processing and calibration approaches are validated by analyzing the response of trihedral corner reflectors in a specifically acquired dataset.
\end{enumerate}
\subsection{Outline}
Part \ref{sec:methods:signal_model} presents  the methods employed to process the raw data into range compressed SLC. This part includes a derivation of the FMCW signal model and of the acquisition geometry, that will be used throughout the rest of this paper. \autoref{sec:methods:squint_correction} deals with the correction of frequency-dependent beam squint due to the slotted waveguide antenna design. These two sections describe the parts of the processing that are common to both KAPRI and GPRI. The quality of the processing is evaluated in \autoref{sec:results:squint_correction} by plotting the oversampled phase and amplitude response of trihedral corner reflectors, where significant range resolution improvements are observed by applying the described squint compensation procedure.\\
After this step, the range compressed, frequency-squint corrected data still shows a residual azimuth phase variation, especially in the VV channel where a linear variation of almost $30^\circ$ is observed for samples inside the antenna beamwidth. This effect is modeled in \autoref{sec:methods:azimuth_processing} as a change in distance between the antennas phase center and the scatterers caused by the rotation of the antenna. A method to correct it is proposed and tested in \autoref{sec:results:azimuth_processing} on an array of five trihedral corner reflectors.\\ An azimuthal shift between the HH and the VV channel is observed on the  response of point targets along with the phase variation; it is ascribed to misaligned antenna patterns. If left uncorrected, it would cause a reduced power and increased SNR for crosspolar measurements. To remove it, modified antenna mounts that permit to mechanically shifting the antennas mainlobe were manufactured. They are tested in \autoref{sec:results:misalignment} by analyzing the response of a crosspolarizing dihedral reflector acquired with different antenna mounting settings.\\
A final step is required before polarimetric calibration: since KAPRIs employs separate antennas for each transmit and receive polarization, spatial baselines are obtained between certain combinations of channels. These baselines add a topographic contribution to the polarimetric phase differences. In \autoref{sec:methods:topo_removal} a method is derived to estimate this contribution using an interferogram obtained from two identically polarized channel on a baseline and rescaled to the undesired polarimetric baseline. Its validity is verified by analyzing the resulting HH-VV phase difference in \autoref{sec:results:topo_removal}.\\
In \autoref{sec:methods:proc_polcal} the polarimetric calibration is discussed. A linear distortion model without crosstalk is assumed; the copolar phase and amplitude imbalances are estimated using a trihedral corner reflector, while the imbalance between the crosspolar channel is determined using the HV-VH phase difference over distributed scatterers assuming reciprocity. The plausibility of zero cross-talk is assessed by computing the polarization purity of all the trihedral reflectors in the calibration dataset, showing a purity of more than $35 dB$ at worst.\\
Finally, in \autoref{sec:results:proc_polcal} the quality of data calibration is assessed by computing polarization signatures for the trihedral corner reflectors and by estimating calibration model residuals on the corner reflector array. 