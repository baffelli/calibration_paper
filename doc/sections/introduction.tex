\section{Introduction}
\PARstart{D}{ifferential} radar interferometry\cite{Gabriel1989, Massonnet1993} techniques are gaining importance for the monitoring of changes in the natural and built environment. The ability to accurately measure movements along the line of sight over large areas renders them interesting for a host of applications. Some examples are the estimation of subsidence rate associated with groundwater or oil extraction and the study of aquifer dynamics\cite{Fielding1998,Strozzi2001,Galloway2007a}, the monitoring of inflation/deflation connected to volcanic activity\cite{Massonnet1995}, the mapping of ice sheet and glacier motion\cite{Goldstein1993,Mohr1998} , the observation of landslides and instable slopes\cite{Carnec1996,Catani2005} and the measurement of seismic displacements\cite{Massonnet1993b,Zebker1994}.\\
Full polarimetric radar data provides additional information on the type of scatterers contained in each resolution cells, permitting a classification of the surface cover\cite{Cloude1997, Lee1999} and to extract geophysical parameters such as moisture content\cite{Hajnsek2003} or informations on the vegetation cover\cite{Ulaby1987} or the height of fresh snow\cite{Leinss2014}.\\
The combination of polarimetric and differential interferometric measurement could bring additional benefits for the observation of natural processes; for example by selecting the polarization that provides the best coherence and thus the least noisy phase measurement\cite{Pipia2009a, Iglesias2014b}.\\
Currently, the majority of radar data employed for differential interferometry is acquired using sensors carried by satellite or aircrafts. These platform are convenient in that they offer a large coverage in with a single data take. Nevertheless, because of costs, technical, physical and logistical restrictions, the revisit time of these system is limited to a few hours at best. 
In many cases, to better understand the dynamics of the natural processes and for real time surveillance and alarming, a denser temporal sampling over longer time spans is desired than afforded by current airborne and spaceborne systems.\\
For observation of smaller areas at shorter revisit times, ground based radar systems basing on different imaging principles were developed. While most  solutions provide range resolution by using some form of modulated wave, the main differences are observed in the method chosen to obtain resolution in the cross track dimension. 
Several systems\cite{Rudolf1999a, Rodelsperger2011, Aguasca2004,Rodelsperger2012} synthesize an aperture by moving the antenna assembly on a short motorized rail. Some systems of this type were designed with polarimetric data acquisition in mind\cite{Iglesias2014,LEE, Kang2009}.\\
\subsection{KAPRI: Real Aperture Polarimetric FMCW Radar}
The alternative to synthesizing an aperture by moving the sensor is to rotationally scan a fan beam antenna, separating the scatterers by using the  narrow beam emitted by a large antenna\cite{Werner2008}. This imaging method has been called type II in \cite{Caduff2015}. 
This configuration has two advantages compared to similar ground-based synthetic aperture systems based on a rail\cite{Monserrat2014} of comparable aperture length length: \begin{enumerate*}
  \item unlike a SAR system, each azimuth line is 
acquired independently from the others, eliminating the problem of decorrelation due to moving targets and atmospheric phase screens. 
These changes destroy the coherence of the scene over the aperture length and will cause spreading of these scatterer over
the azimuth direction, with a severe impact for the image quality, especially regarding the stability of coherent targets. \item a very large 
angular section can be imaged in one single pass, potentially up to $360^{\circ}$. This is not possible with a conventional ground based synthetic aperture radar, where only a small angular sector
can be covered at once; if it is desired to image another sector, the rail has to be rotated in order to look in the desired direction. \
\end{enumerate*}\\
One major limitation of real aperture radar imaging is the inability of reaching arbitrarily small azimuth resolutions: this would require a very large antenna, which is very difficult to build.
Therefore, the azimuth resolution is limited by electrical and mechanical design constraints. On the other hand, a physically large antenna offers the advantage of an higher gain and hence a better SNR, while in the case
of aperture synthesis, the antenna has to be physically small to have a wide beam, which reduces the antenna gain and potentially leads to a worse SNR.\\
Only very few polarimetric real aperture radars (type II systems) are in existence. A C-Band version of the GPRI has been produced\cite{Cherukumilli2012}, sadly only very little literature is avilable on this device. The rest of this paper will be focused on KAPRI, the Ku-Band Polarimetric Advanced Radar Interferometer (KAPRI). KAPRI is an extension of the GPRI \cite{werner_gpri_2012,Strozzi2011, Werner2008}, a portable real-aperture, $K_u$ band radar interferometer operating at 17.2 GHz with a wavelength of 1.74 cm . The original system is designed 
for the monitoring of unstable slopes by zero baseline differential interferometry \cite{Massonnet1993,JGRB:JGRB7093}; additionally a dual receiver and two antennas arranged along a spatial baseline, allow the acquisition of local digital elevation models.\\
To discriminate the scatters in the azimuth a 2 meter long vertically polarized slotted waveguide antenna is employed, giving an azimuth beamwidth of $0.4^\circ$ and an elevation beamwdith of approx. $30^\circ$.\\
In terms of hardware, the main feature that distinguishes KAPRI and GPRI II is the addition of horizontally polarized antennas and switches that permit to connect the transmitter and the receiver to either type of antenna. These changes, together with modifications in the control software, permit to acquire a full polarimetric-interferometric dataset in the space of four regular GPRI II pulses by temporal multiplexing, i.e by cycling through all the combinations of transmitted and received polarization.
\subsection{Goal of This Paper}
In this paper, the complete processing of KAPRI/GPRI raw data is addressed. In particular, two independent aspect of the processing chain are discussed in \autoref{sec:methods} :
\begin{enumerate}
	\item Processing of dechirped samples into single look complex (SLC) data squint. (\autoref{sec:proc_SLC})
	\item Polarimetric calibration of the SLC data. (\autoref{sec:proc_polcal})
\end{enumerate}
The processing is split into parts that logically correspond to subsequent stages in the evolution of GPRI-II into KAPRI. Part \ref{sec:proc_SLC} presents  methods employed to process raw data into SLC form. This includes a derivation of the FMCW signal model and the correction of frequency-dependent beam squint. These methods are relevant for both KAPRI and his predecessor GPRI-II. This discussion also forms the basis for part \ref{sec:proc_polcal} as properly focused SLCs acquired with different combinations of transmitting and receiving polarizations are a prerequisite for the polarimetric calibration. Part \ref{sec:proc_polcal} addresses issues specific to KAPRI, especially the problems arising when the antennas for the different polarizations are not collocated and the correction of phase and amplitude imbalances.\\
In \autoref{sec:results}, the method are applied on data acquired during a calibration campaign, where a number of trihedral corner reflectors (TCR) was used a reference target. The results are discussed in two separate parts, reflecting the structure of \autoref{sec:methods}. In \autoref{sec:res_SLC} the effect of antenna beam-squint correction is illustrated by analyzing the oversampled range compressed response of a TCR alongside the corresponding envelope in the slow-fast time plane. Because an unexplained azimuthal phase variation is observed in the trihedral reflector responses, a new model explaining it in terms of a phase center displacement is introduced and an additional correction step is presented. These modification will be applied to the data before the polarimetric calibration in \ref{sec:proc_polcal}. The calibration methods of \ref{sec:proc_polcal} are then applied  and the quality of the calibration is assessed by plotting polarimetric signatures and computing the residual calibration parameters on the calibrated data.\\ This way of structuring the results section is nonstandard, but the authors feel that it better reflects the iterations that led to the final processing strategy.\\
In \autoref{sec:conclusions} the methods are summarized again and some conclusions are drawn.