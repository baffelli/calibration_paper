\section{Methods}\label{sec:methods}
\subsection{KAPRI: FMCW radar data processing}\label{sec:proc_SLC}
A fundamental requirement to generate calibrated polarimetric data is the availability of properly focused SLC images for all the desired polarizations. To obtain this data, it is necessary to understand the data acquisition process and correct several effects caused by the specific hardware design.  
For this purpose a signal model for type II\cite{Caduff2015} radar data using FMCW signaling\cite{Stove1992} is introduced.\\
\begin{figure}[h]
	\centering
	\includegraphics[scale=1]{real_aperture_signal_model_geometry}
	\caption{Geometry used to derive the FMCW signal model. $R$ is the slant range from the radar to the point scatterer, $\theta_{3dB}$ is the antenna half power beamwidth, which is represented by the gray triangle. The size of the antenna aperture is $L_{ant}$, the corresponding azimuth resolution (in distance units) is $\delta_{az}$. The inset figure is used to derive the azimuth phase variation. $L_{ph}$ is the phase center displacement, $L_{arm}$ is the antenna rotation lever arm, $R_{0}$ the range of closest approach and $\alpha$ the additional angle to obtain closest approach when the phase center is not in the midpoint of the array.}
	\label{fig:real_aperture_signal_model_geometry}
\end{figure}
Consider a coordinate system with origin at the location of a radar as depicted in \autoref{fig:real_aperture_signal_model_geometry}. In this system, the antenna is mounted on a lever arm of length $L_{arm}$; its mainlobe is parallel to the $x$ axis when the pointing angle $\theta$ is 0. The radar images a scene with a complex reflectivity distribution $\rho\left(x,y\right)$ by measuring range profiles $\hat{\rho}\left(R, \theta\right)$ for a number of antenna rotation angles (azimuths) $\theta = \operatorname{arctan}\left(\frac{y}{x}\right)$ by rotating the antenna with the angular speed $\omega$. Each profile is measured by transmitting a linearly modulated signal of duration $t_{chirp}$ with bandwidth $B$ and center frequency $f_c$:\\
\begin{equation}
	s_t\left(t\right) = e^{j 2 \pi \left( t f_{c} +  \frac{B}{t_{chirp}} t^2 \right)}.
\end{equation}
In the case of KAPRI, $f_c= 17.2~GHz$ and $B=200~MHz$.\\
The total time that the signal takes to travel to a scatterer at range $R$ and back is $t_{c} = \frac{2 R}{c}$. The backscattered signal $s_r$ is a copy of the one being transmitted; delayed by $t_{c}$ and scaled by the complex reflectivity of the scatterer $\rho$.
\begin{equation}
		s_r\left(t\right) = \rho e^{j 2 \pi \left( \left(t - \frac{2 R}{c}\right) f_{c} +  \frac{B}{t_{chirp}} \left(t - \frac{2 R}{c}\right)^2 \right)}.
\end{equation}
The received backscattered signal is $s_r$  mixed with the transmitted chirp $s_t$  to remove the linear modulation; the resulting beat signal is then sampled and stored in the device. It frequency $f_{b} = \frac{4 R B}{c t_{chirp}}$ is proportional to the slant range $R$:
\begin{equation}\label{eq:deramp}
	\begin{aligned}
	s_{d}\left(t\right) &=s_t\left(t\right)s_r\left(t\right)^* =\\ 
	&\sigma e^{j 4 \pi \frac{ R}{c}f_c}  e^{j 4 \pi \frac{2 R B }{c t_{chirp}} t}  e^{j 4 \pi \frac{2 R^2 B}{c^2}}.
	\end{aligned}
\end{equation} 
In this expression, two phase terms can be identified: $ e^{j 4 \pi \frac{R}{c}f_c}$ is the two way propagation phase, the quantity of interest for  interferometric measurements. The second phase component is the residual video phase (RVP). This component needs to be compensated for SAR processing, where its variation during the aperture time may cause defocussing.\\
From  linearity it follows from \autoref{eq:deramp} that the range profile $\hat{\sigma}\left(R, \theta\right)$ of a collection of targets with complex reflectivities $\rho_i$ located at ranges $R_{i}$ is recovered by taking the Fourier transform of $s_{d}\left(t\right)$.\\
In analogy to the pulsed radar case, the range resolution $\delta_{r} = \frac{c}{2 B}$ for a FMCW system is inversely proportional to the bandwidth $B$. With $B=200 MHz$ KAPRI can achieve a range resolution of 0.75 m\cite{Strozzi2011}; the effective range resolution is lower because the dechirped data is windowed before the Fourier transform to mitigate range sidelobes. It is interesting to note that for FMCW system the sampling rate is not governed by the transmitted bandwidth; instead it is dictated by the extent of ranges to be imaged\cite{Meta2006}.\\
To obtain two dimensional images range profiles are acquired while the antenna is rotated with angular velocity $\omega$.
Thus, the dechip signal for a point target at $R, \theta_{t}$ in the slow-time versus fast time plane is:
\begin{equation}\label{eq:signal_model}
	\begin{aligned}
	& s_{d}\left(t,\tau\right) = \sigma e^{j 4 \pi \frac{ R}{c}f_c}   \\
	& e^{j 4 \pi \frac{2 R B }{c t_{chirp}} t}  e^{j 4 \pi \frac{2 R^2 B}{c^2}} P\left(\tau \omega - \theta\right),
	\end{aligned}
\end{equation} 
where $t$ is the fast time, $\tau = n t_{chirp}$ is the slow time variable and $P\left(\tau \omega\right)$ describes the two way antenna pattern. Its beamwidth is approximated by:
\begin{equation}\label{eq:azimuth_resolution}
	\theta_{3dB} = \frac{\lambda}{L_{ant}}
\end{equation}
where $L_{ant}$ is the size of the antenna aperture and $\lambda$ is the wavelength employed.
Due to diffraction, the radiation beam emitted by the antenna broadens linearly with increasing distance consequently, the spatial resolution in cross-range $\delta_{az}$ increases with distance:
\begin{equation}\label{eq:azimuth_ground_resolution}
	\delta_{az} = \frac{R \lambda}{L_{ant}}.
\end{equation}
%The typical applications of KAPRI require a ground resolution on the order of a few meters at ranges up to 8 km, requiring a beamwidth. To obtain a beamwidth of $0.4^\circ$ at $17.1 GHz$, $L_{ant} = 2m$. 
KAPRI employs a traveling wave slotted waveguide antenna\cite{Hines1953a,Granet2007}; it constructed by cutting slots resonating at the design frequency in a section of rectangular waveguide. When they are appropriately spaced, the fields emitted at each cut combine in phase, producing a narrow beam. Two types of slotted waveguide antenna exist\cite{Enjiu2013}: the resonant and the traveling wave design. The second type has been chosen because it supports a larger bandwidth; this permits to achieve a finer range resolution. However this antenna design displays frequency-dependent beam squint: if it is operated at a frequency other than the design value, the phase difference at the slots changes with the effect of squinting the beam away from the intended mainlobe direction. This effect has been used for several fast imaging radar, where a mechanical antenna rotation would not be possible\cite{Yang2014,Yang2012,Mayer2003,Alvarez2013}. In the case of KAPRI the squint is undesired: since the amount of squint is large compared to the beamwidth, the beam only illuminates a scatterer for a fraction of $t_{chirp}$, decreasing the effective transmitted bandwidth and hence the range resolution.  If the antenna is rotated slowly enough,  the spacing of each range profile is less than the beamwidth and the effect of beam squint is visible in the slow-fast time domain as skewed point target\autoref{fig:squint_correction}; this particular pattern is due to their partial illumination during the continuous motion of the antenna: For each rotation angle the target is only inside the beamwidth during the time when the antenna rotation matches the beam squint. 
The azimuth-frequency skewing observed in oversampled data is key to mitigate the effect of the beam squint. For each chirp frequency $f$, the data at the corresponding fast time $s_{d}\left(t,\tau\right)$ is shifted back in azimuth by the amount of squint predicted by $\tau_{sq}=\frac{\theta_{sq}}{\omega}$:
\begin{equation}\label{eq:squint_exact}
	\theta_{sq} = \sin^{-1}\left(\frac{\lambda}{\lambda_{g_{ij}}} + \frac{k \lambda}{s}\right).
\end{equation}
where ${\lambda_g}_{ij}$ is the wavelength for the $ij$-TE mode of the waveguide, $\lambda$ is the freespace wavelength, $s$ is the element spacing and $k$ is the mode number. In this case, the waveguide mode used is TE01 and $k=0$ is assumed because all the slots are to be transmitting in phase\cite{kraus88} to direct the main beam at the antenna broadside.\\
In processing KAPRI data a linear approximation for  the variation in beam squint relative to the pointing at the design frequency is used instead of\autoref{eq:squint_exact} :
\begin{equation}\label{eq:squint_linearised}
	\theta_{sq} - \theta_{sq}^{f_{c}}  =  \alpha f
\end{equation}
this choice is necessary because the manufacturer of the antennas provided antenna pattern measurements at different frequencies suggesting that the vertically and the horizontally polarized units have different squint characteristics despite using a waveguide of the same size.\\ No other design information being available, a data-driven method had to be used for the correction of the beam squint.  The data is first range compressed with a fast-time FFT. The location of a point-like scatterer is the selected; the samples around the point are then windowed in range to isolate its response and converted back into the fast time domain, where the complex envelope response in the slow-fast time plane is estimated with an Hilbert transform. The model of \autoref{eq:squint_linearised} is then fitted on the envelope thus extracted to obtain $\alpha$.
To correct for the frequency-dependent beam squint, it is fundamental that the acquisition of the data is performed by rotating the antenna slowly enough. Having an azimuth sample spacing smaller than the antenna beamwidth permits to overcome the effect of beam squinting by reconstructing the illumination of the target with the full bandwidth by interpolating the data in the azimuth-fast time domain. A side benefit of oversampling is that several range profiles that in first approximation can be seen as representing independent realizations of the same profile can be averaged to improve the measurement SNR, producing range-azimuth images with an angular resolution limited by the antenna beamwidth.\\
\begin{figure}[ht]
	\centering
	\includegraphics[scale=0.2]{squint_correction}
	\caption{Illustration of the frequency-dependent antenna squint. When the antenna rotates during the electronic scan, the energy of the target is spread through several azimuth bins.}
	\label{fig:squint_correction}
\end{figure}
After correcting the beam squint,  the first and last $z$ samples of the squint corrected raw data $s_{d}$ are windowed with an Hann window to mitigate the transient signal caused by the sawtooth frequency sweep. A second Kaiser Window is then applied to the dechirped data in the fast-time; this filter serves to reduce range processing sidelobes. Finally, a fast time Fourier transform performs the range compression to obtain the SLC image $\hat{\rho}\left(R, \theta\right)$.\\ Each range line of the SLC thus compressed is then multiplied by $\sqrt{R^3}$ to compensate for the power spreading loss. In this manner, the intensity of the SLC data is directly proportional to the radar brightness $\beta_{0}$.
\subsection{Antenna Pattern Misalingment}\label{sec:misalingment}
In addition to the difference in frequency-dependent squint rate, the horizontally and the vertically polarized antennas are observed to have an azimuth misalignment of approx. $0.2^\circ$,
implying that $\theta_{sq}^{fc}$ is not the same for both units. This reflected as an azimuth offset between the HH and VV channels; it is problematic for cross-polar measurements as the transmitting and receiving antenna patterns are not aligned, causing a power loss of approx 2.25 dB and consequently a reduced SNR for the cross-polar channels. While the HH-VV offset can be easily corrected by coregistering the data, no method is able to correct for the reduced SNR in the crosspolar measurements. To improve crosspolar performance, an adjustable antenna mount was manufactured, that allows to shift the $V$ antenna in azimuth to bring the patterns into alignment.\\
\subsection{Polarimetric Calibration}\label{sec:proc_polcal}
In the case of GPRI the range compression concludes the processing of raw data producing SLC data that is used for interferometric processing.  For KAPRI, additional steps are necessary to obtain polarimetric measurements that are well calibrated.\\ The first of these steps requires to briefly review the antenna configuration used by KAPRI(\autoref{fig:antenna_arrangement}):\\ Six antennas are mounted on a supporting structure connected to the rotary scanner. Of these, 2 are transmitting antennas, one for each polarization. The remaining 4 are connected in groups of two to the dual receivers, each pair consisting an horizontally and a vertically polarized array. This configuration permits to acquire full polarimetric dataset by selecting the desired transmit and receive polarization while minimizing polarimetric crosstalk thanks to the spatial separation. More crucially, the separation of transmitting and receiving antennas increases the TX-RX isolation, a fundamental requirement for FMCW performance\cite{Beasley1990,Stove1992, Strozzi2011}.  However, this configuration is not without consequences on the phase of the resulting polarimetric data.
\begin{figure}[ht]
	\centering
	\includegraphics[scale=0.15]{kapri_antenna_arrangement}
	\caption{KAPRI with the usual antenna arrangement overlaid. }
	\label{fig:antenna_arrangement}
\end{figure}
To see why, consider a channel $i$ measured by transmitting at antenna located at $\mathbf{x_t^i}$ and receiving at $\mathbf{x_r^i}$. These antennas can be replaced by an equivalent antenna located at $\mathbf{x_{eq}^i}$, the midpoint between transmitter and receiver \cite{Pipia2009}. Because $\mathbf{x_t^i}$ and $\mathbf{x_r^i}$ are different for each polarimetric acquisition, the equivalent antenna location $\mathbf{x_{eq}^i}$  will change position depending on the polarization.  Therefore, for some combinations of channels $i$ and $j$ the equivalent phase centers will have a baseline $\mathbf{b_{ij}^{eq}}$ and the polarimetric phase difference $\phi_{ij} = \phi_{ij}^{pol} + \phi_{ij}^{prop}$  will contain an interferometric contribution $\phi_{ij}^{prop}$. This term appears as topographic fringes when visualizing the polarimetric phase difference and will complicate calibration by adding an additional phase contribution unrelated to the polarimetric properties of the scatterers.\\
To remove this contribution, it is necessary to acquire two channels $k$ and $l$ with a non-zero baseline $\mathbf{b_{ij}^{eq}}$ and the same polarization so that $\phi_{kl}^{pol} \approx 0$.\\ Generally, for any two channels $m$ and $n$,  the propagation phase difference can be approximated as a function of the local incidence angle and of the perpendicular baseline separating the phase centers:
\begin{equation}\label{eq:prop_approximation}
		\phi_{mn}^{prop} = \frac{4\pi}{\lambda} b_{mn}^{eq} \sin(\theta - \alpha_{bl}),
\end{equation}
where $\alpha_{bl}$ is the baseline angle w.r.t to the vertical and the look angle $\theta_l$ is the angle between the line of sight vector $\mathbf{p}$ and the vertical axis and $b_{ij}^{eq}$ is the perpendicular baseline between the equivalent phase centers.
Considering \autoref{eq:prop_approximation}, $\phi_{ij}^{prop}$ can be estimated from $\phi_{kl}$ if the look angle does not significantly change from $kl$ to $ij$, i.e if $\theta_{ij} - \alpha_{ij} \approx \theta_{kl} - \alpha_{kl}$. 
\begin{equation}
	\hat{\phi}_{ij}^{prop} = \frac{b_{eq}^{ij}}{b_{eq}^{kl}} \phi_{kl}.
\end{equation}
This formula can only be used if $\frac{b_{eq}^{ij}}{b_{eq}^{kl}}$ is integer\cite{Massonnet1996}, if this condition is not met, phase unwrapping of $\phi_{kl}^{prop}$ is necessary before rescaling.\\
In order to correct for all the possible combinations that have a non-zero baseline, the measured scattering matrix $\mathbf{S}$ is converted into a polarimetric covariance matrix;  $\hat{\phi}_{ij}^{prop}$ it then subtracted from the phase of every non-diagonal element $ij$. The result is a flattened covariance matrix where the sole phase contribution is the polarimetric phase difference.\\
This matrix is the starting point for the polarimetric calibration proper;
the procedure is based on the linear distortion matrix model\cite{Saraband1990, Sarabandi1992a} that relates the observed scattering matrix $\mathbf{S_{meas}}$ with the correct matrix $\mathbf{S}$:
\begin{equation}\label{eq:distorsion_scattering}
	\mathbf{S_{meas}} = \mathbf{R} \mathbf{S} \mathbf{T}.
\end{equation}
or in covariance form
\begin{equation}\label{eq:covariance_distortion}
	\mathbf{C}_{meas} = \mathbf{D} \mathbf{C} \mathbf{D}^{H}.
\end{equation}
where $\mathbf{D}$ is the Kronecker product of $\mathbf{R}$ and $\mathbf{T}$, the matrices that describe the phase and amplitude imbalances and crosstalk in reception and transmission.
In the case of KAPRI, the crosstalk calibration is not performed as the radar is expected to have a good polarization isolation, largely due to the fact that only one polarization is acquired at a time by selecting the appropriate combination of transmitting and receiving antennas. The only source of crosstalk is the presence of cross-polarized lobes in the direction of the antenna mainlobe. The manufacturer has provided simulated radiation patterns for the horizontally polarized antennas, where the isolation between the co and the cross polarized pattern in the main-lobe direction is observed to be better than 60 dB. By computing the $HH-HV$ ratio of the oversampled response of a trihedral corner reflector, the polarization purity of the system was estimated to be at least 35 dB.\\
The distortion matrices are:
\begin{equation}
	\begin{aligned}
	&\mathbf{R} = A \begin{bmatrix}
		1 & 0\\
		0 & f/g e^{i\phi_{r}}
	\end{bmatrix},\\
	&\mathbf{T} = A \begin{bmatrix}
			1 & 0\\
			0 & f g e^{i\phi_{t}}
		\end{bmatrix}
	\end{aligned}
\end{equation}
where $f$ is the one-way copolar amplitude imbalance with respect to the $H$ polarization, and $g$ the amplitude imbalance of the crosspolarized channels. $\phi_t = \phi_{t,v} -\phi_{t,h}$ is the phase offset between the polarizations when transmitting and $\phi_{r} = \phi_{r,v} -\phi_{r,h}$ is the phase offset in reception and $A$ is the absolute amplitude calibration parameter (RCS)\cite{Ainsworth2006a, Fore2015}.\\
The four unknown complex parameters in $\mathbf{D}$ can be determined using a trihedral corner reflector and a reciprocal scatterer with a significant cross polarized contribution\cite{Sarabandi1989,Pipia2009}.
With the above model, an ideal trihedral reflector with the scattering matrix
\begin{equation}
 \mathbf{S} = \sqrt{\sigma_{tri}}
 \begin{bmatrix}1 & 0\\ 0 & 1\end{bmatrix}
\end{equation}
where $\sigma_{tri}$ is its RCS, has a measured covariance matrix $\mathbf{C^{\prime}}$:
\begin{equation}
	\begin{aligned}
	&\mathbf{C^{\prime}} =\\
	&= k \sigma_{tri}\\
	&\begin{bmatrix}
		1 & 0 & 0 & f^2 e^{-i \left(\phi_t + \phi_r\right)}\\
		0 & 0 & 0 & 0\\
		0 & 0 & 0 & 0\\
		f^2 e^{i \left(\phi_t + \phi_r\right)} & 0 & 0 & f^4
	\end{bmatrix}
	\end{aligned}.
\end{equation}
The copolar amplitude imbalance $f$ is estimated of the HHHH and VVVV elements of $\mathbf{C^{\prime}}$:
\begin{equation}
	f = \left(\frac{C^{\prime}_{VVVV}}{C^{\prime}_{HHHH}}\right)^{\frac{1}{4}}.
\end{equation}
Similarly, the copolar imbalance phase $\phi_r + \phi_t$ is determined from the phase of $C_{VVHH}^{prime}$:
\begin{equation}
	\phi_r + \phi_t = \operatorname{arg}\left(C_{meas}^{VVHH}\right).
\end{equation}
For a better localization of the TCR, of both parameters are estimated at the location of the maximum on the oversampled response of the reflector.
Because of the difficulty of placing and correctly orienting a dihedral reflector, the estimation of $g$ and $\phi_t - \phi_r$ is based on the assumption that most pixels in the calibration dataset represent reciprocal scatterers:
\begin{equation}
	g = \left<\frac{C_{rec}^{HVHV}}{C_{rec}^{VHVH}}\right>^\frac{1}{4},
\end{equation}
and:
\begin{equation}
	\phi_t - \phi_r =\operatorname{arg}\left( \left<C_{meas}^{HVVH}\right>\right).
\end{equation}
When $\mathbf{D}$ is estimated, \autoref{eq:covariance_distortion} is inverted to obtain a calibrated covariance matrix.\\
If radiometric calibration is desired, the value of $A$ can be determined after imbalance correction:
\begin{equation}
	k =	\left(\frac{\sigma_{tri}}{C^{\prime}_{HHHH}}\right).
\end{equation}
\subsection{Experimental Data}\label{sec:data}
In order to develop and test the methods described above, a calibration dataset was acquired in September 2016 at an urban-agricultural area near M\"{u}nsingen, Switzerland. The data was acquired from the top of an hill approximately 800 m high, looking down towards the fields and the town. 5 Trihedral Corner Reflectors were placed in the scene for the determination of calibration parameters and to assess imaging quality. Three of these reflectors have triangular faces with a length of 40 cm, corresponding to a RCS of $\sigma=\frac{4}{3}\pi \frac{a^4}{\lambda^2}=25.5 dB$, while the remaining two are cubic corner reflector with $\sigma= 35 dB$, at the nominal central frequency of 17.2 GHz.\\
	\begin{figure}
		\centering
		\includegraphics{figure_7}
		\caption{Pauli RGB composite of the imaged scene. The location of calibration corner reflector is marked by blue circles.}
		\label{fig:pauli_rgb}
	\end{figure}
The dataset was acquired with the horizontally polarized antenna group shifted towards the V group by 1.8 mm to compensate for the pattern misalignment as described in \autoref{sec:misalingment}.\\
\autoref{fig:pauli_rgb} shows the calibrated Pauli RGB composite of the scene, resampled in cartesian coordinates with the location of the reflectors marked using blue circles.
%\pgfplotstableread[col sep=comma]{../tab/table_1.csv}{\tabI}
%\pgfplotstableset{ref style/.style={
%									columns={rsl, RCS},
%									every head row/.style={before row=\toprule,after row=\midrule},
%									every last row/.style={after row=\bottomrule},
%									columns/rsl/.style={precision=3, column name={$R_0$}}
%									columns/RCS/.style={precision=3, column name={$\sigma_0$}}
%									%columns/type/.style={column name={Type} ,string type, column type={s}},
%								%display columns/3/.style={precision=3, column name={$\sigma_0$}},
%									}
%									}

\begin{table}[ht]
	\centering
	\pgfplotstabletypesetfile[
								every head row/.style={before row=\toprule,after row=\midrule},
								every last row/.style={after row=\bottomrule},
								col sep=comma,
								columns={rsl, RCS, type},
								columns/type/.style={column name={Type} ,string type},
								columns/RCS/.style={precision=3, column name=$\sigma_0$},
								columns/rsl/.style={precision=3, column name=$R_0$}
										] {../tab/table_1.csv}
	\caption{Summary of the employed TCRs. Distance from the radar and expected RCS.}
	\label{tab:reflectors}
\end{table}



%\begin{figure}[ht]
%	\centering
%	\begin{subfigure}[b]{\columnwidth}
%		\centering
%		\includegraphics[scale=1]{HIL_20140910_144113_l_03_normal_gc_phase.pdf}
%		\subcaption{Uncorrected}
%		\label{fig:HHVV_phase:unflattened}
%	\end{subfigure}
%	\begin{subfigure}[b]{\columnwidth}
%		\centering
%		\includegraphics[scale=1]{HIL_20140910_144113_l_03_flat_gc_phase.pdf}
%		\subcaption{Corrected}
%		\label{fig:HHVV_phase:flattened}
%	\end{subfigure}
%	\caption{HH VV phase difference. \autoref{fig:HHVV_phase:unflattened}: Phase difference after the correction of the azimuth ramp. The topographic phase ramp is very clearly visible. In  \autoref{fig:HHVV_phase:flattened} the phase after the topographic phase removal is shown. There is no noticeable phase trend.}
%	\label{fig:HHVV_phase}
%\end{figure}
%\FloatBarrier

