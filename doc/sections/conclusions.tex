\section{Conclusions}\label{sec:conclusions}
In this paper, several aspects of the polarimetric calibration of KAPRI, a new portable ground based FMCW radar were discussed. The instrument is
based on the GPRI\cite{werner_gpri_2012} a Ku band radar that was designed for slope instability monitoring using differential interferometry and DEM generation. With the addition of horizontally polarized antennas and electronic switching circuitry, the system was upgraded to a full polarimetric and interferometric radar.\\ Both  devices are based on slotted waveguide antennas, due to their design, the antenna show a large frequency-dependent squint of the main lobe. This behavior was observed in the data and corrected with a linear interpolation of the dechirped data in the chirp time-azimuth domain, yielding correctly focused range profiles resolved in azimuth by the antenna beamwidth.\\
After the frequency dependent squint correction, an azimuthal phase ramp in the response of point like scatterers was observed in the range compressed data, especially for the VV channel. A model was developed to explain this variation in terms of an antenna phase center horizontally displaced from the lever arm that causes a small variation in the slant range distance from the radar during the rotational scan. This change in range is too small compared to the range resolution to cause visible range cell migration. However, the variation in propagation path is suspected to induce the observed phase modulation. Using the model, the horizontal displacement was estimated and its effect was corrected using the model as a azimuth matched filter. The phase after the proposed filtering showed very little azimuthal variation and could be used to determine polarimetric calibration parameters from a trihedral corner reflector.\\
Before proceeding to calibrated, an additional step to remove the topographic phase contribution due to the spatial baseline between the polarized antenna was necessary. The topography flattening is obtained by computing an unwrapped  interferogram using the additional interferometric baseline, rescaling it to the equivalent baseline between the polarization and subtracting it from the covariance matrix elements that are affected by the topographic phase contribution.\\ Once the topographic phase contribution was removed, the polarimetric calibration was performed by assuming zero crosstalk and estimating co- and crosspolar imbalances using a trihedral corner reflector and distributed targets. The assumption of no crosstalk is supported by the use of temporal multiplexing for the acquisition of polarimetric data and by the high polarization purity of the antennas. The calibration quality was tested by applying the method on a scene where five trihedral corner reflector were placed at different ranges from the radar. One of them was used to determine the parameters that where then applied to the entire image. Copolar phase and amplitude imbalances were then recomputed on the response of the calibrated reflectors; the amplitude imbalance is close to unity for all reflector while the mean residual phase imbalance is under 5 degrees.\\
A polarimetric analysis of the calibrated data suggest a dominant surface scattering behavior from most of the natural objects in the scene, probably due to low penetration at Ku-Band and to the high level of roughness compared to the wavelength.