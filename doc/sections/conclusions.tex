\section{Conclusions}\label{sec:conclusions}
In this paper, two main aspects of the calibration of KAPRI, a new polarimetric portable ground-based FMCW radar were discussed:
\begin{enumerate}
	\item The preprocessing of raw data into SLC images, taking into account several effects due to the specific hardware design of the system.
	\item The polarimetric calibration of data into phase and amplitude calibrated polarimetric covariance matrices.
\end{enumerate}
\subsection{Preprocessing}
 The particular antenna design causes a frequency-depending shift of the antenna mainlobe during the chirp that causes a worsened range resolution. It is corrected using a slow time-fast time interpolation procedure; significant range resolution improvement are observed after the correction.\\ The real aperture, azimuth scanning design results in a motion of the antenna phase center relative to the scatterers, causing an observable azimuth phase ramp in point target responses. The variation is significantly different between the antennas, with almost 30$^\circ$ over the 3 dB beamwidth for the V antenna and much smaller for the H unit. This additional phase will complicate polarimetric calibration if left unaltered; this phase ramp is corrected by a SAR-like azimuth filter that reduces the total phase variation to under 10$^\circ$.\\
Because separated transmitting and receiving antennas are used for each polarization, the polarimetric calibration is more intricate due to the presence of an interferometric baseline between channels that adds a topographic phase contribution in the polarimetric phase differences. Using the cross-track interferometric baselines of KAPRI, the topographic contribution can be estimated and subtracted from each element of the covariance matrix affected by it.
\subsection{Calibration}
The resulting flattened covariance matrix is then calibrated  by assuming zero crosstalk and estimating copolar imbalances using a trihedral corner reflector assuming the parameters to be independent from the  incidence angle. The crosspolar imbalance is estimated using distributed targets under the assumption of reciprocity.\\
The calibration quality is assessed by estimating residual calibration parameter on a calibrated scene with five trihedral corner reflectors: the mean amplitude imbalance is close to unity while the mean residual phase imbalance is very close to zero, with an RMS of $7^\circ$; no significant variation with incidence angle is observed. These results suggest that the simplified calibration model\cite{Fore2015,Sarabandi1990} is suitable to calibrate fully polarimetric KAPRI data.