\section{Results}\label{sec:results}
\subsection{Frequency Squint Correction}\label{sec:results:squint_correction}
\begin{figure*}[ht]
	\centering
	\includegraphics{figure_12}
	\caption{Azimuth-frequency response of the "Hindere Chlapf" TCR: the raw data samples around the reflectors azimuth location were extracted, then filtered in range by Fourier transforming them along the frequency axis, appliyng an Hamming window about the range location and converting them  back into the time domain with an inverse Fourier transform. By doing so, only the portion of the range spectrum close to the reflectors location was kept. Finally, the complex envelope of the data was extracted using a discrete Hilbert transform. This is conceptually equivalent to the plot of \autoref{fig:squint_correction}. Panel (a) shows the result for the HH channel, (b) for the VV channel, (c) for the HH channel after the interpolation described in \autoref{sec:methods:squint_correction}  and (d) the same for the VV channel.}
	\label{fig:raw_squint}
\end{figure*}
Panels (a) and (b) of \autoref{fig:scene_mph} show the effect of the processing described in and \autoref{sec:methods:signal_model} and \autoref{sec:methods:squint_correction} ; in (a) the raw data is directly range compressed as is, while in (b) the frequency-squint compensation is applied to the data before range focusing.\\ 
In order to assess the processing quality and to appreciate the effect of the beam squint correction,  the oversampled impulse response of  TCR "Hindere Chlapf" at 673 m range is plotted in \autoref{fig:tcr_mph}. No azimuth averaging was applied to this data, to better visualize the effects described above. The plots on the left side (a), and (d) contain the response obtained by range compression of the acquired data without any transformation. The phase displays a cross-like pattern, centered on the maximum, presumably the effect of the frequency-dependent beam squint during the scan.\\ A visual comparison with the responses obtained enabling the squint compensation shows an improvement in range resolution, which is verified numerically by fitting a spline on the range response at the maximum and computing its $3 dB$ width. The azimuth resolution is also estimated in a similar manner.\\ The numerical results are displayed in each plot of \autoref{fig:tcr_mph}.
\begin{figure*}[ht]
	\centering
	\includegraphics{figure_1}
	\caption{Oversampled phase and amplitude responses for the corner reflector "Hindere Chlapf" at 673 m slant range. (a) HH channel without correction, (b) HH channel with frequency-dependent squint compensation (c) same as (b) with azimuth phase ramp removal. (d) VV channel without correction, (e) VV channel with frequency-dependent squint compensation (f) same as (e) with azimuth phase ramp removal.
	The phase of each response is referenced to its maximum.}
	\label{fig:tcr_mph}
\end{figure*}

\begin{figure*}[ht]
	\centering
	\includegraphics{figure_11}
	\caption{Phase and amplitude responses for a subset of the "Chutzen" calibration dataset, in radar coordinates. (a) VV with neither squint nor azimuth correction (b) VV channel with frequency-dependent squint compensation (c) same as (b) with additional azimuth phase ramp removal. A slight azimuth resolution deterioration is visible between (b) and (c); especially on sharp transitions between areas with high backscatter and shadow. However at this scale it is rather difficult to appreciate the subtle effects of the proposed corrections.}
	\label{fig:scene_mph}
\end{figure*}


\subsection{Azimuth Processing}\label{sec:results:azimuth_processing}
The ability of the phase model described in \autoref{sec:methods:azimuth_processing} to explain the observed phase variation  on the $VV$ channel response is tested on each reflector in the array: the maximum in range and azimuth was identified and the samples corresponding to the half power beamwidth were extracted at the range of maximum intensity. The unwrapped phase was then used to estimate $L_{ph}$ according to described in \autoref{eq:rph_estimation}.

\begin{table}[ht]
	\centering
	\pgfplotstabletypesetfile[			col sep=comma,
										columns={name, rsl, LphHH, LphVV},
										every head row/.style={before row=\toprule,after row=\midrule},
										every last row/.style={after row=\bottomrule},
										columns/name/.style={column name={Name} ,string type},
										columns/rsl/.style={precision=1, column name={$R_0 [m]$}},
										columns/LphHH/.style={precision=2, column name={$L_{ph}^{H} [m]$}},
										columns/LphVV/.style={precision=2, column name={$L_{ph}^{V} [m]$}},
										%columns/res/.style={precision=3, column name={residual}},
										]{../tab/table_2.csv}
	\caption{Result of the phase center displacement fit for six trihedral corner reflectors located at different ranges. In the first column, the estimated phase center displacements for the H antenna are shown, in the second the ones for the V unit.}
	\label{tab:rph_fit}
\end{table}
The resulting model parameters fit values for the H and V antennas are shown in \autoref{tab:rph_fit} alongside with the distance from the radar and the name of the reflector, defined in \autoref{tab:reflectors}. Owing to the lack of sufficiently bright crosspolarizing point targets, the equivalent horizontal phase center locations for HV and VH channels were not estimated from the data and are therefore not displayed in the table. In the following, their location it assumed to be at the midpoint between $L_{ph}^{H}$ and $L_{ph}^{V}$, which corresponds to the theoretical equivalent phase center for these channels.\\
The result of applying the correction of \autoref{eq:correction} to the TCR "Hindere Chlapf" at 673 m is displayed in panels (c) and (f) of \autoref{fig:tcr_mph}, this plot is produced by oversampling the SLC data around the location of the reflector using a cubic spline interpolator.
In \autoref{fig:phase_response_VV:corrected} the phase and amplitude response in both the HH and VV channel is plotted for all reflectors. To produce the plot, all responses were aligned in azimuth, normalized to the maximum and the phase referenced to the phase at closest approach. In this way, an easier comparison of the azimuthal phase variation is facilitated.

\begin{figure*}[ht]
	\centering
	\begin{subfigure}[t]{\textwidth}
		\centering
		\includegraphics{figure_2}
		\subcaption{Response without azimuth filtering.}
		\label{fig:phase_response_VV:uncorrected}
	\end{subfigure}\\
	\begin{subfigure}[t]{\textwidth}
		\centering
		\includegraphics{figure_3}
		\subcaption{Response after azimuth filtering.}
		\label{fig:phase_response_VV:corrected}
	\end{subfigure}
	\caption{Relative phase/amplitude response for all reflectors in the calibration array, continous line: VV channel, dashed: HH channel. To display the relative phase variation, the phase at the maximum is subtracted from each plot. The vertical lines indicate the theoretical 3 dB resolution of the antenna $\theta_{3dB}$. The responses for the HH channel are not plotted because no significant phase trend is observed.}
	\label{fig:phase_response_VV}
\end{figure*}

\subsection{Antenna Pattern Misalingment}\label{sec:results:misalignment}
\begin{figure*}[ht!]
	\centering
	\includegraphics{figure_4}
	\caption{Azimuth power response of a dihedral corner reflector, before and after the correction of antenna pattern mispointing. The observed gain is in line with the expected power loss due to the H and V patterns not perfectly overlapping.}
	\label{fig:hv_power}
\end{figure*}
To verify the impact of the H-V pattern pointing shift as described in \autoref{sec:methods:misalingment} on the performance of crosspolar measurements, the response of a dihedral reflector with an high crosspolar contribution is analyzed for two configurations:\\ \begin{enumerate*}\item the case where the antennas are not mechanically moved \item the case where the optimal shift of 1.8 mm, as described in \autoref{sec:methods:misalingment} is applied to the movable antenna hinge to bring the patterns into alignment.\\ 
\end{enumerate*}
In \autoref{fig:hv_power}, the result of the above experiment is shown as the oversampled, coregistered azimuth response in the HV channel.
\subsection{Removal of Topographic Phase}\label{sec:results:topo_removal}
The removal of the topographic phase contribution in the copolar phase difference, as described in \autoref{sec:methods:topo_removal}, is visualized in  \autoref{fig:topo_phase} by modulating the hue of the image with the HH-VV phase difference, its intensity with the corresponding $\mathbf{C}$ matrix element magnitude and by setting the saturation of the image by a nonlinear threshold function of the copolar coherence magnitude, estimated using a $\mathrm{5 \times 5}$ sample window. 
\begin{figure*}[hb]
	\centering
	\begin{subfigure}[t]{0.5\textwidth}
	\centering
	\includegraphics{figure_9}
	\subcaption{With topographic phase.}
	\label{fig:topo_phase:uncorrected}
	\end{subfigure}~
	\begin{subfigure}[t]{0.5\textwidth}
	\centering
	\includegraphics{figure_10}
	\subcaption{Without topographic phase.}
	\label{fig:topo_phase:corrected}
	\end{subfigure}
	\caption{HH-VV phase difference in radar coordinates, (a) before and (b)  after the removal of the topographic phase term as described in \autoref{sec:methods:topo_removal}. The hue of the image is modulated by the covariance phase, the intensity by the magnitude, the saturation by the copolar coherence magnitude, as shown in the bottom colorbar and plot. The locations and names of reflectors described in \autoref{tab:reflectors} are plotted. A fringe pattern is clearly visible in (a), which is removed by the proposed correction, as plotted in (b), leaving a phase offset that will be removed by the polarimetric calibration.}
	\label{fig:topo_phase}
\end{figure*}
\subsection{Polarimetric Calibration}\label{sec:results:proc_polcal}

								
\begin{table*}[h]
	\centering
	\pgfplotstabletypesetfile[
		col sep=comma,
		columns={name, slant range, HH-VV amplitude imbalance, HH-VV phase imbalance, Polarization purity},
		every head row/.style={before row=\toprule,after row=\midrule},
		every last row/.style={after row=\bottomrule},
		columns/name/.style={column name={Name} ,string type},
		columns/slant range/.style={precision=1, column name=\makecell{$R_{0} [m]$\\ Range distance}},
		columns/HH-VV amplitude imbalance/.style={precision=3, column name={$f$}},
		columns/HH-VV phase imbalance/.style={precision=3, column name={$\phi_r + \phi_t [^\circ]$}},
		columns/Polarization purity/.style={precision=3, column name={Purity $[dB]$}}
	]{../tab/table_3.csv}
	\caption{Copolar phase ($\phi_r + \phi_t$) and amplitude imbalance ($f$) computed on the reflectors using the calibrated dataset. The polarization purity (VV/HV ratio) is shown additionally. Results for the reflector used to determine calibration parameters are not shown.}
	\label{tab:polcal}
\end{table*}		
The methods described in the preceding sections were applied to prepare SLC images for each channel. For the final polarimetric calibration the procedure of \autoref{sec:methods:proc_polcal} was used; one reflector in the scene was used as a calibration target, with the four remaining reflectors used for the determination of the calibration performance.\\
An initial assessment of the data quality is made  by computing polarization signatures\cite{VanZyl1987} for two of the five corner reflectors that were not used to determine the calibration parameters. They are plotted in \autoref{fig:signatures}.
\begin{figure*}[hb]
	\centering
	\begin{subfigure}[t]{\textwidth}
	\centering
	\includegraphics{figure_5}
	\subcaption{Reflector "Hindere Chlapf" at 673 m.}
	\label{fig:signatures:near}
	\end{subfigure}\\
	\begin{subfigure}[t]{\textwidth}
	\centering
	\includegraphics{figure_6}
	\subcaption{Reflector "T\"{u}rle" at 2690 m.}
	\label{fig:signatures:far}
	\end{subfigure}
	\caption{Polarization signatures for two trihedral corner reflectors at the locations "Hindere Chlapf" and "T\"{u}rle". For both plots, each panel shows: (a) uncalibrated copolar signature, (b) uncalibrated crosspolar signature; (c) calibrated copolar signature,(d): calibrated crosspolar signature. The power of each response is normalized to the corresponding maximum. A striking change in signature shape is observed after the calibration; it is mostly due to the removal of the global HHVV phase offset.}
	\label{fig:signatures}
\end{figure*}

\begin{figure*}[ht!]
	\centering
	\includegraphics{figure_8}
	\caption{Dependence of the residual copolar phase ($\phi_r + \phi_t$) and amplitude  ($f$) imbalances on the local incidence angle. The mean and RMS imbalances are shown in each plot. The reflector used for the determination of calibration parameters has been excluded from the plot.}
	\label{fig:inc_angle_trend}
\end{figure*}
A quantitative evaluation of the calibration is obtained by estimating the residual copolar phase and amplitude imbalances $f$ and $\phi_r + \phi_t$ on the trihedral from calibrated data, excluding the reflector used to determine the parameter. The results are shown in \autoref{tab:polcal}.
In \autoref{fig:inc_angle_trend}, the dependence of the residuals on the local incidence angle is plotted; the angle was estimated using a 2 m posting digital elevation model of the scene that was backwards geocoded in the radar geometry.