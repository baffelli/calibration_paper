\section{Results}\label{sec:results}
\subsection{Beam Squint Correction}\label{sec:results:squint_correction}
\begin{figure*}[ht]
	\centering
	\includegraphics{figure_12}
	\caption{Azimuth-frequency response of the "Hindere Chlapf" TCR: the raw data samples around the reflectors azimuth location were extracted, then filtered in range by Fourier transforming them along the frequency axis, appliyng an Hamming window about the range location and converting them  back into the time domain with an inverse Fourier transform. By doing so, only the portion of the range spectrum close to the reflectors location was kept. Finally, the complex envelope of the data was extracted using a discrete Hilbert transform. This is conceptually equivalent to the plot of \autoref{fig:squint_correction}. Panel (a) shows the result for the HH channel, (b) for the VV channel, (c) for the HH channel after the interpolation described in \autoref{sec:methods:squint_correction}  and (d) the same for the VV channel.}
	\label{fig:raw_squint}
\end{figure*}
In \autoref{fig:raw_squint}, the chirp frequency-azimuth response of the "Hindere Chlapf" TCR is displayed. This plot was generated with the procedure described in \autoref{sec:methods:squint_correction} by filtering the range compressed around the location of the reflector and converting it back to the time domain by an inverse Fourier transform. In (a) and (c), the procedure was applied to the HH channel data before and after the squint correction interpolation; (b) and (d) show the same for the VV channel. In each plot, a line shows the results of the model fit described in \autoref{eq:squint_linearised}, the $a$ parameter is overlaid additionally.\\
The same fit procedure has been applied to all reflectors for both the HH and the VV data; the results are shown in \autoref{tab:a_squint_fit}.\\
\begin{table*}[ht]
	\centering
	\pgfplotstabletypesetfile[			col sep=comma,
										columns={name, a_HH, a_VV, a_res_HH, a_res_VV},
										every head row/.style={before row=\toprule,after row=\midrule},
										every last row/.style={after row=\bottomrule},
										columns/name/.style={column name={Name} ,string type},
										columns/a_HH/.style={precision=1, column name={$a_{HH}$}},
										columns/a_VV/.style={precision=1, column name={$a_{VV}$}},
										columns/a_res_HH/.style={precision=1, column name={residual $a_{HH}$}},
										columns/a_res_VV/.style={precision=1, column name={residual $a_{VV}$}},
										%columns/res/.style={precision=3, column name={residual}},
										]{../tab/table_squint.csv}
	\caption{Result of fitting the model of \autoref{eq:squint_linearised} to each reflector in the calibration array. In the second column, the $a$ parameter for the HH channel is shown, in the third the one for the VV channel. The last two columns show the same parameters re-estimated after applying the squint correction using 4.2 and 3.9 $\frac{\circ}{GHz}$ for $a_{HH}$ and $a_{VV}$ respectively.}
	\label{tab:a_squint_fit}
\end{table*}
In \autoref{fig:scene_mph}, panels (a),(b), (d) and (e) the effect of the squint and the result of the correction are visible on range compressed data for the "Hindere Chlapf" reflector. The plots are generated by oversampling the response of the range compressed TCR data in azimuth and range using a cubic order spline approximating a Sinc interpolator. In each plot azimuth and range resolutions are estimated numerically by fitting a spline on the response at the corresponding maximum and computing its 3 dB width. Panels (a) and (d) show the range compressed data for the HH and VV channels. In (b) and (e) the same is repeated after applying squint correction before range compression.
\begin{figure*}[ht]
	\centering
	\includegraphics{figure_1}
	\caption{Oversampled phase and amplitude responses for the corner reflector "Hindere Chlapf" at 673 m slant range. (a) HH channel without correction, (b) HH channel with frequency-dependent squint compensation (c) same as (b) with azimuth phase ramp removal. (d) VV channel without correction, (e) VV channel with frequency-dependent squint compensation (f) same as (e) with azimuth phase ramp removal.
	The phase of each response is referenced to its maximum.}
	\label{fig:tcr_mph}
\end{figure*}

\begin{figure*}[ht]
	\centering
	\includegraphics{figure_11}
	\caption{SLC data of a section of the "Chutzen" calibration dataset, represented in radar coordinates. The brightness corresponds to the intensity, the hue is modulated by the phase, according to the colormap shown above. Different stages of processing are depicted: (a) VV with neither squint nor azimuth correction (b) VV channel with frequency-dependent squint compensation (c) same as (b) with additional azimuth phase ramp removal. A slight azimuth resolution deterioration is visible between (b) and (c); especially on sharp transitions between areas with high backscatter and shadow. However at this scale it is rather difficult to appreciate the subtle effects of the proposed corrections.}
	\label{fig:scene_mph}
\end{figure*}


\subsection{Azimuth Processing}\label{sec:results:azimuth_processing}
The ability of the phase model described in \autoref{sec:methods:azimuth_processing} to explain the observed phase variation  on the $VV$ channel response is tested on each reflector in the array: the maximum in range and azimuth was identified and the samples corresponding to the half power beamwidth were extracted at the range of maximum intensity. The unwrapped phase was then used to estimate $L_{ph}$ according to described in \autoref{eq:rph_estimation}.

\begin{table}[ht]
	\centering
	\pgfplotstabletypesetfile[			col sep=comma,
										columns={name, rsl, LphHH, LphVV},
										every head row/.style={before row=\toprule,after row=\midrule},
										every last row/.style={after row=\bottomrule},
										columns/name/.style={column name={Name} ,string type},
										columns/rsl/.style={precision=1, column name={$R_0 [m]$}},
										columns/LphHH/.style={precision=2, column name={$L_{ph}^{H} [m]$}},
										columns/LphVV/.style={precision=2, column name={$L_{ph}^{V} [m]$}},
										%columns/res/.style={precision=3, column name={residual}},
										]{../tab/table_2.csv}
	\caption{Result of the phase center displacement fit for six trihedral corner reflectors located at different ranges. In the first column, the estimated phase center displacements for the H antenna are shown, in the second the ones for the V unit.}
	\label{tab:rph_fit}
\end{table}
The resulting model parameters fit values for the H and V antennas are shown in \autoref{tab:rph_fit} alongside with the distance from the radar and the name of the reflector, defined in \autoref{tab:reflectors}. Owing to the lack of sufficiently bright crosspolarizing point targets, the equivalent horizontal phase center locations for HV and VH channels were not estimated from the data and are therefore not displayed in the table. In the following, their location it assumed to be at the midpoint between $L_{ph}^{H}$ and $L_{ph}^{V}$, which corresponds to the theoretical equivalent phase center for these channels.\\
The result of applying the correction of \autoref{eq:correction} to the TCR "Hindere Chlapf" at 673 m is displayed in panels (c) and (f) of \autoref{fig:tcr_mph}, this plot is produced by oversampling the SLC data around the location of the reflector using a cubic spline interpolator.
In \autoref{fig:phase_response_VV:corrected} the phase and amplitude response in both the HH and VV channel is plotted for all reflectors. To produce the plot, all responses were aligned in azimuth, normalized to the maximum and the phase referenced to the phase at closest approach. In this way, an easier comparison of the azimuthal phase variation is facilitated.

\begin{figure*}[ht]
	\centering
	\begin{subfigure}[t]{\textwidth}
		\centering
		\includegraphics{figure_2}
		\subcaption{Response without azimuth phase correction.}
		\label{fig:phase_response_VV:uncorrected}
	\end{subfigure}\\
	\begin{subfigure}[t]{\textwidth}
		\centering
		\includegraphics{figure_3}
		\subcaption{Response after azimuth phase correction.}
		\label{fig:phase_response_VV:corrected}
	\end{subfigure}
	\caption{Relative phase/amplitude response for all reflectors in the calibration array, (a) no azimuth phase correction (b) after applying azimuth phase correction. Continuous lines: VV channel, dashed lines: HH channel. To display the relative phase variation, the phase at the maximum is subtracted from each plot. The vertical lines indicate the theoretical 3 dB resolution of the antenna $\theta_{3dB}$.}
	\label{fig:phase_response_VV}
\end{figure*}

\subsection{Antenna Pattern Misalignment}\label{sec:results:misalignment}
\begin{figure*}[ht!]
	\centering
	\includegraphics{figure_4}
	\caption{Oversampled azimuth power response of a dihedral corner reflector, before (green) and after the correction of antenna pattern mispointing (orange). The observed gain is in line with the expected power loss due to the H and V patterns not perfectly overlapping.}
	\label{fig:hv_power}
\end{figure*}
To verify the impact of the H-V pattern pointing shift as described in \autoref{sec:methods:misalingment} on the performance of crosspolar measurements, the response of a dihedral reflector with an high crosspolar contribution is analyzed for two configurations:\\ \begin{enumerate*}\item the case where the antennas are not mechanically moved \item the case where the optimal shift of 1.8 mm, as described in \autoref{sec:methods:misalingment} is applied to the movable antenna hinge to bring the patterns into alignment.\\ 
\end{enumerate*}
In \autoref{fig:hv_power}, the result of the above experiment is shown as the oversampled, coregistered azimuth response in the HV channel.
\subsection{Removal of Topographic Phase}\label{sec:results:topo_removal}
The removal of the topographic phase contribution in the copolar phase difference, as described in \autoref{sec:methods:topo_removal}, is visualized in  \autoref{fig:topo_phase} by modulating the hue of the image with the HH-VV phase difference, its intensity with the corresponding $\mathbf{C}$ matrix element magnitude and by setting the saturation of the image by a nonlinear threshold function of the copolar coherence magnitude, estimated using a $\mathrm{5 \times 5}$ sample window. 
\begin{figure*}[hb]
	\centering
	\begin{subfigure}[t]{0.5\textwidth}
	\centering
	\includegraphics{figure_9}
	\subcaption{With topographic phase.}
	\label{fig:topo_phase:uncorrected}
	\end{subfigure}~
	\begin{subfigure}[t]{0.5\textwidth}
	\centering
	\includegraphics{figure_10}
	\subcaption{Without topographic phase.}
	\label{fig:topo_phase:corrected}
	\end{subfigure}
	\caption{HH-VV phase difference in radar coordinates, (a) before and (b)  after the removal of the topographic phase term as described in \autoref{sec:methods:topo_removal}. The hue of the image is modulated by the covariance phase, the intensity by the magnitude, the saturation by the copolar coherence magnitude, as shown in the bottom colorbar and plot. The locations and names of reflectors described in \autoref{tab:reflectors} are plotted. The interferometric fringe pattern visible in (a) is removed by the proposed correction, as plotted in (b), leaving a phase offset that will be removed by the polarimetric calibration.}
	\label{fig:topo_phase}
\end{figure*}
\subsection{Polarimetric Calibration}\label{sec:results:proc_polcal}

								
\begin{table*}[h]
	\centering
	\pgfplotstabletypesetfile[
		col sep=comma,
		columns={name, slant range, HH-VV amplitude imbalance, HH-VV phase imbalance, Polarization purity},
		every head row/.style={before row=\toprule,after row=\midrule},
		every last row/.style={after row=\bottomrule},
		columns/name/.style={column name={Name} ,string type},
		columns/slant range/.style={precision=1, column name=\makecell{$R_{0} [m]$\\ Range distance}},
		columns/HH-VV amplitude imbalance/.style={precision=3, column name={$f$}},
		columns/HH-VV phase imbalance/.style={precision=3, column name={$\phi_r + \phi_t [^\circ]$}},
		columns/Polarization purity/.style={precision=3, column name={Purity $[dB]$}}
	]{../tab/table_3.csv}
	\caption{Copolar phase ($\phi_r + \phi_t$) and amplitude imbalance ($f$) computed on the reflectors using the calibrated dataset. The polarization purity (VV/HV ratio) is shown additionally. Results for the reflector used to determine calibration parameters are not shown.}
	\label{tab:polcal}
\end{table*}		
The methods described in the preceding sections were applied to prepare SLC images for each channel. For the final polarimetric calibration the procedure of \autoref{sec:methods:proc_polcal} was used; one reflector in the scene was used as a calibration target, with the four remaining reflectors used for the determination of the calibration performance.\\
An initial assessment of the data quality is made  by computing polarization signatures\cite{VanZyl1987} for two of the five corner reflectors that were not used to determine the calibration parameters. They are plotted in \autoref{fig:signatures}.
\begin{figure*}[hb]
	\centering
	\begin{subfigure}[t]{\textwidth}
	\centering
	\includegraphics{figure_5}
	\subcaption{Reflector "Hindere Chlapf" at 673 m slant range.}
	\label{fig:signatures:near}
	\end{subfigure}\\
	\begin{subfigure}[t]{\textwidth}
	\centering
	\includegraphics{figure_6}
	\subcaption{Reflector "T\"{u}rle" at 2690 m slant range.}
	\label{fig:signatures:far}
	\end{subfigure}
	\caption{Polarization signatures for two trihedral corner reflectors at the locations "Hindere Chlapf" and "T\"{u}rle". For both plots, each panel shows: (a) uncalibrated copolar signature, (b) uncalibrated crosspolar signature; (c) calibrated copolar signature,(d): calibrated crosspolar signature. The power of each response is normalized to the corresponding maximum. A distinct change in signature is observed after the calibration; it is mostly due to the removal of the HH-VV phase offset.}
	\label{fig:signatures}
\end{figure*}

\begin{figure*}[ht!]
	\centering
	\includegraphics{figure_8}
	\caption{Dependence of the residual copolar phase ($\phi_r + \phi_t$) and amplitude  ($f$) imbalances on the local incidence angle. The mean and RMS imbalances are shown in each plot. The reflector used for the determination of calibration parameters has been excluded from the plot.}
	\label{fig:inc_angle_trend}
\end{figure*}
A quantitative evaluation of the calibration is obtained by estimating the residual copolar phase and amplitude imbalances $f$ and $\phi_r + \phi_t$ on the trihedral from calibrated data, excluding the reflector used to determine the parameter. The results are shown in \autoref{tab:polcal}.
In \autoref{fig:inc_angle_trend}, the dependence of the residuals on the local incidence angle is plotted; the angle was estimated using a 2 m posting digital elevation model of the scene that was backwards geocoded in the radar geometry.