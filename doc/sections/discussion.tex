\section{Discussion}\label{sec:discussion}
\subsection{Beam Squint Correction}\label{sec:discussion:squint_correction}
%The effect of correcting the frequency-dependent beam squint is not easily visible in the SLC data displayed in \autoref{fig:scene_mph}. Therefore, in the rest of this section the response of single features in the scene will be analyzed.
The effect of the frequency squint on the raw data around TCR "CR2" is visible in \autoref{fig:raw_squint}, panels (a) and (b) for the HH and VV channels. As described in 
\autoref{sec:methods:squint_correction} and sketched in \autoref{fig:squint_correction}, the data matrix appears skewed:  the physical antenna direction and the effective pointing angle of the beam pattern only match for a brief time during each chirp due to the frequency scanning of the antenna. Because of that, if the data is range compressed, only part of the chirp bandwidth illuminates the target at each time, reducing the observed range resolution. This is verified in \autoref{fig:tcr_mph} subfigures (a) and (c), where the oversampled response of TCR "CR2" after range compression is shown.\\
The linear squint factor $a$ estimated on all reflectors of the calibration array is given in \autoref{tab:a_squint_fit};
the average estimated values of 4 and 3.9 $\frac{\circ}{GHz}$  fit well with the figures suggested by the antenna manufacturer: 4.2 and 3.9 $\frac{\circ}{GHz}$ for the H and V antennas respectively.\\
Thanks to the oversampled acquisition it is possible to use the proposed interpolation method to realign the samples in azimuth, compensating the effect of the squint by combining subsequent sub-chirps with different squint angles in a single coherent chirp that covers the entire bandwidth for the whole duration of time when the target is within the antenna beamwidth. This is shown in \autoref{fig:raw_squint} panels (b) and (d). The result of the interpolation is visible in \autoref{fig:raw_squint}, panels (c) and (d): the spectrum is now aligned in azimuth; as a consequence the range resolution is decreased, as visible in \autoref{fig:tcr_mph}. Additionally the phase response seems to become more stable.
Visually, the phase pattern observed in (a) and (c) is also removed entirely; however, a residual azimuthal phase can be observed in the VV channel in (e).
\subsection{Azimuth Processing}\label{sec:discussion:azimuth_processing}
The residual phase ramp observed on the "CR2" reflector in \autoref{sec:discussion:squint_correction} is not unique to that object:\\ A linear variation of 30 degrees over the 3dB antenna beamwidth can be observed for all reflectors in the dataset, as plotted in \autoref{fig:phase_response_VV:uncorrected}.  The model of \autoref{sec:methods:azimuth_processing} was developed to explain this variation in terms of the acquisition geometry.
The estimated phase center location values $\hat{L}_{ph}$ of \autoref{tab:rph_fit} are consistent and display a standard deviation of less than 2 \% of the antenna length, suggesting that the model is able to predict most of the phase variation. The estimated phase center shift for the H unit $\hat{L}_{ph}^{H}$ is less than -5 cm, wile the one for the V antenna, $\hat{L}_{ph}^{V}$, is -12 cm, almost twice as large and with opposite sign. This difference presumably explains the visibility of the azimuthal phase ramp in the VV channel:\\
For a specific target,  $\theta_r$ is limited to the time where it is inside the antenna beamwidth, $\theta_{3dB}$. This small value would not produce any appreciable phase variation. However for increasing values of $L_{ph}$, the entire function of \autoref{eq:range_phase} is rapidly shifted  by a large $\alpha$, simulating the effect of a larger $\theta_r$, as it would be obtained with a much bigger antenna beamwidth.\\
As shown in panel (g) of \autoref{fig:tcr_mph}, the azimuth phase variation is significantly reduced by the proposed correction method, while the azimuth resolution is slightly degraded, while still being less than $0.6^\circ$, the value that would be expected if the samples were to be incoherently integrated. This result hints again that most of the phase ramp has been removed by the proposed method.\\
Similar results are observed for all trihedral reflectors, summarized in \autoref{fig:phase_response_VV:corrected}.  Generally, the HH data shows a much smaller variation, while the VV phase displays an almost linear increase of 30$^\circ$. As expected from the estimated phase center locations of \autoref{tab:rph_fit}, the observed phase slopes for the V and H have opposite signs.\\  After the correction, only the reflector "Chutzen" shows a large residual azimuth phase variation; its closeness to the radar relative to the far field transition distance of the antennas in the order of 500 m could explain the discrepancy: a non-linear  phase variation is already observed before the correction. Excluding this exception, the residual variation is under $5^\circ$ for samples inside the antenna beamwidth, marked by vertical lines in the plot.
\subsection{Antenna Pattern Misalignment}\label{sec:discussion:misalignment}
Correcting for the estimated pattern alignment results in a gain of approx. 2.5 dB  in HV power with respect to the uncompensated reference case, as plotted in \autoref{fig:hv_power}. The estimated gain is very close to the one expected by analyzing the patterns provided by the antenna manufacturer. This results confirms that the 1.8 mm shift setting employed to acquire the calibration data can correctly compensate the crosspolar power loss and the HH-VV misregistration.
\subsection{Removal of Topographic Phase}
The effect of the topographic phase compensation as described in \autoref{sec:methods:topo_removal} is displayed in \autoref{fig:topo_phase};
three topographic fringes are counted in the unflattened interferogram (\autoref{fig:topo_phase:uncorrected}), corresponding to a total phase variation of $9 \pi$ . An estimate of this contribution is obtained from the unwrapped and rescaled interferogram between the upper and the lower HH channels. Because they are separated by a spatial baseline and they employ the same polarization (as seen in \autoref{fig:antenna_arrangement}), this interferogram provides an estimate of the topographic phase without additional polarimetric phase differences. The estimated topographic phase is then subtracted from the covariance matrix; the resulting flattened copolar phase difference is displayed in \autoref{fig:topo_phase:corrected}; no interferometric fringes can be counted. A more quantitative evaluation of the flattening process is obtained computing the correlation coefficient of the copolar phase and $\sin{\left(\theta_l\right)}$, the look vector elevation angle obtained from a DEM. The estimated value is very close to zero, suggesting the correct removal of the topgraphic phase contribution:\\ if there would be a linear relationship between these two quantities, as expected from the approximate  expression for the topographic phase of \eqref{eq:prop_approximation}, a significant level of correlation would be observed. This measure cannot exclude  residual nonlinear relationships between the topography and the copolar phase; an additional visual verification was obtained by plotting their joint histogram; this did  not display any discernible functional relationship. Therefore, it can be safely assumed that the flattening process is able to remove the topographic phase contribution.
%The correction quality is also analyzed in \autoref{fig:topo_correlation} by plotting the joint histogram of the HH-VV phase versus the look vector elevation angle computed using an external digital elevation model. In \autoref{fig:topo_correlation:uncorrected} the correlation is visible as a line. They are the effect of topographic phase wrapping in the HH-VV phase difference; the correlation entirely disappears after subtracting the topographic contribution (\autoref{fig:topo_correlation:corrected}). 
\subsection{Polarimetric Calibration}
An illustration of the effect of the calibration procedure of \autoref{sec:methods:proc_polcal} is given in  \autoref{fig:signatures} by the polarization signatures of two reflectors before and after the calibration.
The signature for the uncorrected data is reminscent of the response of a dihedral scatterer. After calibration, both signatures show a correspondence with the expected polarization signature for trihedral reflectors. The dihedral-like response before the calibration is due to the HH-VV phase imbalancee, which shows a significant offset after the removal of the topographic contribution. By using a trihedral corner reflector or a similar scatterer, this offset is estimated and removed.\\
Amplitude and  phase imbalances are successfully corrected by the proposed method; the mean residual for the amplitude imbalance $f$ is 1.03 with a root mean square deviation of 0.05. A mean of $-4.5^\circ$ is observed for the copolar phase imbalance $\phi_r + \phi_t$ , with a root mean square deviation of $7^\circ$. Two outliers, "CR1" and "CR6" heavily skew the estimated phase statistics; as shown in \autoref{fig:inc_angle_trend}. For "CR1", its placement on the ground together with the extreme closeness to the antennas could explain the offset in the copolar phase. The other extreme case is "CR6", that was placed at a large distance from the radar, this could result in a reduced polarization purity due to the larger influence of clutter in the larger resolution cell.\\ 
Despite the outliers, no dependence of the residuals with the local incidence angle can be observed.
This also suggests the validity of the method proposed in \autoref{sec:methods:proc_polcal} for the estimation and removal of the topographic contribution from the polarimetric phase differences.\\Finally, the assumption of negligible crosstalk for the calibration model seems plausible considering that almost all reflectors exhibit polarization purities above 35 dB.
