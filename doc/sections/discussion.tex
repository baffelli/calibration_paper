\section{Discussion}\label{sec:discussion}
\subsection{Frequency Squint Correction}\label{sec:discussion:squint_correction}
The effect of the frequency squint on the raw data is visible in \autoref{fig:raw_squint}, panels (a) and (b) for the HH and VV channels. As described in 
\autoref{sec:methods:squint_correction} and sketched in \autoref{fig:squint_correction}, the data matrix appears skewed:  the physical antenna direction and the effective pointing angle of the beam pattern only match for a brief time during each chirp due to the frequency scanning of the antenna. Because of that, if the data is range compressed, only part of the chirp bandwidth illuminates the target at each time, reducing the observed range resolution. This is verified in \autoref{fig:tcr_mph} subfigures (a) and (c), where the oversampled response of TCR "Hindere Chlapf" after range compression is shown.\\
The linear squint factor is estimated on all reflector of the calibration array is shown in \autoref{tab:a_squint_fit};
the estimated values fit quite well with the figures suggested by the antenna manufacturer: 4.2 and 3.9 $\frac{\circ}{GHz}$ for the H and V antennas respectively.\\
Thanks to the oversampled acquisition it is possible to use the proposed interpolation method to realign the samples in azimuth, compensating the effect of the squint by combining subsequent sub-chirp with different squint angles in a single coherent chirp that covers the entire bandwidth for the whole duration of time when the target is within the antenna beamwidth. This is shown in \autoref{fig:raw_squint} panels (b) and (d). The result of the interpolation is visible in \autoref{fig:raw_squint}, panels (c) and (d), the spectrum is now aligned; as a consequence the range resolution is increased, as visible in \autoref{fig:tcr_mph}; additionally the phase response seems to become more stable.
Generally, the effect of correcting the frequency-dependent squint immediately visible in the SLC data, as plotted in \autoref{fig:scene_mph}. However, when the response of the  is oversampled and plotted, (\autoref{fig:tcr_mph}), significant changes are readily observed:
In the uncorrected data of panels (a) and (c), the phase displays a cross-like pattern, centered on the maximum, presumably the effect of the frequency-dependent beam squint during the scan. After applying the frequency-azimuth inteprolation on the raw data and range compressing the corrected data, significant improvements in range resolution are measured. Visually, the phase pattern observed in (a) and (c) is also removed entirely; however, a residual azimuthal phase can be observed in the VV channel in (e).
\subsection{Azimuth Processing}\label{sec:discussion:azimuth_processing}
The residual phase ramp observed on the "Hindere Chlapf" reflector in \autoref{sec:discussion:squint_correction} is not an unique feature of that object: A linear variation of 30 degrees over the 3dB antenna beamwidth can be observed for all reflectors in the dataset, as plotted in \autoref{fig:phase_response_VV:uncorrected}.  The model of \autoref{sec:methods:azimuth_processing} was developed to explain this variation in terms of the acquisition geometry.
The estimated phase center location values $\hat{L_{ph}}$ of \autoref{tab:rph_fit} appear to be consistent and display a standard deviation of less than 2 \% of the antenna length, suggesting that the model is able to predict most of the phase variation. The estimated phase center shifts for the H unit $\hat{L_{ph}^{H}}$ is less than -5 cm wile the one for the V antenna $\hat{L_{ph}^{V}}$ is almost twice as large and with opposite sign, at approx. 12 cm. This difference presumably explains the visibility of the azimuthal phase ramp in the VV channel only:\\
For a specific target,  $\theta_r$ is limited to the time where it is inside the antenna beamwidth, $\theta_{3dB}$. This small value would not produce any appreciable phase variation. However for increasing values of $L_{ph}$, the entire function of \autoref{eq:range_phase} is rapidly shifted  by a large $\alpha$, simulating the effect of a larger $\theta_r$, as it would be obtained with a much bigger antenna beamwidth.\\
As shown in panel (g) of \autoref{fig:tcr_mph} the azimuth phase variation is significantly reduced by the proposed correction method, while the azimuth resolution is slightly degraded, while still being less than $0.6^\circ$, the value that would be expected if the samples were to be incoherently integrated. This result hints again that most of the phase ramp has been removed by the proposed method.\\
Similar results are observed for all TCRs, as displayed in \autoref{fig:phase_response_VV:corrected}.  Generally, the HH data shows a much smaller variation, while the VV phase displays an almost linear increase of 30$^\circ$. As expected from the estimated phase center locations of \autoref{tab:rph_fit}, the observed phase slopes for the V and H have opposite signs.\\  After the correction, only the reflector "Chutzen" shows a large residual azimuth phase variation; this behavior is attributed to its closeness to the radar relative to the far field transition distance of the antennas, in the order of 500 m. Excluding this exception, the residual variation is under $5^\circ$ for samples inside the antenna beamwidth.
\subsection{Antenna Pattern Misalingment}\label{sec:discussion:misalignment}
Correcting for the estimated pattern alignment results in a gain of approx. 2.5 dB HV power with respect to the uncompensated reference case, as plotted in \autoref{fig:hv_power}. The estimated gain is very close to the one expected by analyzing the provided antenna patterns. This results confirms that the 1.8 mm shift setting employed to acquire the calibration data can correctly compensate the crosspolar power loss and the HH-VV misregistration.
\subsection{Removal of Topographic Phase}
The effect of the topographic phase compensation as described in \autoref{sec:methods:topo_removal} is quite dramatic;
when the topographic contribution is not compensated for, fringes are clearly visible (\autoref{fig:topo_phase:uncorrected}). An estimate of this contribution is obtained by the unwrapped and rescaled interferogram between the upper and the lower HH channels, which due to their being separated by a spatial baseline (as seen in \autoref{fig:antenna_arrangement}) provides an estimate of the topographic phase without additional polarimetric phase differences. The contribution thus obtained is then subtracted from the covariance matrix; the resulting flattened copolar phase difference is displayed in \autoref{fig:topo_phase:corrected}; it is visually free of topographic contributions.
\subsection{Polarimetric Calibration}
An illustration of the effect of the calibration procedure of \autoref{sec:methods:proc_polcal} is given in  \autoref{fig:signatures} by the polarization signatures of two reflectors before and after the calibration.
The signature for the uncorrected data does not correspond to the signatures of any basic scattering mechanism. In the calibrated response, both reflectors show a correspondence with the expected polarization signature for trihedral reflectors.\\ The dramatic change in shape of the signatures is mainly to be attributed to the correction of the $HH-VV$ phase, which shows a significant offset after the removal of the topographic contribution. By using a trihedral corner reflector or a similar scatterer, this offset is estimated and removed.\\
This interpretation is supported by the residual imbalance table \autoref{tab:polcal}.
Generally, both the amplitude and the phase imbalance appear to be well corrected. The mean residual for the amplitude imbalance $f$ is 1.03 with an RMS of 0.05. Similarly, for the residual phase difference a mean of $1.19^\circ$ is observed, with a RMS of $7^\circ$. The biggest outlier for $f$ is represented by reflector "Chutzen", placed very close to the radar, at 73 m slant range. As previously discussed, it is not located sufficiently far away to be in the far field region of the antennas; this could explain the discrepancy from the rest of the array. In the case of the copolar phase imbalanace $\phi_r + \phi_t$, the results appear to be correlated with the polarization purity of each reflector; the largest outlier being the TCR "T\"{u}rle" located at the largest distance from the radar and having the smallest estimated purity. The lower level of purity, could imply larger clutter contributions in the resolution cell, which would increase the phase bias.\\
The assumption of negligible crosstalk seems plausible considering that all reflectors  except the very far one show a polarization purity above 35 dB. In the former case, the combination of large azimuth resolution cell size, the relatively low RCS of the TCR and  the difficulty of exactly pointing it towards the radar may have lowered the purity because of a larger influence of clutter in the cell. This observation is reflected in the fact that the largest copolar phase residual is observed at this location.\\
No clear trend for $f$ and $\phi_r + \phi_t$ residuals as a function of incidence angle is observed (\autoref{fig:inc_angle_trend}); however it is rather difficult to judge this observation given the small number of data points available and the rather limited extent of covered incidence angles. However, given the small RMS and mean residual errors, the choice of a model that does not consider variation of calibration parameters with incidence angle appears to be justified. The absence of clear incidence-angle related trends is also seen as hinting the appropriateness of the method proposed in \autoref{sec:methods:proc_polcal} to remove the topographic contribution from the polarimetric phase differences: larger variation would otherwise be observed.\\
