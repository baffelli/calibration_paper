\documentclass[journal, 10pt]{IEEEtran}
\usepackage[utf8]{inputenc}
\usepackage{amsmath}
\usepackage{booktabs}
\usepackage{array} 
\usepackage{fullpage}
\usepackage{rotating}
\usepackage[section]{placeins}
\usepackage{cite}
\usepackage{amsthm}
\usepackage{subcaption}
\usepackage{amsfonts}
\usepackage{graphicx}
\usepackage{tikz}
\usetikzlibrary{positioning}
\usepackage[colorlinks=false]{hyperref}
\hypersetup{ hidelinks = true, }
\usepackage{multirow}
\usepackage{datatool}
\usepackage{pgfplotstable}
\usepackage[inline]{enumitem}
\usepackage{makecell}
\usepackage{tabularx}

%Redefine citation
%\renewcommand{\eqautorefname}{Fig.} 

\input{/home/baffelli/PhD/trunk/Texts/macros.tex}
\graphicspath{{../outputs/img/}{../drawings/pdf/}{../fig/pdf/}}
%Title and authors
\title{Polarimetric Calibration of the Ku Band Polarimetric Terrestrial Radar Interferometer (KAPRI)}
\author{\IEEEauthorblockN{
	Simone Baffelli,
	Othmar Frey,
	Charles Werner,
	Irena Hajnsek
}}

%\IEEEauthorblockA{\IEEEauthorrefmark{1}Affiliate 1, Adress}


\begin{document}
\maketitle
\begin{abstract}
Differential interferometry using ground-based radar systems permits to monitor fast displacements in natural terrain with a high degree of flexibility in terms of the choice of location, time of acquisition and revisit time. In combination with polarimetric imaging, discrimination of different scattering mechanisms present in a resolution cell can be obtained while estimating the surface displacement.
In this paper we present the pre-processing steps and the calibration procedure required to produce high-quality calibrated polarimetric single-look complex imagery with KAPRI, a new portable Ku-band polarimetric radar interferometer. The processing of KAPRI data into SLC images is addressed, including the correction of beam squint and of azimuthal phase variations. A polarimetric calibration model adapted to the acquisition mode is presented and used to produce calibrated polarimetric covariance matrix data. The methods are tested on a validation dataset using a scene containing five trihedral corner reflectors. The data processing is assessed by analyzing the oversampled response of a corner reflector, the polarimetric calibration quality is verified by computing polarimetric signatures and residual calibration parameters, which show a RMS under $\mathrm{7^\circ}$ for the residual HH-VV phase imbalance after the calibration.
\end{abstract}
\section{Introduction}
\PARstart{D}{ifferential} radar interferometry\cite{Gabriel1989, Massonnet1993,Rosen2000,Bamler1999} is widely used  to monitor and study changes in the natural and built environment. The ability to accurately measure movements along the line of sight over large areas makes them suitable for many  applications. Some examples are the estimation of subsidence rate associated with groundwater or oil extraction and the study of aquifer dynamics\cite{Hudnut1998,Strozzi2001,Galloway2007a}, the monitoring of inflation/deflation connected to volcanic activity\cite{Massonnet1995}, the mapping of ice sheet and glacier motion\cite{Goldstein1993,Mohr1998} , the observation of landslides and instable slopes\cite{Carnec1996,Catani2005} and the measurement of seismic displacements\cite{Massonnet1993b,Zebker1994}.\\
Fully polarimetric radar data provides additional information on the scattering mechanism within each resolution cell, which are employed for classification of the surface cover\cite{Cloude1997, Lee1999}, to extract geophysical parameters such as moisture content\cite{Hajnsek2003} or information on the vegetation \cite{Ulaby1987} or the height of fresh snow\cite{Leinss2014}.\\
The combination of polarimetric and differential interferometric measurement could bring additional benefits for the observation of natural processes; for example by selecting the scattering mechanism that provides the best coherence and thus the least noisy phase measurement\cite{Pipia2009a, Iglesias2014b}.\\
Currently, the majority of radar data employed for differential interferometry is acquired using sensors carried by satellite or aircrafts. These platform are convenient in that they offer a large coverage in a single pass. Nevertheless, because of costs, technical, physical and logistical restrictions, the revisit time of these system is limited to few hours at best. 
In many cases, to better understand the dynamics of natural processes and for real time surveillance and alarming, a denser temporal sampling over longer time spans is desired than the ones afforded by current radar earth observation systems.\\
\subsection{State of the Art}
Several ground based radar systems were developed for these situations, operating in C \cite{Leva2003, Rudolf1999a,Kang2009}, X \cite{Aguasca2004,Pipia2007a} or Ku band \cite{Leva2003, Rudolf1999a,Werner2008, Rodelsperger2012}. The majority of these systems are based on aperture synthesis, using a moving antenna assembly on a motorized rail. 
An alternative to aperture synthesis is to rotationally scan an electrically large fan beam antenna, resolving  scatterers by using its narrow azimuth beam\cite{Werner2008,werner_gpri_2012}. This imaging method has been called type II in \cite{Caduff2015}. 
This configuration has certain advantages over ground-based synthetic aperture systems\cite{Monserrat2014} of comparable rail length: \begin{enumerate*}
  \item Unlike a SAR system, each azimuth sample is 
acquired independently, eliminating the problem of decorrelation caused by moving targets and atmospheric phase screens during the acquisition time. 
These changes destroy the coherence of the scene over the aperture length and will spread these scatterer in azimuth, with a severe impact on image quality, especially regarding the stability of coherent targets.\\
 \item A very large 
angular section can be imaged in a single pass, potentially up to 360$^{\circ}$. This is harder to achieve with a rail-based synthetic aperture radar, where usually only a smaller angular sector
can be covered at once.\\
\item An electrically large antenna provides an higher gain and hence a better SNR; this is not the case for  aperture synthesis  where the antenna has to be electrically small to provide a sufficiently wide beam.\\
\end{enumerate*}\\
The majority of polarimetric ground based radar system is based on the aperture synthesis principle. An indoor system is presented in \cite{Bennett1996} followed by a portable outdoors version \cite{Bennett2000}. A broadband polarimetric SAR system with two dimensional aperture synthesis is introduced in \cite{Zhou2004}, with measurement results presented in \cite{Hamasaki2005}.
Another example  of a synthetic aperture system is is UPCs RiskSAR \cite{Iglesias2014, Aguasca2004,Pipia2007a,Pipia2009, Pipia2013,Iglesias2014}. Example of ground based SAR polarimetric data at X and C band are shown in \cite{Kang2009, Kang2010}.\\ A dual polarization, multiband GB-SAR system is used in \cite{Yitayew2014} to produce tomograms of snow covered sea ice. A similar concept is used in\cite{Frey2015,Frey2016} to produce full polarimetric tomographic profiles of a snow pack by synthesizing an aperture in the elevation direction.\\
Less research is available on real aperture polarimetric systems, excluding non-imaging devices such as ground based scatterometers. One of the few examples is a C-Band version of the GPRI\cite{Cherukumilli2012}.
\subsection{KAPRI: Real Aperture Polarimetric FMCW Radar}
This paper will focus on KAPRI, the Ku-Band Polarimetric Advanced Radar Interferometer (KAPRI)\cite{Baffelli2016a}. It is an extension of GPRI \cite{werner_gpri_2012,Strozzi2011, Werner2008}, a portable real-aperture, $K_u$ band radar interferometer operating at 17.2 GHz with a wavelength of 1.74 cm, designed 
for the monitoring of unstable slopes using zero baseline differential interferometry \cite{Massonnet1993}. It is equipped with two antennas arranged along a spatial baseline and a dual channel receiver to derive local digital elevation models using single pass interferometric methods.\\
It employs  2 meter long, vertically polarized slotted waveguide antennas, giving the system an azimuth 3dB beamwidth of $0.385^\circ$ and an elevation beamwidth of $35^\circ$.\\
In terms of hardware, the feature distinguishing KAPRI from its predecessor is the addition of horizontally polarized antennas and switches that permit to connect transmitter and receiver to either type of antenna. Together with modifications in the control software, they enable it to acquire a full polarimetric-interferometric dataset by cycling through all the combinations of transmitted and received polarization during the acquisition.
\begin{table}
	\centering
	\pgfplotstabletypeset[
		col sep=&,	% specify the column separation character
		row sep=\\,	% specify the row separation character
		columns/Parameter/.style={string type}, % specify the type of data
		columns/Value/.style={string type}, % specify the type of data for all columns
		every head row/.style={before row=\toprule,after row=\midrule},
		every last row/.style={after row=\bottomrule},
	]{
		Parameter & Value\\
		Modulation & FM-CW 250 $\mu s$ to 16 $ms$\\
		Center frequency & 17.2 GHz\\
		Bandwidth & 200 MHz\\
		Range resolution& 0.95 @ -26 dB peak sidelobe\\
		Azimuth pattern beamwidth& $0.385^\circ$ 3dB beamwidth\\
		Elevation pattern beamwidth& $35^\circ$ 3dB beamwidth\\
	}
	\caption{Summary of main KAPRIs parameters.}
\end{table}
\subsection{Contributions of This Paper}
The following contributions are made in this paper:
\begin{enumerate}
	\item Preprocessing methods adapted to KAPRIs hardware are presented, that can be used to generate correct SLC images from the acquired raw data.
	\item A polarimetric calibration model suitable to the systems design is presented, that includes the correction of effects due to differences in polarized antenna design and the presence of spatial baselines between the polarimeteric phase centers.
	\item The validity and quality of the proposed solutions is  quantified by analyzing the response of trihedral corner reflectors in a specifically acquired dataset.
\end{enumerate}
\subsection{Outline}
Part \ref{sec:methods:signal_model} presents  the methods employed to process the raw data into range compressed SLC. This part includes a derivation of the FMCW signal model and of the acquisition geometry, that will be used throughout the rest of this paper. \autoref{sec:methods:squint_correction} deals with the correction of frequency-dependent beam squint due to the slotted waveguide antenna design. These two sections describe the parts of the processing that are common to both KAPRI and GPRI. The quality of the processing is evaluated in \autoref{sec:results:squint_correction} by plotting the oversampled phase and amplitude response of trihedral corner reflectors, where significant range resolution improvements are observed by applying the described squint compensation procedure.\\
After this step, the range compressed, squint-corrected data shows a residual azimuth phase variation, especially in the VV channel where a linear variation of almost $30^\circ$ is observed inside the antenna beamwidth. This effect is modeled in \autoref{sec:methods:azimuth_processing} as a change in distance between the antennas phase center and the scatterers due to the rotational acquisition geometry. A method to correct it is proposed and tested in \autoref{sec:results:azimuth_processing} on an array of five trihedral corner reflectors.\\ An azimuthal shift between the HH and the VV channel is observed on the  response of point targets along with the phase variationp; it is attributed to misaligned antenna patterns. If left uncorrected, it would cause a reduced power and increased SNR for crosspolar measurements. To remove it, modified antenna mounts that permit to mechanically shifting the antennas mainlobe were manufactured. They are tested in \autoref{sec:results:misalignment} by analyzing the response of a crosspolarizing dihedral reflector acquired with different antenna mounting settings.
The data thus processed is ready for polarimetric calibration; however because KAPRIs employs spatially separated antennas for each polarisation both in transmission and reception, spatial baselines are obtained between certain combinations of channels that cause an additional topographical contribution in the polarimetric phase differences. In \autoref{sec:methods:topo_removal} a method is derived to estimate this contribution using an interferogram obtained from two identically polarized channel on a baseline and rescaled to the undesired polarimetric baseline. Its validity is verified by analyzing the resulting HH-VV phase difference in \autoref{sec:results:topo_removal}.\\
For the polarimetric calibration, a linear distortion model without crosstalk is assumed; the copolar phase and amplitude imbalances are estimated using a trihedral corner reflector, while the imbalance between the crosspolar channel is determined using distributed scatterers assuming reciprocity. Finally, the quality of data calibration is assessed by computing polarization signatures for the trihedral corner reflectors and by estimation calibration model residuals on the reflector array. 
\section{Methods}\label{sec:methods}
\subsection{KAPRI: FMCW radar data processing}\label{sec:proc_SLC}
A basic requirement to generate calibrated polarimetric data is the availability of properly processed SLC images for all elements of the polarimetric scattering matrix. It is therefore necessary to understand the data acquisition process and correct several effects specific to KAPRI.  
For this purpose a signal model for type II\cite{Caduff2015} radar data using FMCW signaling\cite{Stove1992} is introduced.\\
\begin{figure}[h]
	\centering
	\includegraphics[scale=1]{real_aperture_signal_model_geometry}
	\caption{Geometry used to derive the FMCW signal model. $R$ is the slant range from the radar to the point scatterer, $\theta_{3dB}$ is the antenna half power beamwidth, which is represented by the gray triangle. The size of the antenna aperture is $L_{ant}$, the corresponding azimuth resolution (in distance units) is $\delta_{az}$. The inset figure is used to derive the azimuth phase variation. $L_{ph}$ is the phase center displacement, $L_{arm}$ is the antenna rotation lever arm, $R_{0}$ the range of closest approach and $\alpha$ the additional angle to obtain closest approach when the phase center is not in the midpoint of the array.}
	\label{fig:real_aperture_signal_model_geometry}
\end{figure}
Consider a coordinate system having its origin at the location of a radar, as depicted in \autoref{fig:real_aperture_signal_model_geometry}. In this system, the antenna is mounted on a lever arm of length $L_{arm}$; its mainlobe is parallel to the $x$ axis when the pointing angle $\theta$ is 0. The radar images a scene with a complex reflectivity distribution $\rho\left(x,y\right)$ by measuring range profiles $\hat{\rho}\left(R, \theta\right)$ for a number of antenna azimuths angles $\theta = \operatorname{arctan}\left(\frac{y}{x}\right)$ by rotating the antenna assembly with angular speed $\omega$. Each profile is measured by transmitting a linearly modulated signal of duration $t_{chirp}$ with bandwidth $B$ and center frequency $f_c$:
\begin{equation}
	s_t\left(t\right) = e^{j 2 \pi \left( t f_{c} +  \frac{B}{t_{chirp}} t^2 \right)}.
\end{equation}
The total time that the signal needs to travel to a scatterer at range $R$ and back is $t_{c} = \frac{2 R}{c}$. The backscattered signal $s_r$ is a copy of the one being transmitted; delayed by $t_{c}$ and scaled by the complex reflectivity of the scatterer $\rho$.
\begin{equation}
		s_r\left(t\right) = \rho e^{j 2 \pi \left( \left(t - \frac{2 R}{c}\right) f_{c} +  \frac{B}{t_{chirp}} \left(t - \frac{2 R}{c}\right)^2 \right)}.
\end{equation}
In the radar, $s_r$ is mixed with the transmitted chirp $s_t$  to remove the linear modulation resulting in a deramped signal $s_d$ which is sampled and stored: 
\begin{equation}\label{eq:deramp}
	\begin{aligned}
	s_{d}\left(t\right) &=s_t\left(t\right)s_r\left(t\right)^* =\\ 
	&\sigma e^{j 4 \pi \frac{ R}{c}f_c}  e^{j 4 \pi \frac{2 R B }{c t_{chirp}} t}  e^{j 4 \pi \frac{2 R^2 B}{c^2}}.
	\end{aligned}
\end{equation} 
Its frequency $f_{b} = \frac{4 R B}{c t_{chirp}}$ is proportional to the slant range $R$;
two phase terms can be identified: $ e^{j 4 \pi \frac{R}{c}f_c}$ is the two way propagation phase, the quantity of interest for  interferometric measurements. The second phase component is the residual video phase (RVP). This component needs to be compensated for SAR processing, where its variation during the aperture time may cause defocussing.\\
From \autoref{eq:deramp}, it follows by the linearity of the Fourier transform, that the range profile $\hat{\sigma}\left(R, \theta\right)$ of a collection of targets with complex reflectivities $\rho_i$ located at ranges $R_{i}$ is recovered by the transform of $s_{d}\left(t\right)$.\\
As in pulsed radars, the range resolution using FMCW signals is inversely proportional to the bandwidth $\delta_{r} = \frac{c}{2 B}$. With a bandwidth of 200 MHz, KAPRI can achieve a range resolution of 0.75 m\cite{Strozzi2011}; the effective range resolution is lower because the dechirped data is windowed before the Fourier transform to mitigate range sidelobes.\\ It is interesting to note that for FMCW system the sampling rate is not governed by the transmitted bandwidth; instead it is dictated by the extent of ranges to be imaged\cite{Meta2006}.\\
To obtain two dimensional images, range profiles are acquired while the antenna is rotated with angular velocity $\omega$. 
Thus, the deramp signal for a point target at $R, \theta_{t}$ in the slow-time versus fast time plane is:
\begin{equation}\label{eq:signal_model}
	\begin{aligned}
	& s_{d}\left(t,\tau\right) = \rho e^{j 4 \pi \frac{ R}{c}f_c}   \\
	& e^{j 4 \pi \frac{2 R B }{c t_{chirp}} t}  e^{j 4 \pi \frac{2 R^2 B}{c^2}} P\left(\tau \omega - \theta\right),
	\end{aligned}
\end{equation} 
where $t$ is the fast time, $\tau = n t_{chirp}$ is the slow time variable and $P\left(\tau \omega\right)$ describes the two way antenna pattern. Its beamwidth is approximated by:
\begin{equation}\label{eq:azimuth_resolution}
	\theta_{3dB} = \frac{\lambda}{L_{ant}}
\end{equation}
where $L_{ant}$ is the size of the antenna aperture and $\lambda$ is the wavelength employed.
If the rotation speed is chosen such that $\omega \tau \ll \theta_{3dB}$, the acquisition is oversampled in azimuth and several range profiles represent the same target. Assuming they are all subject to the same white noise process, their average -also called azimuth presum or decimation- produces measurements with an increased SNR and an azimuth resolution limited by the real beam size.\\ 
Due to diffraction, the radiation beam emitted by the antenna broadens linearly with increasing distance consequently, the spatial resolution in cross-range $\delta_{az}$ increases with distance:
\begin{equation}\label{eq:azimuth_ground_resolution}
	\delta_{az} = \frac{R \lambda}{L_{ant}}.
\end{equation}
KAPRI employs a traveling wave slotted waveguide antenna\cite{Hines1953a,Granet2007}; it is constructed by cutting slots resonating at the design frequency in a section of rectangular waveguide. When they are appropriately spaced, the fields emitted at each cut combine in phase, producing a narrow beam in the desired direction. Two types of slotted waveguide antenna exist\cite{Enjiu2013}: the resonant and the traveling wave design. The second type has been chosen because it can be operated with a larger bandwidth; achieving a finer range resolution. The main drawback of this antenna design is its frequency-dependent beam squint: when it is operated at frequency differing from the design value, the phase differences at the slots change; altering the direction of the mainlobe. This effect has been uses for several fast imaging radar, where a mechanical antenna rotation would not be possible\cite{Yang2014,Yang2012,Mayer2003,Alvarez2013}. However, in the case of KAPRI the squint is undesired: with a large amount of squint w.r.t beamwidth, the beam only illuminates a scatterer for a fraction of $t_{chirp}$, decreasing the effective transmitted bandwidth and hence the range resolution. \\ 
If the antenna is rotated slowly enough,  the spacing of each range profile is less than the beamwidth and the effect of beam squint is visible in the slow-fast time domain as skewing of point target responses, as illustrated in~\autoref{fig:squint_correction}: during the chirp, the target is only at the center of the beamwidth when the antenna rotation angle matches the beam squint angle. 
The azimuth-frequency skewing observed in oversampled data is key to mitigate the effect of the beam squint. For each chirp frequency $f$, the data at the corresponding fast time $s_{d}\left(t,\tau\right)$ is shifted back in azimuth by $\tau_{sq}=\frac{\theta_{sq}}{\omega}$, where the squint angle is:
\begin{equation}\label{eq:squint_exact}
	\theta_{sq} = \sin^{-1}\left(\frac{\lambda}{\lambda_{g_{ij}}} + \frac{k \lambda}{s}\right).
\end{equation}
here ${\lambda_g}_{ij}$ is the wavelength for the $ij$-TE mode of the waveguide, $\lambda$ is the freespace wavelength, $s$ is the element spacing and $k$ is the mode number. In this case, the waveguide mode used is TE01 and $k=0$ is assumed because all the slots are to be transmitting in phase\cite{kraus88} to direct the main beam at the antenna broadside.\\
In processing KAPRI data a linear approximation for the squint relative to the pointing at the design frequency $f_c$ is used instead:
\begin{equation}\label{eq:squint_linearised}
	\theta_{sq} - \theta_{sq}^{f_{c}}  =  \alpha f.
\end{equation}
this choice is necessary because manufacturer-provided antenna pattern measurements at different frequencies suggest that the vertically and the horizontally polarized units have different squint characteristics.\\ No other design information being available, a data-driven method had to be used for the correction of the beam squint. To estimate the linear squint rate $\alpha$, the response of a strong point-like target is extracted from the data by windowing it in range compressed data and converting it back in the fast time domain; its envelope is then extracted with a discrete Hilbert transform. Finally a fit using the model of \autoref{eq:squint_linearised} is performed to obtain an estimate of $\alpha$.\\
To correct the frequency-dependent beam squint the azimuth sample spacing must be smaller than the antenna beamwidth: this permits to reconstruct the full bandwidth illumination of each scatterer with by combining chirp samples acquired at subsequent azimuth positions.
\begin{figure}[ht]
	\centering
	\includegraphics[scale=0.2]{squint_correction}
	\caption{Illustration of the frequency-dependent antenna squint. When the antenna rotates during the electronic scan, the energy of the target is spread through several azimuth bins.}
	\label{fig:squint_correction}
\end{figure}
After correcting the beam squint, azimuth presumming can be applied if desired to reduce the data size and increase SNR. Then, an Hann window is applied to the first and last $z$ samples of the raw data $s_{d}$ to mitigate the transient signal caused by the abrupt change in frequency due to the repetition of the chirp and the end of each pulse. A second Kaiser Window is then applied to reduce range processing sidelobes that are caused by the finite bandwidth. Finally, a fast time Fourier transform implements the range compression to obtain the SLC image $\hat{\rho}\left(R, \theta\right)$.\\ Each range line of the SLC thus compressed is then multiplied by $\sqrt{R^3}$ to compensate for the power spreading loss. In this manner, the intensity of the SLC data is directly proportional to the radar brightness $\beta_{0}$\cite{Raney1994}. 
\subsection{Antenna Pattern Misalingment}\label{sec:misalingment}
In addition to the difference in frequency-dependent squint rate, the horizontally and the vertically polarized antennas appear have an azimuth mispointing of approx. $0.2^\circ$. It was first observed by analyzing the response of strong point-like target, where a significant misregistration between the $HH$ and the $VV$ images was observed. The misalignment is particularly problematic for cross-polar measurements: the transmitting and receiving patterns are not aligned. Using the available pattern information a power loss of approx 2.5 dB  compared to the ideal case is predicted. This loss reduces the SNR for the cross-polar channels, leading to noisier measurements.\\ While the offset between copolar channels can be corrected by coregistration without major consequences, there is no method capable to compensate the SNR loss in the crosspolar measurements a posteriori. An adjustable antenna mount was manufactured by replacing one of two hinges where the antennas are fixed on the towers (see \autoref{fig:adjustable_bracket}) with an adjustable bracket that allows to slide the antenna back and forth on the one side, obtaining the effect of rotating it around the center. Basing on the size of the antenna mounting bracket and on the amount of misaligment, it was determined that the horizontally polarized antennas need to be shifted by 1.8 mm to compensate form the pattern mispointing.
\begin{figure}[ht]
	\centering
	\includegraphics[scale=0.4]{antenna_offset_2}
	\caption{Illustration of the adjustable antenna mount allowing to shift the patterns to bring the H and the V antennas into alignment. The left bracket can slide back an forth, allowing the antenna to pivot on the right hinge. For small shifts, this movement approximates a rotation around the center of the antenna tower.}
	\label{fig:adjustable_bracket}
\end{figure}
\subsection{Polarimetric Calibration}\label{sec:proc_polcal}
In the case of GPRI the processing is concluded after range compression and the data is ready for interferometric applications.  For KAPRI data, additional steps are necessary to obtain polarimetric measurements that are well calibrated.\\ The first of these steps requires a brief review of the polarimetric antenna configuration, as it is depicted in \autoref{fig:antenna_arrangement}:\\ Six antennas are mounted on a supporting structure connected to the rotary scanner. Of these, 2 are transmitting antennas, one for each polarization. The remaining 4 are connected in pairs through switches to the receivers; each pair composed of an horizontally and a vertically polarized unit. This configuration permits to acquire full polarimetric dataset by selecting the desired antennas for the transmitter and for each receiver separately. This arrangement ensures a low level of polarimetric crosstalk because only one combination is acquired at each time and the antennas are physically separated. Additionally, the separation of transmitting and receiving antennas increases the transmit-receive isolation, a fundamental requirement for FMCW performance\cite{Beasley1990,Stove1992, Strozzi2011}.  Unfortunately this configuration complicates the phase calibration of polarimetric data by introducing additional phase terms unrelated to scattering properties.
\begin{figure}[ht]
	\centering
	\includegraphics[scale=0.15]{kapri_antenna_arrangement}
	\caption{KAPRI with the usual antenna arrangement overlaid. The blue and red dots represent the equivalent phase centers for the upper $HHV$ and $VV$ channels.}
	\label{fig:antenna_arrangement}
\end{figure}
Consider an image $i$ acquired by transmitting at antenna located at $\mathbf{x_t^i}$ and receiving at $\mathbf{x_r^i}$. These antennas can be replaced by an equivalent antenna located at $\mathbf{x_{eq}^i}$, the midpoint between transmitter and receiver \cite{Pipia2009}. Thus, for polarimetric observations using  KAPRI the equivalent antenna phase center $\mathbf{x_{eq}^i}$ location will change depending on the chosen polarization. Therefore, there exist certain combinations of channels $i$ and $j$ where the equivalent phase centers will be separated by a baseline $\mathbf{b_{ij}^{eq}}$. Consequently, the polarimetric phase difference $\phi_{ij} = \phi_{ij}^{pol} + \phi_{ij}^{prop}$  will contain an interferometric contribution $\phi_{ij}^{prop}$. This term appears as topographic fringes when visualizing the polarimetric phase difference and will complicate calibration by adding an additional phase contribution unrelated to the polarimetric properties of the scatterers.\\
To obtain valid polarimetric phase differences, it is necessary to estimate and subtract the interferometric contribution. This can be done  by considering two additional channel $k$ and $l$  acquired with  a non-zero baseline $\mathbf{b_{ij}^{eq}}$ and with the same polarization, where $\phi_{kl}^{pol} \approx 0$. Generally, for any two channels $m$ and $n$,  the propagation phase difference can be approximated as a function of the local incidence angle and of the perpendicular baseline separating the phase centers:
\begin{equation}\label{eq:prop_approximation}
		\phi_{mn}^{prop} = \frac{4\pi}{\lambda} b_{mn}^{eq} \sin(\theta - \alpha_{bl}),
\end{equation}
where $\alpha_{bl}$ is the baseline angle w.r.t to the vertical and the look angle $\theta_l$ is the angle between the line of sight vector $\mathbf{p}$ and the vertical axis and $b_{ij}^{eq}$ is the perpendicular baseline between the equivalent phase centers.
Thus from \autoref{eq:prop_approximation}, $\phi_{ij}^{prop}$ can be estimated from $\phi_{kl}$ if the look angle does not significantly change from $kl$ to $ij$, i.e if $\theta_{ij} - \alpha_{ij} \approx \theta_{kl} - \alpha_{kl}$. 
\begin{equation}
	\hat{\phi}_{ij}^{prop} = \frac{b_{eq}^{ij}}{b_{eq}^{kl}} \phi_{kl}.
\end{equation}
This formula can only be used if $\frac{b_{eq}^{ij}}{b_{eq}^{kl}}$ is integer\cite{Massonnet1996}, if this condition is not met, phase unwrapping of $\phi_{kl}^{prop}$ is necessary before rescaling.\\
In order to correct for all the possible combinations that have a non-zero baseline, the measured scattering matrix $\mathbf{S}$ is converted into a polarimetric covariance matrix;  $\hat{\phi}_{ij}^{prop}$ it then subtracted from the phase of every non-diagonal element $ij$. The result is a flattened covariance matrix where the sole phase contribution is the polarimetric phase difference.\\
This matrix is the starting point for the polarimetric calibration proper;
the procedure is based on the linear distortion matrix model\cite{Saraband1990, Sarabandi1992a} that relates the observed scattering matrix $\mathbf{S_{meas}}$ with the correct matrix $\mathbf{S}$:
\begin{equation}\label{eq:distorsion_scattering}
	\mathbf{S_{meas}} = \mathbf{R} \mathbf{S} \mathbf{T}.
\end{equation}
or in covariance form
\begin{equation}\label{eq:covariance_distortion}
	\mathbf{C}_{meas} = \mathbf{D} \mathbf{C} \mathbf{D}^{H}.
\end{equation}
where $\mathbf{D}$ is the Kronecker product of $\mathbf{R}$ and $\mathbf{T}$, the matrices that describe the phase and amplitude imbalances and crosstalk in reception and transmission.
In the case of KAPRI, crosstalk calibration is not performed as the radar is expected to have a good polarization isolation, largely due to the fact that only one polarization is acquired at a time by selecting the appropriate combination of transmitting and receiving antennas. The only source of crosstalk is the presence of cross-polarized lobes in the direction of the antenna mainlobe. The manufacturer has provided simulated radiation patterns for the horizontally polarized antennas, where the isolation between the co and the cross polarized pattern in the main-lobe direction is observed to be better than 60 dB. By computing the $VV-HV$ ratio of the oversampled response of a trihedral corner reflector, the polarization purity of the system was estimated to be better than 35 dB.\\
Thus, neglecting crosstalk the matrices can be written as:
\begin{equation}
	\begin{aligned}
	&\mathbf{R} = A \begin{bmatrix}
		1 & 0\\
		0 & f/g e^{i\phi_{r}}
	\end{bmatrix},\\
	&\mathbf{T} = A \begin{bmatrix}
			1 & 0\\
			0 & f g e^{i\phi_{t}}
		\end{bmatrix}
	\end{aligned}
\end{equation}
where $f$ is the one-way copolar amplitude imbalance with respect to the $H$ polarization, and $g$ the amplitude imbalance of the crosspolarized channels. $\phi_t = \phi_{t,h} -\phi_{t,v}$ is the phase offset between the polarizations when transmitting and $\phi_{r} = \phi_{r,h} -\phi_{r,v}$ is the phase offset in reception and $A$ is the absolute amplitude calibration parameter (RCS)\cite{Ainsworth2006a, Fore2015}.\\
The four unknown complex parameters in $\mathbf{D}$ can be determined using a trihedral corner reflector and a reciprocal scatterer with a significant cross polarized contribution\cite{Sarabandi1989,Pipia2009}.\\
With the above model, an ideal trihedral reflector with the scattering matrix
\begin{equation}
 \mathbf{S} = \sqrt{\sigma_{tri}}
 \begin{bmatrix}1 & 0\\ 0 & 1\end{bmatrix}
\end{equation}
where $\sigma_{tri}$ is its RCS, has a measured covariance matrix $\mathbf{C^{\prime}}$:
\begin{equation}
	\begin{aligned}
	&\mathbf{C^{\prime}} =\\
	&= k \sigma_{tri}\\
	&\begin{bmatrix}
		1 & 0 & 0 & f^2 e^{-i \left(\phi_t + \phi_r\right)}\\
		0 & 0 & 0 & 0\\
		0 & 0 & 0 & 0\\
		f^2 e^{i \left(\phi_t + \phi_r\right)} & 0 & 0 & f^4
	\end{bmatrix}
	\end{aligned}.
\end{equation}
The copolar amplitude imbalance $f$ is estimated of the HHHH and VVVV elements of $\mathbf{C^{\prime}}$:
\begin{equation}
	f = \left(\frac{C^{\prime}_{VVVV}}{C^{\prime}_{HHHH}}\right)^{\frac{1}{4}}.
\end{equation}
Similarly, the copolar imbalance phase $\phi_r + \phi_t$ is determined from the phase of $C_{VVHH}^{\prime}$:
\begin{equation}
	\phi_r + \phi_t = \operatorname{arg}\left(C_{meas}^{VVHH}\right).
\end{equation}
Both parameters are estimated on the oversampled response of a corner reflector.
Because of the difficulty of placing and correctly orienting a dihedral reflector, the estimation of $g$ and $\phi_t - \phi_r$ is based on the assumption that most pixels in the calibration dataset represent reciprocal scatterers:
\begin{equation}
	g = \left<\frac{C_{rec}^{HVHV}}{C_{rec}^{VHVH}}\right>^\frac{1}{4},
\end{equation}
and:
\begin{equation}
	\phi_t - \phi_r =\operatorname{arg}\left( \left<C_{meas}^{HVVH}\right>\right).
\end{equation}
When $\mathbf{D}$ is estimated, \autoref{eq:covariance_distortion} is inverted to obtain a calibrated covariance matrix.\\
If radiometric calibration is desired, the value of $A$ can be determined after imbalance correction:
\begin{equation}
	A =	\left(\frac{\sigma_{tri}}{C^{\prime}_{HHHH} R^{3}}\right).
\end{equation}
where $R$ is the slant range to the trihedral corner reflector. This is necessary to compensate for the range spread loss compensation as performed in \autoref{sec:methods}.
\subsection{Experimental Data}\label{sec:data}
A calibration dataset was acquired in September 2016 at an urban-agricultural area near M\"{u}nsingen, Switzerland to test the methods described above. The data was acquired from the top of an hill approximately 800 m high, looking down towards fields and the town. 6 Trihedral Corner Reflectors were placed in the scene for the determination of calibration parameters and to assess imaging quality. Three of these reflectors have triangular faces with a length of 40 cm, corresponding to a RCS of $\sigma=\frac{4}{3}\pi \frac{a^4}{\lambda^2}=25.5 dB$, while the remaining two are cubic corner reflector with $\sigma= 35 dB$, at the nominal central frequency of 17.2 GHz.
	\begin{figure*}
		\centering
		\includegraphics{figure_7}
		\caption{Pauli RGB composite of the imaged scene, resampled to cartesian coordinates with 2 m pixel spacing, The radar is scanning from left to right. The location of corner reflectors is marked by blue circles. The bright structures on the bottoms are buildings in the town of M\"{u}nsingen.}
		\label{fig:pauli_rgb}
	\end{figure*}
The dataset was acquired with the horizontally polarized antenna group shifted towards the V group by 1.8 mm to compensate for the pattern misalignment as described in \autoref{sec:misalingment}.\\
\autoref{fig:pauli_rgb} shows the calibrated Pauli RGB composite of the scene, resampled in Cartesian coordinates with the location of the reflectors marked using blue circles; their location and RCS is summarized in \autoref{tab:reflectors}.\\
A separate dataset containing a dihedral reflector was acquired at our H\"{o}nggerberg campus in order to investigate the effect of antenna pattern misalignment on crosspolar acquisitions and to test the suitability of the computed adjustment value as discussed in \autoref{sec:misalingment}. This was done separately because of the logistical complications associated with the transportation and the setup of large calibration targets.


\begin{table}[ht]
	\centering
	\pgfplotstabletypesetfile[
								every head row/.style={before row=\toprule,after row=\midrule},
								every last row/.style={after row=\bottomrule},
								col sep=comma,
								columns={rsl, RCS, type},
								columns/type/.style={column name={Type} ,string type},
								columns/RCS/.style={precision=3, column name=$\sigma_0$},
								columns/rsl/.style={precision=3, column name=$R_0$}
										] {../tab/table_1.csv}
	\caption{Summary of the employed TCRs. Distance from the radar and expected RCS.}
	\label{tab:reflectors}
\end{table}




\section{Results}\label{sec:results}
\subsection{Beam Squint Correction}\label{sec:results:squint_correction}
\begin{figure*}[Ht!]
	\centering
	\includegraphics{figure_12}
	\caption{Azimuth-frequency response of the "Hindere Chlapf" TCR: the raw data samples around the reflectors azimuth location were extracted, then filtered in range by Fourier transforming them along the frequency axis, appliyng an Hamming window about the range location and converting them  back into the time domain with an inverse Fourier transform. By doing so, only the portion of the range spectrum close to the reflectors location was kept. Finally, the complex envelope of the data was extracted using a discrete Hilbert transform. This is conceptually equivalent to the plot of \autoref{fig:squint_correction}. Panel (a) shows the result for the HH channel, (b) for the VV channel, (c) for the HH channel after the interpolation described in \autoref{sec:methods:squint_correction}  and (d) the same for the VV channel.}
	\label{fig:raw_squint}
\end{figure*}
In \autoref{fig:raw_squint}, the frequency-azimuth response relative to the pointing at the center frequency is shown for the "Hindere Chlapf" TCR. This plot was generated with the procedure described in \autoref{sec:methods:squint_correction} by filtering the range compressed around the location of the reflector and converting it back to the time domain by an inverse Fourier transform. In (a) and (c), the procedure was applied to the HH channel data before and after the squint correction interpolation; (b) and (d) show the same for the VV channel. In each plot, a line shows the results of the model fit described in \autoref{eq:squint_linearised}, the $a$ parameter is overlaid additionally.\\
The same fit procedure has been applied to all reflectors for both the HH and the VV data; the results are shown in \autoref{tab:a_squint_fit}.\\
\begin{table*}[ht]
	\centering
	\pgfplotstabletypesetfile[			col sep=comma,
										columns={name, a_HH, a_VV, a_res_HH, a_res_VV},
										every head row/.style={before row=\toprule,after row=\midrule},
										every last row/.style={after row=\bottomrule},
										columns/name/.style={column name={Name} ,string type},
										columns/a_HH/.style={precision=1, column name={$a_{HH}$}},
										columns/a_VV/.style={precision=1, column name={$a_{VV}$}},
										columns/a_res_HH/.style={precision=1, column name={residual $a_{HH}$}},
										columns/a_res_VV/.style={precision=1, column name={residual $a_{VV}$}},
										%columns/res/.style={precision=3, column name={residual}},
										]{../tab/table_squint.csv}
	\caption{Result of fitting the model of \autoref{eq:squint_linearised} to each reflector in the calibration array. In the second column, the $a$ parameter for the HH channel is shown, in the third the one for the VV channel. The last two columns show the same parameters re-estimated after applying the squint correction using 4.2 and 3.9 $\frac{\circ}{GHz}$ for $a_{HH}$ and $a_{VV}$ respectively.}
	\label{tab:a_squint_fit}
\end{table*}
In \autoref{fig:tcr_mph}, panels (a),(b), (d) and (e) the effect of the squint and the result of the correction are visible on range compressed data for the "Hindere Chlapf" reflector. The plots are generated by oversampling the response of the range compressed TCR data in azimuth and range using a cubic order spline approximating a sinc interpolator. In each plot azimuth and range resolutions are estimated numerically by fitting a spline on the response at the corresponding maximum and computing its 3 dB width. Panels (a) and (d) show the range compressed data for the HH and VV channels. In (b) and (e) the same is repeated after applying squint correction before range compression.
\begin{figure*}[ht]
	\centering
	\includegraphics{figure_1}
	\caption{Oversampled phase and amplitude responses for the corner reflector "Hindere Chlapf" at 673 m slant range. (a) HH channel without correction, (b) HH channel with frequency-dependent squint compensation (c) same as (b) with azimuth phase ramp removal. (d) VV channel without correction, (e) VV channel with frequency-dependent squint compensation (f) same as (e) with azimuth phase ramp removal.
	The phase of each response is referenced to its maximum.}
	\label{fig:tcr_mph}
\end{figure*}

%\begin{figure*}[ht]
%	\centering
%	\includegraphics{figure_11}
%	\caption{SLC data of a section of the "Chutzen" calibration dataset, represented in radar coordinates. The brightness corresponds to the intensity, the hue is modulated by the phase, according to the colormap shown above. Different stages of processing are depicted: (a) VV with neither squint nor azimuth correction (b) VV channel with frequency-dependent squint compensation (c) same as (b) with additional azimuth phase ramp removal. A slight azimuth resolution deterioration is visible between (b) and (c); especially on sharp transitions between areas with high backscatter and shadow. However at this scale it is rather difficult to appreciate the subtle effects of the proposed corrections.}
%	\label{fig:scene_mph}
%\end{figure*}


\subsection{Azimuth Processing}\label{sec:results:azimuth_processing}
\begin{figure*}[ht]
	\centering
	\begin{subfigure}[t]{\textwidth}
		\centering
		\includegraphics{figure_2}
		\subcaption{Response without azimuth phase correction.}
		\label{fig:phase_response_VV:uncorrected}
	\end{subfigure}\\
	\begin{subfigure}[t]{\textwidth}
		\centering
		\includegraphics{figure_3}
		\subcaption{Response after azimuth phase correction.}
		\label{fig:phase_response_VV:corrected}
	\end{subfigure}
	\caption{Relative phase/amplitude response for all reflectors in the calibration array, (a) no azimuth phase correction (b) after applying azimuth phase correction. Continuous lines: VV channel, dashed lines: HH channel. To display the relative phase variation, the phase at the maximum is subtracted from each plot. The vertical lines indicate the theoretical 3 dB resolution of the antenna $\theta_{3dB}$.}
	\label{fig:phase_response_VV}
\end{figure*}
The ability of the phase model described in \autoref{sec:methods:azimuth_processing} to explain the observed phase variation  on the $VV$ channel response is tested on each reflector in the array: the maximum in range and azimuth was identified and the samples corresponding to the half power beamwidth were extracted at the range of maximum intensity. The unwrapped phase was then used to estimate $L_{ph}$ according to described in \autoref{eq:rph_estimation}.

\begin{table}[ht]
	\centering
	\pgfplotstabletypesetfile[			col sep=comma,
										columns={name, rsl, LphHH, LphVV},
										every head row/.style={before row=\toprule,after row=\midrule},
										every last row/.style={after row=\bottomrule},
										columns/name/.style={column name={Name} ,string type},
										columns/rsl/.style={precision=1, column name={$R_0 [m]$}},
										columns/LphHH/.style={precision=2, fixed, column name={$L_{ph}^{H} [m]$}},
										columns/LphVV/.style={precision=2, fixed, column name={$L_{ph}^{V} [m]$}},
										%columns/res/.style={precision=3, column name={residual}},
										]{../tab/table_2.csv}
	\caption{Result of the phase center displacement fit for six trihedral corner reflectors located at different ranges. In the first column, the estimated phase center displacements for the H antenna are shown, in the second the ones for the V unit.}
	\label{tab:rph_fit}
\end{table}
The resulting model parameters fit values for the H and V antennas are shown in \autoref{tab:rph_fit} alongside with the distance from the radar and the name of the reflector, defined in \autoref{tab:reflectors}. Owing to the lack of sufficiently bright crosspolarizing point targets, the equivalent horizontal phase center locations for HV and VH channels were not estimated from the data and are therefore not displayed in the table. In the following, their location it assumed to be at the midpoint between $L_{ph}^{H}$ and $L_{ph}^{V}$, which corresponds to the theoretical equivalent phase center for these channels.\\
The result of applying the correction of \autoref{eq:correction} to the TCR "Hindere Chlapf" at 673 m is displayed in panels (c) and (f) of \autoref{fig:tcr_mph}, this plot is produced by oversampling the SLC data around the location of the reflector using a cubic spline interpolator.
In \autoref{fig:phase_response_VV:corrected} the phase and amplitude response in both the HH and VV channel is plotted for all reflectors. To produce the plot, all responses were aligned in azimuth, normalized to the maximum and the phase referenced to the phase at closest approach. In this way, an easier comparison of the azimuthal phase variation is facilitated.



\subsection{Antenna Pattern Misalignment}\label{sec:results:misalignment}
\begin{figure}[ht!]
	\centering
	\includegraphics{figure_4}
	\caption{Oversampled azimuth power response of a dihedral corner reflector, before (green) and after the correction of antenna pattern mispointing (orange). The observed gain fits the power loss expected due to the H and V patterns not perfectly overlapping. This figure was computed using manufactured-supplied antenna patterns for the horizontally and vertically polarized unit.}
	\label{fig:hv_power}
\end{figure}
To verify the impact of the H-V pattern pointing shift as described in \autoref{sec:methods:misalingment} on the performance of crosspolar measurements, the response of a dihedral reflector with an high crosspolar contribution is analyzed for two configurations:\\ 
\begin{enumerate}
	\item the antennas are not mechanically moved; the H and V pattern are not aligned in azimuth.
	\item the optimal shift of 1.8 mm, as described in \autoref{sec:methods:misalingment} is applied to the movable antenna hinge to bring the patterns into alignment.\\ 
\end{enumerate}
In \autoref{fig:hv_power}, the result of the above experiment is shown as the oversampled, coregistered azimuth response in the HV channel.
\subsection{Removal of Topographic Phase}\label{sec:results:topo_removal}
\begin{figure*}[ht!]
	\centering
	\begin{subfigure}[t]{\textwidth}
	\centering
	\includegraphics{figure_9}
	\subcaption{HHVV phase difference with topographic phase.}
	\label{fig:topo_phase:uncorrected}
	\end{subfigure}\\
	\begin{subfigure}[t]{\textwidth}
	\centering
	\includegraphics{figure_10}
	\subcaption{HHVV phase difference with topographic phase removed by the proposed method.}
	\label{fig:topo_phase:corrected}
	\end{subfigure}\\
	\caption{HH-VV phase difference in radar coordinates, (a) before and (b)  after the removal of the topographic phase term as described in \autoref{sec:methods:topo_removal}. The hue of the image is modulated by the covariance phase, the intensity by the magnitude, the saturation by the copolar coherence magnitude, as shown in the bottom colorbar and plot. The locations and names of reflectors described in \autoref{tab:reflectors} are plotted. The interferometric fringe pattern visible in (a) is removed by the proposed correction, as plotted in (b), leaving a phase offset that will be removed by the polarimetric calibration.}
	\label{fig:topo_phase}
\end{figure*}
\begin{figure}[Ht!]	
	\centering
	\begin{subfigure}[t]{\columnwidth}
	\centering
	\includegraphics{figure_9_b}
	\subcaption{Joint histogram of HHVV phase and look vector elevation angle, before the topographic phase removal.}
	\label{fig:topo_correlation:uncorrected}
	\end{subfigure}\\
	\begin{subfigure}[t]{\columnwidth}
	\centering
	\includegraphics{figure_10_b}
	\subcaption{Joint histogram of HHVV phase and look vector elevation angle, after the topographic phase removal.}
	\label{fig:topo_correlation:corrected}
	\end{subfigure}
	\label{fig:topo_correlation}
	\caption{Plot of the joint histogram of the HH-VV phase difference versus the look vector elevation angle for  80-th percentile brightest pixels with a copolar coherence higher than 0.6. In (a) the histogram before the topographic phase removal is plotted, in (b) the same analysis is repeated after the correction. In the latter plot, the cross-covariance of the data is computed and used to determine an error ellipsis. The ellipsis angle quantifies the level of correlation between the copolar phase difference and the topography in the scene; it is expected to be close to zero for a successful removal of the topographic contribution, because the copolar phase is expected to be independent from topography.}

\end{figure}
The removal of the topographic phase contribution in the copolar phase difference, as described in \autoref{sec:methods:topo_removal}, is visualized in  \autoref{fig:topo_phase} by showing an excerpt from the data in radar coordinates; where the hue of each pixel is set to the HH-VV phase difference, their intensity to the corresponding $\mathbf{C}$ matrix element magnitude and by setting the saturation of the image with a nonlinear threshold function of the copolar coherence magnitude, estimated using a $\mathrm{5 \times 5}$ sample window. A second independent verification of the phase trend removal is carried out by plotting the joint histogram of the HH-VV phase versus the look vector elevation angle for the 80-th percentile brightest pixels with a copolar coherence higher than 0.6. The look vector elevation angle is determined by the geocoding procedure using an external DEM. The idea behind this plot is to visualize the correlation between the topography of the scene and the HH-VV phase difference: no correlation is expected after successfully removing the topographic phase contribution. As a quantitative measure of correlation, the cross-correlation matrix of copolar phase versus elevation angle is computed; from its eigendecomposition the orientation angle of the first principal component is determined. This angle is expected to be zero for a perfect correction of the topographic phase contribution: the distribution of the copolar phase should be independent from the distribution of the elevation angles.

\subsection{Polarimetric Calibration}\label{sec:results:proc_polcal}

								
\begin{table*}[h]
	\centering
	\pgfplotstabletypesetfile[
		col sep=comma,
		columns={name, slant range, HH-VV amplitude imbalance, HH-VV phase imbalance, Polarization purity},
		every head row/.style={before row=\toprule,after row=\midrule},
		every last row/.style={after row=\bottomrule},
		columns/name/.style={column name={Name} ,string type},
		columns/slant range/.style={precision=1, column name=\makecell{$R_{0} [m]$\\ Range distance}},
		columns/HH-VV amplitude imbalance/.style={precision=3, column name={$f$}},
		columns/HH-VV phase imbalance/.style={precision=3, column name={$\phi_r + \phi_t [^\circ]$}},
		columns/Polarization purity/.style={precision=3, column name={Purity $[dB]$}}
	]{../tab/table_3.csv}
	\caption{Copolar phase ($\phi_r + \phi_t$) and amplitude imbalance ($f$) computed on the reflectors using the calibrated dataset. The polarization purity (VV/HV ratio) is shown additionally. Results for the reflector used to determine calibration parameters are not shown.}
	\label{tab:polcal}
\end{table*}		
The methods described in the preceding sections were applied to prepare SLC images for each channel. For the final polarimetric calibration the procedure of \autoref{sec:methods:proc_polcal} was used; one reflector in the scene was used as a calibration target, with the four remaining reflectors used for the determination of the calibration performance.\\
An initial assessment of the data quality is made  by computing polarization signatures\cite{VanZyl1987} for two of the five corner reflectors that were not used to determine the calibration parameters. They are plotted in \autoref{fig:signatures}.
\begin{figure*}[hb]
	\centering
	\begin{subfigure}[t]{\textwidth}
	\centering
	\includegraphics{figure_5}
	\subcaption{Reflector "Hindere Chlapf" at 673 m slant range.}
	\label{fig:signatures:near}
	\end{subfigure}\\
	\begin{subfigure}[t]{\textwidth}
	\centering
	\includegraphics{figure_6}
	\subcaption{Reflector "T\"{u}rle" at 2690 m slant range.}
	\label{fig:signatures:far}
	\end{subfigure}
	\caption{Polarization signatures for two trihedral corner reflectors at the locations "Hindere Chlapf" and "T\"{u}rle". For both plots, each panel shows: (a) uncalibrated copolar signature, (b) uncalibrated crosspolar signature; (c) calibrated copolar signature,(d): calibrated crosspolar signature. The power of each response is normalized to the corresponding maximum. A distinct change in signature is observed after the calibration; it is mostly due to the removal of the HH-VV phase offset.}
	\label{fig:signatures}
\end{figure*}

\begin{figure}[ht!]
	\centering
	\includegraphics{figure_8}
	\caption{Dependence of the residual copolar phase ($\phi_r + \phi_t$) and amplitude  ($f$) imbalances on the local incidence angle. The mean and RMS imbalances are shown in each plot. The reflector used for the determination of calibration parameters has been excluded from the plot.}
	\label{fig:inc_angle_trend}
\end{figure}
A quantitative evaluation of the calibration is obtained by estimating the residual copolar phase and amplitude imbalances $f$ and $\phi_r + \phi_t$ on the trihedral from calibrated data, excluding the reflector used to determine the parameter. The results are shown in \autoref{tab:polcal}.
In \autoref{fig:inc_angle_trend}, the dependence of the residuals on the local incidence angle is plotted; the angle was estimated using a 2 m posting digital elevation model of the scene that was backwards geocoded in the radar geometry.
\section{Discussion}\label{sec:discussion}
\subsection{Beam Squint Correction}\label{sec:discussion:squint_correction}
%The effect of correcting the frequency-dependent beam squint is not easily visible in the SLC data displayed in \autoref{fig:scene_mph}. Therefore, in the rest of this section the response of single features in the scene will be analyzed.
The effect of the frequency squint on the raw data around TCR "Hindere Chlapf" is visible in \autoref{fig:raw_squint}, panels (a) and (b) for the HH and VV channels. As described in 
\autoref{sec:methods:squint_correction} and sketched in \autoref{fig:squint_correction}, the data matrix appears skewed:  the physical antenna direction and the effective pointing angle of the beam pattern only match for a brief time during each chirp due to the frequency scanning of the antenna. Because of that, if the data is range compressed, only part of the chirp bandwidth illuminates the target at each time, reducing the observed range resolution. This is verified in \autoref{fig:tcr_mph} subfigures (a) and (c), where the oversampled response of TCR "Hindere Chlapf" after range compression is shown.\\
The linear squint factor $a$ estimated on all reflectors of the calibration array is given in \autoref{tab:a_squint_fit};
the average estimated values of 4 and 3.9 $\frac{\circ}{GHz}$  fit well with the figures suggested by the antenna manufacturer: 4.2 and 3.9 $\frac{\circ}{GHz}$ for the H and V antennas respectively.\\
Thanks to the oversampled acquisition it is possible to use the proposed interpolation method to realign the samples in azimuth, compensating the effect of the squint by combining subsequent sub-chirps with different squint angles in a single coherent chirp that covers the entire bandwidth for the whole duration of time when the target is within the antenna beamwidth. This is shown in \autoref{fig:raw_squint} panels (b) and (d). The result of the interpolation is visible in \autoref{fig:raw_squint}, panels (c) and (d): the spectrum is now aligned in azimuth; as a consequence the range resolution is decreased, as visible in \autoref{fig:tcr_mph}. Additionally the phase response seems to become more stable.
Visually, the phase pattern observed in (a) and (c) is also removed entirely; however, a residual azimuthal phase can be observed in the VV channel in (e).
\subsection{Azimuth Processing}\label{sec:discussion:azimuth_processing}
The residual phase ramp observed on the "Hindere Chlapf" reflector in \autoref{sec:discussion:squint_correction} is not unique to that object:\\ A linear variation of 30 degrees over the 3dB antenna beamwidth can be observed for all reflectors in the dataset, as plotted in \autoref{fig:phase_response_VV:uncorrected}.  The model of \autoref{sec:methods:azimuth_processing} was developed to explain this variation in terms of the acquisition geometry.
The estimated phase center location values $\hat{L}_{ph}$ of \autoref{tab:rph_fit} are consistent and display a standard deviation of less than 2 \% of the antenna length, suggesting that the model is able to predict most of the phase variation. The estimated phase center shift for the H unit $\hat{L}_{ph}^{H}$ is less than -5 cm, wile the one for the V antenna, $\hat{L}_{ph}^{V}$, is 12 cm, almost twice as large and with opposite sign. This difference presumably explains the visibility of the azimuthal phase ramp in the VV channel:\\
For a specific target,  $\theta_r$ is limited to the time where it is inside the antenna beamwidth, $\theta_{3dB}$. This small value would not produce any appreciable phase variation. However for increasing values of $L_{ph}$, the entire function of \autoref{eq:range_phase} is rapidly shifted  by a large $\alpha$, simulating the effect of a larger $\theta_r$, as it would be obtained with a much bigger antenna beamwidth.\\
As shown in panel (g) of \autoref{fig:tcr_mph}, the azimuth phase variation is significantly reduced by the proposed correction method, while the azimuth resolution is slightly degraded, while still being less than $0.6^\circ$, the value that would be expected if the samples were to be incoherently integrated. This result hints again that most of the phase ramp has been removed by the proposed method.\\
Similar results are observed for all trihedral reflectors, summarized in \autoref{fig:phase_response_VV:corrected}.  Generally, the HH data shows a much smaller variation, while the VV phase displays an almost linear increase of 30$^\circ$. As expected from the estimated phase center locations of \autoref{tab:rph_fit}, the observed phase slopes for the V and H have opposite signs.\\  After the correction, only the reflector "Chutzen" shows a large residual azimuth phase variation; its closeness to the radar relative to the far field transition distance of the antennas in the order of 500 m could explain the discrepancy: a non-linear  phase variation is already observed before the correction. Excluding this exception, the residual variation is under $5^\circ$ for samples inside the antenna beamwidth, marked by vertical lines in the plot.
\subsection{Antenna Pattern Misalignment}\label{sec:discussion:misalignment}
Correcting for the estimated pattern alignment results in a gain of approx. 2.5 dB  in HV power with respect to the uncompensated reference case, as plotted in \autoref{fig:hv_power}. The estimated gain is very close to the one expected by analyzing the patterns provided by the antenna manufacturer. This results confirms that the 1.8 mm shift setting employed to acquire the calibration data can correctly compensate the crosspolar power loss and the HH-VV misregistration.
\subsection{Removal of Topographic Phase}
The effect of the topographic phase compensation as described in \autoref{sec:methods:topo_removal} is displayed in \autoref{fig:topo_phase};
three topographic fringes are counted in the unflattened interferogram (\autoref{fig:topo_phase:uncorrected}), corresponding to a total phase variation of $9 \pi$ . An estimate of this contribution is obtained from the unwrapped and rescaled interferogram between the upper and the lower HH channels. Because they are separated by a spatial baseline and they employ the same polarization (as seen in \autoref{fig:antenna_arrangement}), this interferogram provides an estimate of the topographic phase without additional polarimetric phase differences. The estimated topographic phase is then subtracted from the covariance matrix; the resulting flattened copolar phase difference is displayed in \autoref{fig:topo_phase:corrected}; no interferometric fringes are observed.\\ The correction quality is also analyzed in \autoref{fig:topo_correlation} by plotting the joint histogram of the HH-VV phase versus the look vector elevation angle computed using an external digital elevation model. In \autoref{fig:topo_correlation:uncorrected} the correlation is visible as a line. They are the effect of topographic phase wrapping in the HH-VV phase difference; the correlation entirely disappears after subtracting the topographic contribution (\autoref{fig:topo_correlation:corrected}). This is checked quantitatively by computing the cross-covariance of copolar phase versus elevation angle and determining the orientation of the resulting covariance ellipse. The 95 percent confidence ellipsis assuming normally distributed data is plotted in~\autoref{fig:topo_correlation:corrected}; the estimated orientation angle of 0 radians, giving an ellipsis parallel to the data axes, suggests that the method correctly removes the topographic phase component from the copolar phase.
\subsection{Polarimetric Calibration}
An illustration of the effect of the calibration procedure of \autoref{sec:methods:proc_polcal} is given in  \autoref{fig:signatures} by the polarization signatures of two reflectors before and after the calibration.
The signature for the uncorrected data does not correspond to the signatures of any basic scattering mechanism. In the calibrated response, both reflectors show a correspondence with the expected polarization signature for trihedral reflectors.\\ The dramatic change in shape of the signatures is mainly to be attributed to the correction of the HH-VV phase, which shows a significant offset after the removal of the topographic contribution. By using a trihedral corner reflector or a similar scatterer, this offset is estimated and removed.\\
This interpretation is supported by the residual imbalance estimates in \autoref{tab:polcal}.
Both amplitude and  phase imbalances appear to be well corrected; the mean residual for the amplitude imbalance $f$ is 1.03 with an root mean square deviation of 0.05. A mean of $1.19^\circ$ is observed for the copolar phase imbalance $\phi_r + \phi_t$ , with a root mean square deviation of $7^\circ$. The biggest outlier for $f$ is represented by reflector "CR1", placed very close to the radar, at 73 m slant range. As previously discussed, it is not located sufficiently far away to be in the far field region of the antennas; this could explain the discrepancy from the rest of the array. In the case of the copolar phase imbalance $\phi_r + \phi_t$, the results appear to be correlated with the polarization purity of each reflector; the largest outlier being the farthest TCR, "CR6". This is the target with the smallest estimated purity; which likely implies larger clutter contributions in the resolution cell, which would increase the phase bias.\\
The assumption of negligible crosstalk seems plausible considering that almost all reflectors exhibit polarization purities above 35 dB, the same reasons discussed above could explain the difference observed for  "T\"{u}rle".
No clear trend for $f$ and $\phi_r + \phi_t$ residuals as a function of incidence angle is observed (\autoref{fig:inc_angle_trend}); nevertheless a clear judgment is difficult given the small number of data points available and the limited extent of incidence angles covered. However, given the small RMS and mean residual errors, the choice of a model that does not consider variation of calibration parameters with incidence angle appears to be justified. The absence of distinct incidence-angle related trends also suggests the validity of the method proposed in \autoref{sec:methods:proc_polcal} for the estimation and removal of the topographic contribution from the polarimetric phase differences.\\

\section{Conclusions}\label{sec:conclusions}
In this paper, two main aspects of the calibration of KAPRI, a new polarimetric portable ground-based FMCW radar were discussed:
\begin{enumerate}
	\item The preprocessing of raw data into SLC images, taking into account several effects due to the specific hardware design of the system.
	\item The polarimetric calibration of data into phase and amplitude calibrated polarimetric covariance matrices.
\end{enumerate}
\subsection{Preprocessing}
 The particular antenna design causes a frequency-depending shift of the antenna mainlobe during the chirp that causes a worsened range resolution. It is corrected using a slow time-fast time interpolation procedure; significant range resolution improvement are observed after the correction.\\ The real aperture, azimuth scanning design results in a motion of the antenna phase center relative to the scatterers, causing an observable azimuth phase ramp in point target responses. The variation is significantly different between the antennas, with almost 30$^\circ$ over the 3 dB beamwidth for the V antenna and much smaller for the H unit. This additional phase will complicate polarimetric calibration if left unaltered; is corrected by a SAR-like azimuth filter that reduces the total phase variation to under 10$^\circ$.\\
Because separated transmitting and receiving antennas are used for each polarization, the polarimetric calibration is complicated by the presence of an interferometric baseline between channels that adds a topographic phase contribution in the polarimetric phase differences. Thanks to an additional interferometric baseline, the topographic contribution can be estimated and subtracted from each element of the covariance matrix affected by it.
\subsection{Calibration}
The resulting flattened covariance matrix is then calibrated  by assuming zero crosstalk and estimating copolar imbalances using a trihedral corner reflector assuming the parameters to be independent from the  incidence angle. The crosspolar imbalance is estimated using distributed targets under the assumption of reciprocity.\\
The calibration quality is assessed by estimating residual calibration parameter on a calibrated scene with five trihedral corner reflectors: the mean amplitude imbalance is close to unity while the mean residual phase imbalance is very close to zero, with an RMS of $7^\circ$; no significant variation with incidence angle is observed. These results suggest the suitability of the simplified calibration model for the calibration of fully polarimetric KAPRI data.
%The preprocessing starts with the correction of a frequency-depending shift of the antenna mainlobe direction, which reduces the range resolution of the data; this effect is caused by the slotted waveguide type antenna used by the system. The squint is corrected by a linear interpolation of the dechirped data in the chirp time-azimuth domain, yielding correctly focused range profiles resolved in azimuth by the antenna beamwidth.\\
%After the frequency dependent squint correction, an azimuthal phase ramp in the response of point-like scatterers was observed in the range compressed data, especially for the VV channel. A model was developed to explain this variation in terms of an antenna phase center horizontally displaced from the lever arm that causes a small variation in the slant range distance from the radar during the rotational scan. This change in range is too small compared to the range resolution to cause visible range cell migration. However, the variation in propagation path is suspected to induce the observed phase modulation. Using the model, the horizontal displacement was estimated and its effect was corrected using the model as a azimuth matched filter. The phase after the proposed filtering showed very little azimuthal variation and the data thus corrected can be used for polarimetric calibration.\\
%In the calibration procedure an additional step to remove the topographic phase contribution due to the spatial separation of the polarimetric phase centers was necessary. The topographic phase component was estimated with an interferogram obtained from channels with the same polarization separated by an nonzero baseline and rescaled to the equivalent baseline between the phase centers.\\ Once the topographic phase contribution was removed, the polarimetric calibration was performed by assuming zero crosstalk and estimating copolar imbalances using a trihedral corner reflector assuming the parameters to be independent from the  incidence angle. The crosspolar imbalances were estimated using distributed targets under the assumption of reciprocity.\\ The choice of neglecting crosstalk is supported by the use of temporal multiplexing for the acquisition of polarimetric data and by the high polarization purity of the antennas.\\ The calibration quality was tested by applying the method on a scene where five trihedral corner reflector were placed at different ranges from the radar. One of them was used to determine the parameters that where then applied to the entire image. Copolar phase and amplitude imbalances were then recomputed on the remaining reflectors; the mean amplitude imbalance is close to unity while the mean residual phase imbalance is very close to zero, with an RMS of $7^\circ$; no significant variation with incidence angle is observed. These results suggest the suitability of the simplified calibration model for the calibration of fully polarimetric KAPRI data.
\pagebreak
\bibliographystyle{IEEEtran}
\bibliography{library}
\end{document}
