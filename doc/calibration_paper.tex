\documentclass[12pt]{IEEEtran}
\usepackage{amsmath}
\usepackage{booktabs}
\usepackage{array} 
\usepackage{fullpage}
\usepackage{rotating}
\usepackage[section]{placeins}
\usepackage{cite}
\usepackage{amsthm}
\usepackage{subcaption}
\usepackage{amsfonts}
\usepackage{graphicx}
\usepackage{tikz}
\usetikzlibrary{positioning}
\usepackage[colorlinks=false]{hyperref}
\hypersetup{ hidelinks = true, }
\usepackage{multirow}
\usepackage{datatool}
\usepackage{pgfplotstable}
\usepackage[inline]{enumitem}

\input{/home/baffelli/PhD/trunk/Texts/macros.tex}
\graphicspath{{../outputs/img/}{/home/baffelli/PhD/trunk/Drawings/pdf/}{../fig/}}
%Title and authors
\title{Polarimetric Calibration of the KAPRI Ku Band Polarimetric Terrestrial Radar Interferometer}
\author{\IEEEauthorblockN{
	Simone Baffelli,
	Othmar Frey,
	Charles Werner
	Irena Hajnsek
}}

%\IEEEauthorblockA{\IEEEauthorrefmark{1}Affiliate 1, Adress}


\begin{document}
\maketitle
\begin{abstract}
Differential Interferometry using ground based radar systems permits to monitor fast changes in natural terrain with an high degree of flexibility in terms of location, time of acquisition and revisit time. In combination with polarimetric imaging, discrimination of different scattering mechanisms present in a resolution cell can be obtained while estimating the surface displacement.
In this paper we introduce KAPRI, a portable Ku-Band polarimetric radar interferometer. The system uses the deramp-on-receive FMCW architecture for ranging, azimuth resolution is obtained by a large slotted waveguide antenna with a beamwidth of $0.4^\circ$. The processing of KAPRI data into SLC images is addressed, including the correction of beam squint and of azimuthal phase variations. A polarimetric calibration model adapted to the acquisition mode is presented, that is used to produce calibrated polarimetric covariance matrix data. The methods are tested on a validation dataset of a scene containing five trihedral corner reflectors. The data processing is assessed by analyzing the oversampled response of a corner reflector, the polarimetric calibration quality is verified by computing polarimetric signatures and residual calibration parameters, which show a RMS HHVV phase imbalance under $7^\circ$ after the calibration.
\end{abstract}
\section{Introduction}
\PARstart{D}{ifferential} radar interferometry\cite{Gabriel1989, Massonnet1993} techniques are gaining importance for the monitoring of changes in the natural and built environment. The ability to accurately measure movements along the line of sight over large areas renders them interesting for a host of applications. Some examples are the estimation of subsidence rate associated with groundwater or oil extraction and the study of aquifer dynamics\cite{Fielding1998,Strozzi2001,Galloway2007a}, the monitoring of inflation/deflation connected to volcanic activity\cite{Massonnet1995}, the mapping of ice sheet and glacier motion\cite{Goldstein1993,Mohr1998} , the observation of landslides and instable slopes\cite{Carnec1996,Catani2005} and the measurement of seismic displacements\cite{Massonnet1993b,Zebker1994}.\\
Full polarimetric radar data provides additional information on the type of scatterers contained in each resolution cells, permitting a classification of the surface cover\cite{Cloude1997, Lee1999} and to extract geophysical parameters such as moisture content\cite{Hajnsek2003} or informations on the vegetation cover\cite{Ulaby1987} or the height of fresh snow\cite{Leinss2014}.\\
The combination of polarimetric and differential interferometric measurement could bring additional benefits for the observation of natural processes; for example by selecting the polarization that provides the best coherence and thus the least noisy phase measurement\cite{Pipia2009a, Iglesias2014b}.\\
Currently, the majority of radar data employed for differential interferometry is acquired using sensors carried by satellite or aircrafts. These platform are convenient in that they offer a large coverage in with a single data take. Nevertheless, because of costs, technical, physical and logistical restrictions, the revisit time of these system is limited to a few hours at best. 
In many cases, to better understand the dynamics of the natural processes and for real time surveillance and alarming, a denser temporal sampling over longer time spans is desired than afforded by current airborne and spaceborne systems.\\
For observation of smaller areas at shorter revisit times, ground based radar systems basing on different imaging principles were developed. While most  solutions provide range resolution by using some form of modulated wave, the main differences are observed in the method chosen to obtain resolution in the cross track dimension. 
Several systems\cite{Rudolf1999a, Rodelsperger2011, Aguasca2004,Rodelsperger2012} synthesize an aperture by moving the antenna assembly on a short motorized rail. Some systems of this type were designed with polarimetric data acquisition in mind\cite{Iglesias2014,LEE, Kang2009}.\\
\subsection{KAPRI: Real Aperture Polarimetric FMCW Radar}
The alternative to synthesizing an aperture by moving the sensor is to rotationally scan a fan beam antenna, separating the scatterers by using the  narrow beam emitted by a large antenna\cite{Werner2008}. This imaging method has been called type II in \cite{Caduff2015}. 
This configuration has two advantages compared to similar ground-based synthetic aperture systems based on a rail\cite{Monserrat2014} of comparable aperture length length: \begin{enumerate*}
  \item unlike a SAR system, each azimuth line is 
acquired independently from the others, eliminating the problem of decorrelation due to moving targets and atmospheric phase screens. 
These changes destroy the coherence of the scene over the aperture length and will cause spreading of these scatterer over
the azimuth direction, with a severe impact for the image quality, especially regarding the stability of coherent targets. \item a very large 
angular section can be imaged in one single pass, potentially up to $360^{\circ}$. This is not possible with a conventional ground based synthetic aperture radar, where only a small angular sector
can be covered at once; if it is desired to image another sector, the rail has to be rotated in order to look in the desired direction. \
\end{enumerate*}\\
One major limitation of real aperture radar imaging is the inability of reaching arbitrarily small azimuth resolutions: this would require a very large antenna, which is very difficult to build.
Therefore, the azimuth resolution is limited by electrical and mechanical design constraints. On the other hand, a physically large antenna offers the advantage of an higher gain and hence a better SNR, while in the case
of aperture synthesis, the antenna has to be physically small to have a wide beam, which reduces the antenna gain and potentially leads to a worse SNR.\\
Only very few polarimetric real aperture radars (type II systems) are in existence. A C-Band version of the GPRI has been produced\cite{Cherukumilli2012}, sadly only very little literature is avilable on this device. The rest of this paper will be focused on KAPRI, the Ku-Band Polarimetric Advanced Radar Interferometer (KAPRI). KAPRI is an extension of the GPRI \cite{werner_gpri_2012,Strozzi2011, Werner2008}, a portable real-aperture, $K_u$ band radar interferometer operating at 17.2 GHz with a wavelength of 1.74 cm . The original system is designed 
for the monitoring of unstable slopes by zero baseline differential interferometry \cite{Massonnet1993,JGRB:JGRB7093}; additionally a dual receiver and two antennas arranged along a spatial baseline, allow the acquisition of local digital elevation models.\\
To discriminate the scatters in the azimuth a 2 meter long vertically polarized slotted waveguide antenna is employed, giving an azimuth beamwidth of $0.4^\circ$ and an elevation beamwdith of approx. $30^\circ$.\\
In terms of hardware, the main feature that distinguishes KAPRI and GPRI II is the addition of horizontally polarized antennas and switches that permit to connect the transmitter and the receiver to either type of antenna. These changes, together with modifications in the control software, permit to acquire a full polarimetric-interferometric dataset in the space of four regular GPRI II pulses by temporal multiplexing, i.e by cycling through all the combinations of transmitted and received polarization.
\subsection{Goal of This Paper}
In this paper, the complete processing of KAPRI/GPRI raw data is addressed. In particular, two independent aspect of the processing chain are discussed in \autoref{sec:methods} :
\begin{enumerate}
	\item Processing of dechirped samples into single look complex (SLC) data squint. (\autoref{sec:proc_SLC})
	\item Polarimetric calibration of the SLC data. (\autoref{sec:proc_polcal})
\end{enumerate}
The processing is split into parts that logically correspond to subsequent stages in the evolution of GPRI-II into KAPRI. Part \ref{sec:proc_SLC} presents  methods employed to process raw data into SLC form. This includes a derivation of the FMCW signal model and the correction of frequency-dependent beam squint. These methods are relevant for both KAPRI and his predecessor GPRI-II. This discussion also forms the basis for part \ref{sec:proc_polcal} as properly focused SLCs acquired with different combinations of transmitting and receiving polarizations are a prerequisite for the polarimetric calibration. Part \ref{sec:proc_polcal} addresses issues specific to KAPRI, especially the problems arising when the antennas for the different polarizations are not collocated and the correction of phase and amplitude imbalances.\\
In \autoref{sec:results}, the method are applied on data acquired during a calibration campaign, where a number of trihedral corner reflectors (TCR) was used a reference target. The results are discussed in two separate parts, reflecting the structure of \autoref{sec:methods}. In \autoref{sec:res_SLC} the effect of antenna beam-squint correction is illustrated by analyzing the oversampled range compressed response of a TCR alongside the corresponding envelope in the slow-fast time plane. Because an unexplained azimuthal phase variation is observed in the trihedral reflector responses, a new model explaining it in terms of a phase center displacement is introduced and an additional correction step is presented. These modification will be applied to the data before the polarimetric calibration in \ref{sec:proc_polcal}. The calibration methods of \ref{sec:proc_polcal} are then applied  and the quality of the calibration is assessed by plotting polarimetric signatures and computing the residual calibration parameters on the calibrated data.\\ This way of structuring the results section is nonstandard, but the authors feel that it better reflects the iterations that led to the final processing strategy.\\
In \autoref{sec:conclusions} the methods are summarized again and some conclusions are drawn.
\section{Methods}\label{sec:methods}
\subsection{KAPRI: FMCW radar data processing}\label{sec:proc_SLC}
A fundamental requirement to generate calibrated polarimetric data is the availability of properly focused SLC images for all the desired polarizations. To obtain this data, it is necessary to understand the data acquisition process and correct several effects caused by the specific hardware design.  
For this purpose a signal model for type II\cite{Caduff2015} radar data using FMCW signaling\cite{Stove1992} is introduced.\\
\begin{figure}[h]
	\centering
	\includegraphics[scale=1]{real_aperture_signal_model_geometry}
	\caption{Geometry used to derive the FMCW signal model. $R$ is the slant range from the radar to the point scatterer, $\theta_{3dB}$ is the antenna half power beamwidth, which is represented by the gray triangle. The size of the antenna aperture is $L_{ant}$, the corresponding azimuth resolution (in distance units) is $\delta_{az}$. The inset figure is used to derive the azimuth phase variation. $L_{ph}$ is the phase center displacement, $L_{arm}$ is the antenna rotation lever arm, $R_{0}$ the range of closest approach and $\alpha$ the additional angle to obtain closest approach when the phase center is not in the midpoint of the array.}
	\label{fig:real_aperture_signal_model_geometry}
\end{figure}
Consider a coordinate system with origin at the location of a radar as depicted in \autoref{fig:real_aperture_signal_model_geometry}. In this system, the antenna is mounted on a lever arm of length $L_{arm}$; its mainlobe is parallel to the $x$ axis when the pointing angle $\theta$ is 0. The radar images a scene with a complex reflectivity distribution $\rho\left(x,y\right)$ by measuring range profiles $\hat{\rho}\left(R, \theta\right)$ for a number of antenna rotation angles (azimuths) $\theta = \operatorname{arctan}\left(\frac{y}{x}\right)$ by rotating the antenna with the angular speed $\omega$. Each profile is measured by transmitting a linearly modulated signal of duration $t_{chirp}$ with bandwidth $B$ and center frequency $f_c$:\\
\begin{equation}
	s_t\left(t\right) = e^{j 2 \pi \left( t f_{c} +  \frac{B}{t_{chirp}} t^2 \right)}.
\end{equation}
In the case of KAPRI, $f_c= 17.2~GHz$ and $B=200~MHz$.\\
The total time that the signal takes to travel to a scatterer at range $R$ and back is $t_{c} = \frac{2 R}{c}$. The backscattered signal $s_r$ is a copy of the one being transmitted; delayed by $t_{c}$ and scaled by the complex reflectivity of the scatterer $\rho$.
\begin{equation}
		s_r\left(t\right) = \rho e^{j 2 \pi \left( \left(t - \frac{2 R}{c}\right) f_{c} +  \frac{B}{t_{chirp}} \left(t - \frac{2 R}{c}\right)^2 \right)}.
\end{equation}
The received backscattered signal is $s_r$  mixed with the transmitted chirp $s_t$  to remove the linear modulation; the resulting beat signal is then sampled and stored in the device. It frequency $f_{b} = \frac{4 R B}{c t_{chirp}}$ is proportional to the slant range $R$:
\begin{equation}\label{eq:deramp}
	\begin{aligned}
	s_{d}\left(t\right) &=s_t\left(t\right)s_r\left(t\right)^* =\\ 
	&\sigma e^{j 4 \pi \frac{ R}{c}f_c}  e^{j 4 \pi \frac{2 R B }{c t_{chirp}} t}  e^{j 4 \pi \frac{2 R^2 B}{c^2}}.
	\end{aligned}
\end{equation} 
In this expression, two phase terms can be identified: $ e^{j 4 \pi \frac{R}{c}f_c}$ is the two way propagation phase, the quantity of interest for  interferometric measurements. The second phase component is the residual video phase (RVP). This component needs to be compensated for SAR processing, where its variation during the aperture time may cause defocussing.\\
From  linearity it follows from \autoref{eq:deramp} that the range profile $\hat{\sigma}\left(R, \theta\right)$ of a collection of targets with complex reflectivities $\rho_i$ located at ranges $R_{i}$ is recovered by taking the Fourier transform of $s_{d}\left(t\right)$.\\
As for pulsed radars, the range resolution  for a FMCW system is inversely proportional to the bandwidth $\delta_{r} = \frac{c}{2 B}$. Using $B=200 MHz$ KAPRI achieves a range resolution of 0.75 m\cite{Strozzi2011}; however the effective range resolution is lower because the dechirped data is windowed before the Fourier transform to mitigate range sidelobes. It is interesting to note that for FMCW system the sampling rate is not governed by the transmitted bandwidth; instead it is dictated by the extent of ranges to be imaged\cite{Meta2006}.\\
To obtain two dimensional images range profiles are acquired while the antenna is rotated with angular velocity $\omega$.
Thus, the dechip signal for a point target at $R, \theta_{t}$ in the slow-time versus fast time plane is:
\begin{equation}\label{eq:signal_model}
	\begin{aligned}
	& s_{d}\left(t,\tau\right) = \sigma e^{j 4 \pi \frac{ R}{c}f_c}   \\
	& e^{j 4 \pi \frac{2 R B }{c t_{chirp}} t}  e^{j 4 \pi \frac{2 R^2 B}{c^2}} P\left(\tau \omega - \theta\right),
	\end{aligned}
\end{equation} 
where $t$ is the fast time, $\tau = n t_{chirp}$ is the slow time variable and $P\left(\tau \omega\right)$ describes the two way antenna pattern. Its beamwidth is approximated by:
\begin{equation}\label{eq:azimuth_resolution}
	\theta_{3dB} = \frac{\lambda}{L_{ant}}
\end{equation}
where $L_{ant}$ is the size of the antenna aperture and $\lambda$ is the wavelength employed.
Due to diffraction, the radiation beam emitted by the antenna broadens linearly with increasing distance consequently, the spatial resolution in cross-range $\delta_{az}$ increases with distance:
\begin{equation}\label{eq:azimuth_ground_resolution}
	\delta_{az} = \frac{R \lambda}{L_{ant}}.
\end{equation}
%The typical applications of KAPRI require a ground resolution on the order of a few meters at ranges up to 8 km, requiring a beamwidth. To obtain a beamwidth of $0.4^\circ$ at $17.1 GHz$, $L_{ant} = 2m$. 
KAPRI employs a traveling wave slotted waveguide antenna\cite{Hines1953a,Granet2007}; it constructed by cutting slots resonating at the design frequency in a section of rectangular waveguide. When they are appropriately spaced, the fields emitted at each cut combine in phase, producing a narrow beam. Two types of slotted waveguide antenna exist\cite{Enjiu2013}: the resonant and the traveling wave design. The second type has been chosen because it supports a larger bandwidth; this permits to achieve a finer range resolution. However this antenna design displays frequency-dependent beam squint: if it is operated at a frequency other than the design value, the phase difference at the slots changes with the effect of squinting the beam away from the intended mainlobe direction. This effect has been used for several fast imaging radar, where a mechanical antenna rotation would not be possible\cite{Yang2014,Yang2012,Mayer2003,Alvarez2013}. In the case of KAPRI the squint is undesired: since the amount of squint is large compared to the beamwidth, the beam only illuminates a scatterer for a fraction of $t_{chirp}$, decreasing the effective transmitted bandwidth and hence the range resolution.  If the antenna is rotated slowly enough,  the spacing of each range profile is less than the beamwidth and the effect of beam squint is visible in the slow-fast time domain as skewing of point target responses, as illustrated in\autoref{fig:squint_correction}: the target is only inside the beamwidth during the chirp at the time when the antenna rotation angle matches the beam squint angle. 
The azimuth-frequency skewing observed in oversampled data is key to mitigate the effect of the beam squint. For each chirp frequency $f$, the data at the corresponding fast time $s_{d}\left(t,\tau\right)$ is shifted back in azimuth by $\tau_{sq}=\frac{\theta_{sq}}{\omega}$, where the squint angle is:
\begin{equation}\label{eq:squint_exact}
	\theta_{sq} = \sin^{-1}\left(\frac{\lambda}{\lambda_{g_{ij}}} + \frac{k \lambda}{s}\right).
\end{equation}
here ${\lambda_g}_{ij}$ is the wavelength for the $ij$-TE mode of the waveguide, $\lambda$ is the freespace wavelength, $s$ is the element spacing and $k$ is the mode number. In this case, the waveguide mode used is TE01 and $k=0$ is assumed because all the slots are to be transmitting in phase\cite{kraus88} to direct the main beam at the antenna broadside.\\
In processing KAPRI data a linear approximation for the squint relative to the pointing at the design frequency $f_c$ is used instead:
\begin{equation}\label{eq:squint_linearised}
	\theta_{sq} - \theta_{sq}^{f_{c}}  =  \alpha f.
\end{equation}
this choice is necessary because manufacturer-provided antenna pattern measurements at different frequencies suggest that the vertically and the horizontally polarized units have different squint characteristics.\\ No other design information being available, a data-driven method had to be used for the correction of the beam squint. To estimate the linear squint rate $\alpha$, the response of a strong point-like target is extracted from the data by windowing it in range compressed data and converting it back in the fast time domain; its envelope is then extracted with a discrete Hilbert transform. Finally a fit using the model of \autoref{eq:squint_linearised} is performed to obtain an estimate of $\alpha$.\\
To correct the frequency-dependent beam squint the azimuth sample spacing smaller must be smaller than the antenna beamwidth: this permits to reconstruct the full bandwidth illumination of each scatterer with by combining chirp samples acquired at subsequent azimuth positions. A side benefit of oversampling is that several range profiles that in first approximation can be seen as representing independent realizations of the same scatterers can be averaged to improve the measurement SNR, producing range-azimuth images with an angular resolution limited by the antenna beamwidth.\\
\begin{figure}[ht]
	\centering
	\includegraphics[scale=0.2]{squint_correction}
	\caption{Illustration of the frequency-dependent antenna squint. When the antenna rotates during the electronic scan, the energy of the target is spread through several azimuth bins.}
	\label{fig:squint_correction}
\end{figure}
After correcting the beam squint, the first and last $z$ samples of the raw data $s_{d}$ are windowed with an Hann window to mitigate the transient signal caused by the abrupt change in frequency due to the repetition of the chrip. A second Kaiser Window is then applied to reduce range processing sidelobes that are caused by the finite bandwidth. Finally, a fast time Fourier transform implements the range compression to obtain the SLC image $\hat{\rho}\left(R, \theta\right)$.\\ Each range line of the SLC thus compressed is then multiplied by $\sqrt{R^3}$ to compensate for the power spreading loss. In this manner, the intensity of the SLC data is directly proportional to the radar brightness $\beta_{0}$.
\subsection{Antenna Pattern Misalingment}\label{sec:misalingment}
In addition to the difference in frequency-dependent squint rate, the horizontally and the vertically polarized antennas appear have an azimuth mispointing of approx. $0.2^\circ$. It was first observed by analyzing the response of strong point-like target, where a significant misregistration between the $HH$ and the $VV$ images was observed. The misalignment is particularly problematic for cross-polar measurements: the transmitting and receiving patterns are not aligned. Using the available pattern information a power loss of approx 2.5 dB  compared to the ideal case is predicted. This loss reduces the SNR for the cross-polar channels, leading to noisier measurements.\\ While the offset between copolar channels can be corrected by coregistration without major consequences, there is no method capable to compensate the SNR loss in the crosspolar measurements a posteriori. An adjustable antenna mount was manufactured by replacing one of two hinges where the antennas are fixed on the towers (see \autoref{fig:adjustable_bracket}) with an adjustable bracket that allows to slide the antenna back and forth on the one side, obtaining the effect of rotating it around the center. Basing on the size of the antenna mounting bracket and on the amount of misaligment, it was determined that the horizontally polarized antennas need to be shifted by 1.8 mm to compensate form the pattern mispointing.
\begin{figure}[ht]
	\centering
	\includegraphics[scale=0.3]{antenna_offset_2}
	\caption{Illustration of the adjustable antenna mount allowing to shift the patterns to bring the H and the V antennas into alignment.}
	\label{fig:adjustable_bracket}
\end{figure}
\subsection{Polarimetric Calibration}\label{sec:proc_polcal}
In the case of GPRI the processing is concluded after range compression and the data is ready for interferometric applications. In the case of  KAPRI, additional steps are necessary to obtain polarimetric measurements that are well calibrated.\\ The first of these steps requires a brief review of the polarimetric antenna configuration, as it is depicted in \autoref{fig:antenna_arrangement}:\\ Six antennas are mounted on a supporting structure connected to the rotary scanner. Of these, 2 are transmitting antennas, one for each polarization. The remaining 4 are connected through switches to the receivers in pairs; each pair composed of an horizontally and a vertically polarized unit. This configuration permits to acquire full polarimetric dataset by selecting the desired antennas for the transmitter and for each receiver separately. Because only one combination is acquired at a time and the antennas are not collocated, this arrangement ensures a low level of polarimetric crosstalk. Additionally, the separation of transmitting and receiving antennas increases the TX-RX isolation, a fundamental requirement for FMCW performance\cite{Beasley1990,Stove1992, Strozzi2011}.  However, this configuration complicates the phase calibration of polarimetric data.
\begin{figure}[ht]
	\centering
	\includegraphics[scale=0.15]{kapri_antenna_arrangement}
	\caption{KAPRI with the usual antenna arrangement overlaid. }
	\label{fig:antenna_arrangement}
\end{figure}
Consider an image $i$ acquired by transmitting at antenna located at $\mathbf{x_t^i}$ and receiving at $\mathbf{x_r^i}$. These antennas can be replaced by an equivalent antenna located at $\mathbf{x_{eq}^i}$, the midpoint between transmitter and receiver \cite{Pipia2009}. Thus, for polarimetric observations using  KAPRI the equivalent antenna phase center $\mathbf{x_{eq}^i}$ location will change depending on the chosen polarization. Therefore, there exist certain combinations of channels $i$ and $j$ where the equivalent phase centers will be separated by a baseline $\mathbf{b_{ij}^{eq}}$. Consequently, the polarimetric phase difference $\phi_{ij} = \phi_{ij}^{pol} + \phi_{ij}^{prop}$  will contain an interferometric contribution $\phi_{ij}^{prop}$. This term appears as topographic fringes when visualizing the polarimetric phase difference and will complicate calibration by adding an additional phase contribution unrelated to the polarimetric properties of the scatterers.\\
To obtain valid polarimetric phase differences, it is necessary to estimate and subtract the interferometric contribution. This can be done  by considering two additional channel $k$ and $l$  acquired with  a non-zero baseline $\mathbf{b_{ij}^{eq}}$ and with the same polarization, where $\phi_{kl}^{pol} \approx 0$. Generally, for any two channels $m$ and $n$,  the propagation phase difference can be approximated as a function of the local incidence angle and of the perpendicular baseline separating the phase centers:
\begin{equation}\label{eq:prop_approximation}
		\phi_{mn}^{prop} = \frac{4\pi}{\lambda} b_{mn}^{eq} \sin(\theta - \alpha_{bl}),
\end{equation}
where $\alpha_{bl}$ is the baseline angle w.r.t to the vertical and the look angle $\theta_l$ is the angle between the line of sight vector $\mathbf{p}$ and the vertical axis and $b_{ij}^{eq}$ is the perpendicular baseline between the equivalent phase centers.
Thus from \autoref{eq:prop_approximation}, $\phi_{ij}^{prop}$ can be estimated from $\phi_{kl}$ if the look angle does not significantly change from $kl$ to $ij$, i.e if $\theta_{ij} - \alpha_{ij} \approx \theta_{kl} - \alpha_{kl}$. 
\begin{equation}
	\hat{\phi}_{ij}^{prop} = \frac{b_{eq}^{ij}}{b_{eq}^{kl}} \phi_{kl}.
\end{equation}
This formula can only be used if $\frac{b_{eq}^{ij}}{b_{eq}^{kl}}$ is integer\cite{Massonnet1996}, if this condition is not met, phase unwrapping of $\phi_{kl}^{prop}$ is necessary before rescaling.\\
In order to correct for all the possible combinations that have a non-zero baseline, the measured scattering matrix $\mathbf{S}$ is converted into a polarimetric covariance matrix;  $\hat{\phi}_{ij}^{prop}$ it then subtracted from the phase of every non-diagonal element $ij$. The result is a flattened covariance matrix where the sole phase contribution is the polarimetric phase difference.\\
This matrix is the starting point for the polarimetric calibration proper;
the procedure is based on the linear distortion matrix model\cite{Saraband1990, Sarabandi1992a} that relates the observed scattering matrix $\mathbf{S_{meas}}$ with the correct matrix $\mathbf{S}$:
\begin{equation}\label{eq:distorsion_scattering}
	\mathbf{S_{meas}} = \mathbf{R} \mathbf{S} \mathbf{T}.
\end{equation}
or in covariance form
\begin{equation}\label{eq:covariance_distortion}
	\mathbf{C}_{meas} = \mathbf{D} \mathbf{C} \mathbf{D}^{H}.
\end{equation}
where $\mathbf{D}$ is the Kronecker product of $\mathbf{R}$ and $\mathbf{T}$, the matrices that describe the phase and amplitude imbalances and crosstalk in reception and transmission.
In the case of KAPRI, crosstalk calibration is not performed as the radar is expected to have a good polarization isolation, largely due to the fact that only one polarization is acquired at a time by selecting the appropriate combination of transmitting and receiving antennas. The only source of crosstalk is the presence of cross-polarized lobes in the direction of the antenna mainlobe. The manufacturer has provided simulated radiation patterns for the horizontally polarized antennas, where the isolation between the co and the cross polarized pattern in the main-lobe direction is observed to be better than 60 dB. By computing the $VV-HV$ ratio of the oversampled response of a trihedral corner reflector, the polarization purity of the system was estimated to be better than 35 dB.\\
Thus, neglecting crosstalk the matrices can be written as:
\begin{equation}
	\begin{aligned}
	&\mathbf{R} = A \begin{bmatrix}
		1 & 0\\
		0 & f/g e^{i\phi_{r}}
	\end{bmatrix},\\
	&\mathbf{T} = A \begin{bmatrix}
			1 & 0\\
			0 & f g e^{i\phi_{t}}
		\end{bmatrix}
	\end{aligned}
\end{equation}
where $f$ is the one-way copolar amplitude imbalance with respect to the $H$ polarization, and $g$ the amplitude imbalance of the crosspolarized channels. $\phi_t = \phi_{t,h} -\phi_{t,v}$ is the phase offset between the polarizations when transmitting and $\phi_{r} = \phi_{r,h} -\phi_{r,v}$ is the phase offset in reception and $A$ is the absolute amplitude calibration parameter (RCS)\cite{Ainsworth2006a, Fore2015}.\\
The four unknown complex parameters in $\mathbf{D}$ can be determined using a trihedral corner reflector and a reciprocal scatterer with a significant cross polarized contribution\cite{Sarabandi1989,Pipia2009}.\\
With the above model, an ideal trihedral reflector with the scattering matrix
\begin{equation}
 \mathbf{S} = \sqrt{\sigma_{tri}}
 \begin{bmatrix}1 & 0\\ 0 & 1\end{bmatrix}
\end{equation}
where $\sigma_{tri}$ is its RCS, has a measured covariance matrix $\mathbf{C^{\prime}}$:
\begin{equation}
	\begin{aligned}
	&\mathbf{C^{\prime}} =\\
	&= k \sigma_{tri}\\
	&\begin{bmatrix}
		1 & 0 & 0 & f^2 e^{-i \left(\phi_t + \phi_r\right)}\\
		0 & 0 & 0 & 0\\
		0 & 0 & 0 & 0\\
		f^2 e^{i \left(\phi_t + \phi_r\right)} & 0 & 0 & f^4
	\end{bmatrix}
	\end{aligned}.
\end{equation}
The copolar amplitude imbalance $f$ is estimated of the HHHH and VVVV elements of $\mathbf{C^{\prime}}$:
\begin{equation}
	f = \left(\frac{C^{\prime}_{VVVV}}{C^{\prime}_{HHHH}}\right)^{\frac{1}{4}}.
\end{equation}
Similarly, the copolar imbalance phase $\phi_r + \phi_t$ is determined from the phase of $C_{VVHH}^{prime}$:
\begin{equation}
	\phi_r + \phi_t = \operatorname{arg}\left(C_{meas}^{VVHH}\right).
\end{equation}
Both parameters are estimated on the oversampled response of a corner reflector.
Because of the difficulty of placing and correctly orienting a dihedral reflector, the estimation of $g$ and $\phi_t - \phi_r$ is based on the assumption that most pixels in the calibration dataset represent reciprocal scatterers:
\begin{equation}
	g = \left<\frac{C_{rec}^{HVHV}}{C_{rec}^{VHVH}}\right>^\frac{1}{4},
\end{equation}
and:
\begin{equation}
	\phi_t - \phi_r =\operatorname{arg}\left( \left<C_{meas}^{HVVH}\right>\right).
\end{equation}
When $\mathbf{D}$ is estimated, \autoref{eq:covariance_distortion} is inverted to obtain a calibrated covariance matrix.\\
If radiometric calibration is desired, the value of $A$ can be determined after imbalance correction:
\begin{equation}
	A =	\left(\frac{\sigma_{tri}}{C^{\prime}_{HHHH} R^{3}}\right).
\end{equation}
where $R$ is the slant range to the trihedral corner reflector.
\subsection{Experimental Data}\label{sec:data}
A calibration dataset was acquired in September 2016 at an urban-agricultural area near M\"{u}nsingen, Switzerland to test the methods described above. The data was acquired from the top of an hill approximately 800 m high, looking down towards fields and the town. 6 Trihedral Corner Reflectors were placed in the scene for the determination of calibration parameters and to assess imaging quality. Three of these reflectors have triangular faces with a length of 40 cm, corresponding to a RCS of $\sigma=\frac{4}{3}\pi \frac{a^4}{\lambda^2}=25.5 dB$, while the remaining two are cubic corner reflector with $\sigma= 35 dB$, at the nominal central frequency of 17.2 GHz.
	\begin{figure}
		\centering
		\includegraphics{figure_7}
		\caption{Pauli RGB composite of the imaged scene. The location of calibration corner reflector is marked by blue circles.}
		\label{fig:pauli_rgb}
	\end{figure}
The dataset was acquired with the horizontally polarized antenna group shifted towards the V group by 1.8 mm to compensate for the pattern misalignment as described in \autoref{sec:misalingment}.\\
\autoref{fig:pauli_rgb} shows the calibrated Pauli RGB composite of the scene, resampled in Cartesian coordinates with the location of the reflectors marked using blue circles; their location and RCS is summarized in \autoref{tab:reflectors}.\\
A separate dataset containing a dihedral reflector was acquired at our H\"{o}nggerberg campus in order to investigate the effect of antenna pattern misalignment on crosspolar acquisitions and to test the suitability of the computed adjustment value as discussed in \autoref{sec:misalingment}. This was done separately because of the logistical complications associated with the transportation and the setup of large calibration targets.


\begin{table}[ht]
	\centering
	\pgfplotstabletypesetfile[
								every head row/.style={before row=\toprule,after row=\midrule},
								every last row/.style={after row=\bottomrule},
								col sep=comma,
								columns={rsl, RCS, type},
								columns/type/.style={column name={Type} ,string type},
								columns/RCS/.style={precision=3, column name=$\sigma_0$},
								columns/rsl/.style={precision=3, column name=$R_0$}
										] {../tab/table_1.csv}
	\caption{Summary of the employed TCRs. Distance from the radar and expected RCS.}
	\label{tab:reflectors}
\end{table}



%\begin{figure}[ht]
%	\centering
%	\begin{subfigure}[b]{\columnwidth}
%		\centering
%		\includegraphics[scale=1]{HIL_20140910_144113_l_03_normal_gc_phase.pdf}
%		\subcaption{Uncorrected}
%		\label{fig:HHVV_phase:unflattened}
%	\end{subfigure}
%	\begin{subfigure}[b]{\columnwidth}
%		\centering
%		\includegraphics[scale=1]{HIL_20140910_144113_l_03_flat_gc_phase.pdf}
%		\subcaption{Corrected}
%		\label{fig:HHVV_phase:flattened}
%	\end{subfigure}
%	\caption{HH VV phase difference. \autoref{fig:HHVV_phase:unflattened}: Phase difference after the correction of the azimuth ramp. The topographic phase ramp is very clearly visible. In  \autoref{fig:HHVV_phase:flattened} the phase after the topographic phase removal is shown. There is no noticeable phase trend.}
%	\label{fig:HHVV_phase}
%\end{figure}
%\FloatBarrier


\section{Results}\label{sec:results}
\subsection{FMCW Data Processing}\label{sec:res_SLC}
In order to assess the processing quality and to appreciate the effect of the beam squint correction, in  \autoref{fig:tcr_mph} the oversampled impulse response of the 90 cm TCR at 659 m slant range is plotted. The plots on the left side (\autoref{fig:tcr_mph:HH_uncorr}, \autoref{fig:tcr_mph:VV_uncorr}) contain the response obtained by directly range compressing the dechirped data. The phase displays interesting patterns, with the equiphase contours forming a cross shape centered at the center of the reflector. The shift between the HH and the VV channel is visible too, as the response for the one channel is shifted with respect to the other plot.\\ A visual comparison with the same response obtained enabling the squint compensation, shows an improvement in both the range and the azimuth resolutions; this is confirmed numerically by fitting a spline on the azimuth and range responses and computing its 3 dB width. The numerical results are shown above each plot of \autoref{fig:tcr_mph}; signficant improvements in both range and azimuth are clear, especially so for the HH channel, that shows a range resolution improvement of 0.24 m. This gain is reduced to only 0.17 m for the VV channel.\\ Similarly, a reduction in the phase variation is visible especially for the HH channel, (\autoref{fig:tcr_mph:HH_corr}) while the phase response of VV channel shows a residual phase ramp in azimuth direction (\autoref{fig:tcr_mph:HH_corr}).
\def\chutzepref{20160914_145059}
\begin{figure*}[ht]
	\begin{subfigure}[b]{0.6\columnwidth}
		\centering
		\includegraphics[scale=0.7]{\chutzepref_1331_1490_AAAl_chan_mph_plot}
		\subcaption{HH channel without beam squint correction.}
		\label{fig:tcr_mph:HH_uncorr}
	\end{subfigure}~
	\begin{subfigure}[b]{0.6\columnwidth}
		\centering
		\includegraphics[scale=0.7]{\chutzepref_1331_1490_AAAl_desq_mph_plot}
		\subcaption{HH channel with squint correction.}
		\label{fig:tcr_mph:HH_corr}
	\end{subfigure}~
	\begin{subfigure}[b]{0.6\columnwidth}
		\centering
		\includegraphics[scale=0.7]{\chutzepref_1331_1490_AAAl_corr_mph_plot}
		\subcaption{HH channel with squint and azimuth ramp correction.}
		\label{fig:tcr_mph:HH_corr_ph}
	\end{subfigure}\\
	\begin{subfigure}[b]{0.6\columnwidth}
		\centering
		\includegraphics[scale=0.7]{\chutzepref_1331_1490_BBBl_chan_mph_plot}
		\subcaption{VV channel without beam squint correction.}
		\label{fig:tcr_mph:VV_uncorr}
	\end{subfigure}~
	\begin{subfigure}[b]{0.6\columnwidth}
		\centering
		\includegraphics[scale=0.7]{\chutzepref_1331_1490_BBBl_desq_mph_plot}
		\subcaption{VV channel with squint correction.}
		\label{fig:tcr_mph:VV_corr}
	\end{subfigure}~
	\begin{subfigure}[b]{0.6\columnwidth}
		\centering
		\includegraphics[scale=0.7]{\chutzepref_1331_1490_BBBl_corr_mph_plot}
		\subcaption{VV channel with squint and azimuth ramp correction.}
		\label{fig:tcr_mph:VV_corr_ph}
	\end{subfigure}~

	\caption{Oversampled phase and amplitude response for the trihedral corner reflector.}
	\label{fig:tcr_mph}
\end{figure*}
\subsection{Antenna Pattern Misalingment}
\begin{figure}[ht!]
	\centering
	\includegraphics{../outputs/img/HV_gain}
	\caption{Before H-V shift correction}
	\label{fig:hv_power:uncalibrated}
	\caption{Azimuth response for a point-like target with high crosspolarized contribution.}
	\label{fig:hv_power}
\end{figure}
To verify the impact of the $H$-$V$ pattern pointing shift on the performance of crosspolar measurement, the response of a dihedral reflector with an high crosspolar contribution is analyzed for two configurations:\\ \begin{enumerate*}\item the case where the antennas are not mechanically moved \item the case where the optimal shift of 1.8 mm is applied to bring the patterns into alignment.\\ Because of logistical problems, this experiment was carried out at the H\"{o}nggerberg campus separately from the rest of the measurements shown in this section.
\end{enumerate*}
In \autoref{fig:hv_power}, the result of the above experiment is shown as the oversampled, coregistered azimuth response in the $HV$ channel; a gain of approx. 2.5 dB is observed after correcting for the pattern misalignment. The estimated gain is very close to the one expected by analyzing the provided antenna patterns. This results confirms that the 1.8 mm shift setting employed to acquire the calibration data can correctly compensate the crosspolar power lost and the $HH-VV$ misregistration. However,  
\subsection{Azimuth Processing}
This residual phase described in \autoref{sec:res_SLC}  is problematic in two manners: \begin{enumerate*}
	\item because of the coherent average of pixels affected by the phase ramp\label{item:SNR}  $\sigma$ after azimuth averaging (see \autoref{sec:proc_SLC}) will appear smaller than it actually is.
  \item because of the different phase behavior of the antennas, the coherence magnitude of polarimetric phase differences will be reduced; the resulting coherence phase will be affected by a residual phase variation. For example, if the HH-VV phase is needed for calibration in the method described in \autoref{sec:proc_polcal} and the azimuth variation is not taken into account, the additional phase will lead to incorrect calibration parameters.\label{item:phase_variation}\end{enumerate*}
The ramp are caused by the antennas being mounted offset from the rotation center of the  radar~\cite{Lee2014}, so that during the scan the distance from the antenna phase center to a fixed point in the scene changes as a function of the antenna azimuth position. From a geometrical perspective it can be interpreted as a radar moving along a circular trajectory of radius $L_{arm}$. In SAR system the phase variation is exploited to synthesize a large aperture by increasing the antenna beamwidth and the length of the trajectory, so that the illumination time of the point is maximized. In contrast to that case, the illumination time here is to short to allow synthesizing an aperture larger than the antennas physical beamwidth and the only effect of the circular trajectory is the azimuth phase modulation.\\
To quantify the phase modulation, consider the antenna configuration in \autoref{fig:real_aperture_signal_model_geometry}:
The antennas are mounted on a lever arm of length $L_{arm} = 0.25$ m that connects them to the azimuth scanner. The phase center of the antenna is horizontally displaced from the lever arm attachment by $L_{ph}$, this parameter was added empirically to obtain better phase fits; it models antenna manufacturing imprecisions. A point target is considered, located at the slant range  of closest approach $R_{0}$, obtained when the phase center, the target and the lever arm all lie on a line. The antenna is now rotated by an angle $\theta_r$ relative to the situation of closest approach. The distance $R$ from the target will change during this movement and a corresponding variation of the phase will observed, according to:
\begin{equation}\label{eq:range_phase}
	\phi_{scan} = \frac{4 \pi}{\lambda}R\left(\theta_r\right).
\end{equation}
To compute $R$, the law of cosines is applied to the triangle of~\autoref{fig:real_aperture_signal_model_geometry}, with the included angle $\theta_r$, one side length $L_{ant} = \sqrt{L_{arm}^2 + L_{ph}^2}$ and the other side $c = L_{ant} + R_0$.  $L_{ant}$ is the equivalent antenna rotation arm for a system with no phase center shift.
$R$, the range from the target to the antenna as a function of the rotation from the closest approach $\theta_r$ is:
\begin{equation}\label{eq:range}
	R = \sqrt{ c^2 +  L_{ant}^2 - 2 c L_{ant} \cos{\left(\theta_r - \alpha\right)}}.
\end{equation}
The function is shifted by the angle  $\alpha = \operatorname{\arctan}\left({\frac{L_{ph}}{L_{arm}}}\right)$ that describes the additional rotation needed for an antenna with nonzero $L_{ph}$ to be at closest approach with the target compared to the regular case. This shift is necessary because antenna rotation is measured assuming a zero $L_{ph}$; the azimuth position read at the encored of the antenna scanner does not correspond to the azimuth of the phase center.\\
As seen in \autoref{eq:range}, the distance of each scatterer is a function of the relative rotation angle $\theta_r$ of the antenna. Thus azimuth and distance are coupled, this effect is known as Range Cell Migration RCM. In this case, the range resolution is small enough compared to the RCM that only the effect of the rotation on the phase (\autoref{eq:range_phase}) will be considered.\\
The complete characterization of the azimuth phase requires the knowledge of the antenna phase center displacement $L_{ph}$. This value is generally not known a priori as it is assumed that the phase center is located at the midpoint of the array. However, when the experimental data was analyzed assuming this case to be true (which implies $L_{ant} = L_{arm}$),  \autoref{eq:range} failed to model the observed azimuth phase variation for the response of a trihedral corner reflector. The model was thus modified to account for the possibility of a displaced antenna phase center. The parameter $L_{ph}$ can be determined from the measured data of a point target. SLC images are generated according to the procedure described in \autoref{sec:proc_SLC}. The azimuth phase profile for the point target is then extracted and used in a nonlinear optimization problem with the phase simulated according to \autoref{eq:range}:
\begin{equation}\label{eq:rph_estimation}
	\hat{r_{ph}} = \underset{\left(L_{ph}, \phi_{off}\right)}{\operatorname{argmax}}{\vert\vert\phi_{meas} - \phi_{sim}\vert\vert}^2.
\end{equation}
Where $\phi_{sim} = \phi_{scan} + \phi_{off}$ is the simulated phase with an additional offset that accounts for the phase induced by the noise, the intrinsic scattering phase and the system.
The model is tested by applying the fit procedure of \autoref{eq:rph_estimation} to the calibration array. For each reflector, the maximum in range and azimuth was identified and the samples corresponding to the half power beamwidth were extracted at the range of maximum intensity. The unwrapped phase was then used to estimate $r_{ph}$ according to the procedure described in \autoref{eq:rph_estimation}.
\begin{table}[ht]
	\centering
	\begin{tabular}{lccl}
		\hline
		reflector & $R_0$ & $\hat{L_{ph}}$ & comments\\
		1	& 74 m & -0.17 & \\
		2  & 672 m & -0.10 & \\
		3 & 824 m & -0.11 & \\
		4 & 838 m & -0.14 &\\
		5 & 1048 m & -0.11&\\
		6 & 2690 m & -0.11&\\
		\hline
	\end{tabular}
	\caption{Result of the phase center displacement fit for six trihedral corner reflectors located at different ranges.}
	\label{tab:rph_fit}
\end{table}
The resulting fit values (\autoref{tab:rph_fit}) have a standard deviation of less than 2 \% of the antenna length, suggesting that the model is able to predict  most of the phase variation.\\
When $L_{ph}$ has been estimated, the azimuth phase variation is corrected using \autoref{eq:range} as a range-variant matched filter. Each azimuth line in the range compressed, squint corrected data $s_{d}$ is convolved with:
\begin{equation}\label{eq:correction}
	\begin{aligned}
		s_{d}^{corr}\left(\theta, R_{0}\right) = &\int\limits_{-\frac{L_{int}}{2}}^{\frac{L_{int}}{2}}e^{\jmath \frac{4\pi}{\lambda}\left(R\left(\theta - \theta^{\prime}, R_{0}\right) - R_{0}\right)}\\
		&s_{d}\left(\theta^\prime\right)w(\theta - \theta^{\prime}) d\theta^\prime,
	\end{aligned}
\end{equation}
where $w$ is a windowing function. The formula replaces the incoherent azimuth averaging described in \autoref{sec:proc_SLC} the samples are now averaged with the appropriate phase factor so that the SNR improvement and the correction of the azimuth trend are combined in a single step. In computing the filter, the range of closest approach $R_{0}$ is subtracted from the current range $R$ to correct the relative phase variation only; this is important for interferometric processing where the absolute phase has to be preserved.\\
The procedure is similar to the azimuth focusing of synthetic aperture data, where the cross-range resolution is obtained by the integration of the data in the azimuth-time direction. However, in the case of real aperture systems the resolution is limited by  physical antenna beamwidth and the response of a target ideally occupies a single azimuth sample. Integrating the data in azimuth degrades the resolution because samples that do not contain information on the same scatterer are combined together. To correct the phase variation without an excessive deterioration in azimuth resolution, the integration is limited to a window  $w$ of length $L_{int}$. The optimal trade-off was empirically determined to be $0.6^\circ$, slightly larger than $\theta_{3dB}$.\\
The result of applying the correction to the TCR at 658 m is displayed in the rightmost column of \autoref{fig:tcr_mph}; the phase ramp is significantly reduced while the azimuth resolution is slightly degraded.
The phase and amplitude responses for all reflectors are plotted for the VV channel in \autoref{fig:phase_response_VV}. A linear variation of 30 degrees over the 3dB antenna beamwidth can be observed. The reflectors at 74 m shows a quadratic azimuth phase variation; it was likely too close to the radar, so that it was not illuminated with the complete antenna pattern as the far field transition distance of the slotted array is expected to be of the order of 500 m. This situation could explain its distorted amplitude response as well. 
\begin{figure}[ht]
	\begin{subfigure}[t]{\columnwidth}
		\includegraphics{\chutzepref_BBBl_coreg_phase_plot}
		\subcaption{Azimuth phase ramp not corrected.}
	\end{subfigure}
	\begin{subfigure}[t]{\columnwidth}
		\includegraphics{\chutzepref_BBBl_corr_phase_plot}
		\subcaption{After azimuth filter.}
	\end{subfigure}
	\caption{Relative phase/amplitude response for all reflectors in the calibration array, VV channel. The phase at the azimuth position where the highest intensity is measured has been subtracted in order to show the relative phase variation only. The dotted lines display the theoretical 3 dB resolution of the antennas $\theta_{3dB}$}
	\label{fig:phase_response_VV}
\end{figure}
\subsection{Polarimetric Calibration}\label{sec:res_polcal}
\begin{figure*}[ht]
	\begin{subfigure}[t]{2\columnwidth}
		\includegraphics[scale=2]{\chutzepref_l_cal_gc_pauli}
		\subcaption{Calibrated Pauli RGB Composite.}
		\label{fig:gc:pauli_rgb}
	\end{subfigure}
	\begin{subfigure}[t]{\columnwidth}
		\includegraphics{\chutzepref_l_cal_gc_alpha}
		\subcaption{Calibrated mean Cloude-Pottier $\alpha$ parameter.}
		\label{fig:gc:alpha}
	\end{subfigure}
	\begin{subfigure}[t]{\columnwidth}
		\includegraphics{\chutzepref_l_cal_gc_H}
		\subcaption{Calibrated  Cloude-Pottier $H$ parameter.}
		\label{fig:gc:H}
	\end{subfigure}
	\caption{Calibrated Pauli RGB, entropy and $\alpha$ parameters, geocoded with 1m pixel spacing.}
	\label{fig:gc}
\end{figure*}
The methods described in the preceding sections were applied to prepare SLC images for each channel. For the final polarimetric calibration the procedure of \autoref{sec:proc_polcal} was used; one reflector in the scene was used as a calibration target, with the four remaining reflectors used for the determination of the calibration performance.\\
An initial assessment of the data quality is made by plotting polarization signatures\cite{VanZyl1987} for the  five corner reflectors that were not used to determine the calibration parameters. \autoref{fig:pol_signatures} shows the results for five reflectors both before and after the proposed method.
The signature in the uncorrected data is complex and cannot be easily interpreted as representing any known basic scattering mechanism. In the calibrated response, all reflectors show a good correspondence with the expected polarization signature of a trihedral reflector.\\
As a quantitative assessment of the calibration,  $f$ and $\phi_r + \phi_t$ are re-estimated on the calibrated data for the five reflectors that are not used to estimate the calibration. The results are shown in \autoref{fig:pol_signatures} in the caption under each signature. Generally, both the amplitude and the phase imbalance appear to be well corrected, with a RMS $f$ of 1.0 and $14^\circ$ for $\phi_r + \phi_t$. However, the result is heavily biased by the near and far range reflectors in \autoref{fig:pol_signatures:refl1} and \autoref{fig:pol_signatures:refl5} , where a larger residual phase and amplitude imbalance is still visible. In the case of the reflector at 74 m, the less than ideal response is presumably explained by it being placed too close to the radar compared to the expected far field distance of the antennas, that is in the order of 600 m. For the far reflector, the most likely reason for the poor performance is its placement  close to vegetation and field enclosures and the difficulty of precisely pointing it in elevation and azimuth at large distances from the radar. This is corroborated by the smaller observed RCS compared to the other reflectors and by the noisy amplitude and phase responses in \autoref{fig:phase_response_VV}.\\
Finally, to visualize the result of the calibration, RGB Pauli composites of the scene are shown in \autoref{fig:gc:pauli}. Because very little information can be extracted from this image, a more quantitative understanding of the scene is gained by applying the Cloude-Pottier decomposition on the calibrated data. The data is first converted in cartesian coordinates with 1 meter pixel spacing; it is then multi-looked with a window size of $5 \times 2$ pixels, finally the decomposition parameters are computed from the thus obtained covariance matrix. The $\alpha$ and $H$ parameters are plotted in \autoref{fig:gc:alpha} and \autoref{fig:gc:H} . The response of most natural surfaces appears to be a low entropy, low $\alpha$ scattering usually associated with Bragg/X-Bragg surface scattering. The dominance of surface scattering is attributed to a combination of object roughness  at the scale of the radar wavelength (1.7 cm) and the small penetration depth. A few pixels in the far range area corresponding to the town of M\"{u}nsingen display low entropy dihedral scattering behavior, these points can be likely associated with double bounce interaction between buildings and ground.\\



\def\refI{../outputs/img/\chutzepref_32_519}
\def\refII{../outputs/img/\chutzepref_1032_570}
\def\refIII{../outputs/img/\chutzepref_1051_561}
\def\refIV{../outputs/img/\chutzepref_1331_149}
\def\refV{../outputs/img/\chutzepref_3518_354}

\pgfplotstableread[col sep=comma]{\refI_l_cal_cal_params.csv}{\refIt}
\pgfplotstableread[col sep=comma]{\refII_l_cal_cal_params.csv}{\refIIt}
\pgfplotstableread[col sep=comma]{\refIII_l_cal_cal_params.csv}{\refIIIt}
\pgfplotstableread[col sep=comma]{\refIV_l_cal_cal_params.csv}{\refIVt}
\pgfplotstableread[col sep=comma]{\refV_l_cal_cal_params.csv}{\refVt}
\pgfplotstableset{ref style/.style={
									display columns/0/.style={precision=2, column name=$\phi_r + \phi_t [\circ]$},
									display columns/1/.style={precision=2, column name=$f$},
									display columns/2/.style={precision=1, column name=$\frac{HH}{HV} [dB]$},
									display columns/4/.style={precision=1, column name=$R_{0}$},
									display columns/3/.style={precision=1, column name=$\beta_{0}$}
				}}

\newcommand\signatures[1]{
						\begin{tikzpicture}[]
							\node(copoluncal) at (0,0){\includegraphics{#1_flat_signature}};
							\node[right  =  0.1cm  of copoluncal] (xpoluncal) {\includegraphics{#1_flat_signature_x}};
							\node[below = 0.5 cm of copoluncal] (copolcal) at (0,0){\includegraphics{#1_cal_signature}};
							\node[right  = 0.1cm of copolcal] (xpolcal) {\includegraphics{#1_cal_signature_x}};
							\node[left = 0.1 cm of copoluncal.west, rotate = 90, above = 0.25cm of copoluncal.west ] (uncaltitle) {Uncalibrated};
							\node[left = 0.1 cm of copolcal.west, rotate = 90, above = 0.25cm of copolcal.west] (caltitle) {Calibrated};
						\end{tikzpicture}
					}


\begin{figure*}[hb]
		\begin{subfigure}[t]{\columnwidth}
		\centering
			\signatures{\refI_l}

			\subcaption{Reflector 1\\ 
					\pgfplotstabletypeset[ref style] {\refIt}
					}
			\label{fig:pol_signatures:refl1}
		\end{subfigure}
		\begin{subfigure}[t]{\columnwidth}
			\centering
			\signatures{\refII_l}
			\subcaption{Reflector 2 \\  
						\pgfplotstabletypeset[ref style] {\refIIt}
						}
			\label{fig:pol_signatures:refl2}
		\end{subfigure}
		\begin{subfigure}[t]{\columnwidth}
			\centering
			\signatures{\refIII_l}
			\subcaption{Reflector 3\\ 						
									\pgfplotstabletypeset[ref style] {\refIIIt}
						}
			\label{fig:pol_signatures:refl3}
		\end{subfigure}
		\begin{subfigure}[t]{\columnwidth}
			\centering
			\signatures{\refIV_l}
			\subcaption{Reflector 4\\ 
									\pgfplotstabletypeset[ref style] {\refIVt}
			           }
			\label{fig:pol_signatures:refl4}
		\end{subfigure}\\
\end{figure*}	
\begin{figure*}[H]		
		 \ContinuedFloat
		\begin{subfigure}[t]{\columnwidth}
			\centering
			\signatures{\refV_l}
			\subcaption{Reflector 5\\ 
									\pgfplotstabletypeset[ref style] {\refVt}
        }
			\label{fig:pol_signatures:refl5}
		\end{subfigure}
		\caption{Polarization signatures for the calibration array, acquired using the lower receiver. The top row for each subplot shows the response before the calibration procedure, the bottom one the calibrated response. Each caption contains the calibration parameters re-estimated after the calibration. The imbalances are in degrees, the slant range distance in meters.}
		\label{fig:pol_signatures}
\end{figure*}
\section{Conclusions}\label{sec:conclusions}
In this paper, two main aspects of the calibration of KAPRI, a new polarimetric portable ground-based FMCW radar were discussed:
\begin{enumerate}
	\item The preprocessing of raw data into SLC images, taking into account several effects due to the specific hardware design of the system.
	\item The polarimetric calibration of data into phase and amplitude calibrated polarimetric covariance matrices.
\end{enumerate}
\subsection{Preprocessing}
 The particular antenna design causes a frequency-depending shift of the antenna mainlobe during the chirp that causes a worsened range resolution. It is corrected using a slow time-fast time interpolation procedure; significant range resolution improvement are observed after the correction.\\ The real aperture, azimuth scanning design results in a motion of the antenna phase center relative to the scatterers, causing an observable azimuth phase ramp in point target responses. The variation is significantly different between the antennas, with almost 30$^\circ$ over the 3 dB beamwidth for the V antenna and much smaller for the H unit. This additional phase will complicate polarimetric calibration if left unaltered; this phase ramp is corrected by a SAR-like azimuth filter that reduces the total phase variation to under 10$^\circ$.\\
Because separated transmitting and receiving antennas are used for each polarization, the polarimetric calibration is more intricate due to the presence of an interferometric baseline between channels that adds a topographic phase contribution in the polarimetric phase differences. Using the cross-track interferometric baselines of KAPRI, the topographic contribution can be estimated and subtracted from each element of the covariance matrix affected by it.
\subsection{Calibration}
The resulting flattened covariance matrix is then calibrated  by assuming zero crosstalk and estimating copolar imbalances using a trihedral corner reflector assuming the parameters to be independent from the  incidence angle. The crosspolar imbalance is estimated using distributed targets under the assumption of reciprocity.\\
The calibration quality is assessed by estimating residual calibration parameter on a calibrated scene with five trihedral corner reflectors: the mean amplitude imbalance is close to unity while the mean residual phase imbalance is very close to zero, with an RMS of $7^\circ$; no significant variation with incidence angle is observed. These results suggest that the simplified calibration model\cite{Fore2015,Sarabandi1990} is suitable to calibrate fully polarimetric KAPRI data.
\bibliographystyle{IEEEtran}
\bibliography{library}
\end{document}
